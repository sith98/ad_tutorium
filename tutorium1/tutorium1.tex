\documentclass[11pt,a4paper]{article}

\usepackage{gastex}
\usepackage{etoolbox}

\newcommand{\showLoesung}{1} %<---als Schalter
%\newcommand{\showInhalt}{1} %<---als Schalter

\usepackage{alltt,moreverb,amsmath,enumerate}
\usepackage[normalem]{ulem}
\usepackage[T1]{fontenc}
\usepackage{ae,aecompl} %helvet,mathptm
%\usepackage[left=15mm,right=15mm,top=20mm,bottom=20mm]{geometry}
\usepackage[margin=.5in]{geometry}
%\usepackage[latin1]{inputenc} % f�r Linux
\usepackage[utf8]{inputenc} % Umlaute etc. direkt schreiben (unter Windows)
\usepackage[german]{babel}
\usepackage[url]{oth-logoPNG}
%\usepackage{i2sym,i2ams}

\usepackage{tikz}
\usetikzlibrary{arrows,shapes,trees,positioning,automata,decorations.pathreplacing,decorations.pathmorphing}
\usepackage{tkz-graph}
\usepackage{color}

\usepackage{longtable}
\usepackage{tabularx}

%\usepackage{epic}
%\usepackage{eepic}
\usepackage{comment,ifthen}
\usepackage{../include/todo}

\usepackage[T1]{fontenc}
\usepackage{textcomp}

\usepackage{listings}                   % Listings in Core-Erlang und Maude
\usepackage{lstmisc}

\usepackage{epic}                       % Bildbefehle (picture)
%\usepackage{eepic}                      % erweiterte Bildbefehle

\usepackage{bbm}                        % Mengensymbole (N,C,R,B)
\usepackage{latexsym}                   % zusaetzliche Mathesymbole
\usepackage{amsmath}                    % Mathepaket von der AMS
\usepackage{amstext}
\usepackage{amsfonts}
\usepackage{stmaryrd}                   % zusaetzliche Mathesymbole
\usepackage{mathtools}
\usepackage{amsthm}
\usepackage{cancel}

\usepackage{hyperref}
\usepackage{url}                        % Zum Setzen von URLs in typewriter-face

\pagestyle{empty}

\let\epsilon=\varepsilon
\let\phi=\varphi

\frenchspacing

\setlength{\parindent}{0pt}
\setlength{\textwidth}{18.6cm}
\setlength{\textheight}{26.5cm}
\setlength{\hfuzz}{1mm}

%%% Read dates of assignments from file
\usepackage{xparse}
\ExplSyntaxOn
\ior_new:N \g_hringriin_file_stream

\NewDocumentCommand{\ReadFile}{mm}
 {
  \hringriin_read_file:nn { #1 } { #2 }
  \cs_new:Npn #1 ##1
   {
    \str_if_eq:nnTF { ##1 } { * }
      { \seq_count:c { g_hringriin_file_ \cs_to_str:N #1 _seq } }
      { \seq_item:cn { g_hringriin_file_ \cs_to_str:N #1 _seq } { ##1 } }
   }
 }

\cs_new_protected:Nn \hringriin_read_file:nn
 {
  \ior_open:Nn \g_hringriin_file_stream { #2 }
  \seq_gclear_new:c { g_hringriin_file_ \cs_to_str:N #1 _seq }
  \ior_map_inline:Nn \g_hringriin_file_stream
   {
    \seq_gput_right:cx 
     { g_hringriin_file_ \cs_to_str:N #1 _seq }
     { \tl_trim_spaces:n { ##1 } }
   }
  \ior_close:N \g_hringriin_file_stream
 }

\ExplSyntaxOff

\ReadFile{\uebungsabgabe}{../skel/UEBUNGSABGABE.def}

%%% Read subject info from file
\newcommand{\dozent}[1]{\def\DOZENT{#1}}
\newcommand{\tutoren}[1]{\def\TUTOREN{#1}}
\newcommand{\vorlesung}[1]{\def\VORLESUNG{#1}}
\newcommand{\semester}[1]{\def\SEMESTER{#1}}

\InputIfFileExists{../skel/VORLESUNG.def}{\providecommand{\TUTOREN}{}}%
{\typeout{***********}
 \typeout{Warnung: Kein File vorhanden, das die Vorlesung spezifiziert!}
 \typeout{Spezifikation muss daher im Text des Blattes oder ueber die
          Tastatur erfolgen.}
 \typeout{***********}}

\def\Uebung#1#2#3{
  \othLehrstuhlLogo[\DOZENT]
  \begin{center}
	{~\\[-2em]\Large\bf \VORLESUNG}\\[0.5em]
    \LARGE --~Tutorium #1 (Übung #2)~--\\[4mm]
  \
  \normalsize
  \textbf{#3}
    \rule{\textwidth}{0.1pt}\\[1cm]
  \end{center}
}

\def\Hinweis#1{
	{~\\[-3em]\bf Hinweis: }
	\begin{minipage}[t]{16.5cm}
	#1
	\end{minipage}\\[1em]
    \rule{\textwidth}{0.1pt}
}

\def\Tipps#1{
	{~\\[-3em]\bf Tipps: }
	\begin{minipage}[t]{16.5cm}
	#1
	\end{minipage}\\[1em]
    \rule{\textwidth}{0.1pt}
}
  
\def\MyHeader{
  \othLehrstuhlLogo[Prof.~Dr.~rer.~nat.~Carsten~Kern]%[Carsten~Kern,~Stefan~Rieger]
}

\newcommand{\sem}[1]{[\![#1\,]\!]}

\def\aufgabe#1#2{\subsection*{Aufgabe #1 (#2)}\par}
\def\endaufgabe{}

\newenvironment{loesung}{\subsection*{L\"osungsvorschlag:}}{}
\newenvironment{hinweis}{}{}
\ifthenelse{\isundefined{\showLoesung}}{\excludecomment{loesung}}{\pagestyle{plain}\excludecomment{hinweis}}

\newenvironment{tipps}{}{}
\ifthenelse{\isundefined{\showTipps}}{\excludecomment{tipps}}{\excludecomment{hinweis}}

\newenvironment{inhalt}{\subsection*{Kommentar:}}{}
\ifthenelse{\isundefined{\showInhalt}}{\excludecomment{inhalt}}{}

\long\def\Exercise#1#2{\begin{exercise}{#1}#2\end{exercise}}

\def\underbar#1{%
  \setbox0=\hbox{#1}%
  \dimen0=\dp0\relax%
  \dp0=0pt%
  \setbox0=\hbox{\underline{\box0}}%
  \dp0=\dimen0\relax%
  \box0%
  }

\makeatletter
\def\@makeunderbar[#1]#2{\expandafter\def\csname#1\endcsname{\underbar{#2}}}
\def\makeunderbar{\@ifnextchar[{\@makeunderbar}{\@makeunderbar[]}}
\makeatother

\def\T{\mathrm{T}}
\def\P{\mathrm{P}}
\def\CT{\mathrm{CT}}
\def\COp{\mathrm{COp}}

\makeunderbar{Comp}
\makeunderbar{Ops}
\makeunderbar{trans}
\makeunderbar[strans]{s-trans}
\makeunderbar[ntrans]{n-trans}
\makeunderbar{fix}

\def\labelenumi{\alph{enumi})}
\let\<=\langle
\let\>=\rangle

\parindent=0pt
\parskip=1ex

\definecolor{javared}{rgb}{0.6,0,0} % for strings
\definecolor{javagreen}{rgb}{0.25,0.5,0.35} % comments
\definecolor{javapurple}{rgb}{0.5,0,0.35} % keywords
\definecolor{javadocblue}{rgb}{0.25,0.35,0.75} % javadoc
 
\lstset{language=C++,
basicstyle=\ttfamily\footnotesize,
keywordstyle=\color{javapurple}\bf,
stringstyle=\color{javared},
commentstyle=\color{javagreen}\it\bf,
morecomment=[s][\color{javadocblue}]{/**}{*/},
numbers=left,
numberstyle=\tiny\color{gray},
stepnumber=1,
numbersep=10pt,
tabsize=3,
showspaces=false,
showstringspaces=false}

\usepackage{enumitem}

\begin{document}
\thispagestyle{empty}

\Uebung{1}{2}{14. Oktober 2021}  % FIXME: Blattnummer, Datum, Zeit

%%%%%%%%%%%%%%%%%%%%%%%%%%%%%%%%%%%%%%%%%%%%%%%%%%%%%%%%%%%%%%%%%%%%%%

\begin{aufgabe}{1}{$O$-Notation}
Zeigen oder widerlegen Sie, dass:
\begin{enumerate}
    \item $\operatorname{log}\left(\frac{3^n}{n + 1}\right) = O(n)$
    \item $13n^2 + 35n + 42 = O(n^2)$, indem Sie die Grenzwert-Definition der $O$-Notation verwenden.
    \item $100n + n \log n = O(n)$
    \item $\sum\limits_{i=1}^{n} 2^i = O(2^n)$
    \item $(n + 4)^5 = \Theta(n^5)$, ohne den Ausdruck auszumultiplizieren.
    \item $\log(3n^4) = \Theta(\log n)$
    \item $(n + 1)! = O(n!)$
    \item $n^{2 + \cos(\pi n)} = \Omega(n) = O(n^3)$
    \item $\sum\limits_{i=1}^{n} \frac{n + 1}{2^i - 1} = \Omega(n)$
\end{enumerate}
Zeigen Sie außerdem, dass für beliebige Funktionen $f, g, h \in \mathbb{N} \rightarrow \mathbb{R}^{+}$ gilt:
\begin{enumerate}
    \setcounter{enumi}{9}
    \item $f(n) = O\big(g(n)\big)$ genau dann, wenn $g(n) = \Omega\big(f(n)\big)$
    \item aus $f(n) = O\big(g(n)\big)$ und $g(n) = O\big(h(n)\big)$ folgt $f(n) = O\big(h(n)\big)$ (die $O$-Notation ist transitiv)
\end{enumerate}
\begin{description}
    \item[Hinweis:] Für manche Teilaufgaben ist die geometrische Reihe nützlich: $\sum\limits_{i=0}^n a^i = \frac{1 - a^{n+1}}{1 - a}$ für alle $a \in \mathbb{R} \setminus \{1\}$
\end{description}
\end{aufgabe}
\begin{loesung}
\begin{enumerate}
    \item
    \begin{equation}
        \log\left(\frac{3^n}{n + 1}\right) = \log(3^n) - \log(n + 1)
        \leq \log(3^n) = \log(3) \cdot n = O(n)
    \end{equation}

    \item 
    \begin{align}
        \lim\limits_{n \rightarrow \infty} \frac{13n^2 + 35n + 42}{n^2} &= \lim\limits_{n \rightarrow \infty} \left(13 + \frac{35}{n} + \frac{42}{n^2}\right) \\
        &= 13 + \lim\limits_{n \rightarrow \infty} \frac{35}{n} + \lim\limits_{n \rightarrow \infty} \frac{42}{n^2} = 13 = c
    \end{align}
    \begin{equation}
        \Rightarrow 13n^2 + 35n + 42 = O(n^2)
    \end{equation}

    \item
    Wenn $100n + n \log n = O(n)$, gibt es per Defintion Konstanten $n_0$ und $c$, sodass für alle $n \geq n_0$
    \begin{align}
        100n + n \log n &\leq c\cdot n  \\
        \label{eq:a}
        100 + \log n &\leq c
    \end{align}
    Es gibt kein $n_0$, bei der Ungleichung \ref{eq:a} für ein konstantes $c$ und $n \geq n_0$ zutrifft, da $\lim\limits_{n \rightarrow \infty} \log n = \infty$.
    Also ist $100n + n \log n \neq O(n)$.
    
    \item
    \begin{align}
        \sum\limits_{i=1}^{n} 2^i &\leq \sum\limits_{i=0}^{n} 2^i
        \overset{\text{geometr. Reihe}}{=} \frac{1 - 2^{n+1}}{1 - 2} \\
        &= 2^{n+1} - 1 \leq 2^{n+1} = 2 \cdot 2^n = O(2^n)
    \end{align}

    \item
    \begin{equation}
        (n+4)^5 \geq n^5 = \Omega(n^5)
    \end{equation}
    \begin{equation}
        (n+4)^5 \overset{\text{für }n \geq 4}{\leq} (2n)^5 = 32 \cdot n ^5 = O(n^5)
    \end{equation}
    \begin{equation}
        (n+4)^5 = \Omega(n^5) \wedge (n+4)^5 = O(n^5) \Rightarrow (n+4)^5 = \Theta(n^5)
    \end{equation}

    \item
    \begin{equation}
        \log(3n^4) = \log(3) + \log(n^4) = \log(3) + 4\log(n)
    \end{equation}
    \begin{equation}
        \log(3) + 4\log(n)\geq 4\log(n) = \Omega(\log n) 
    \end{equation}
    Die Logarithmus-Funktion ist monoton steigend. Daher gilt für $n \geq 3$:
    \begin{equation}
        \log(3) + 4\log(n) \leq \log(n) + 4\log(n) = 5 \log(n) = O(\log(n)) 
    \end{equation}
    \begin{equation}
        \log(3n^4) = \Omega(\log n) \wedge \log(3n^4) = O(\log n) \Rightarrow \log(3n^4) = \Theta(\log n)
    \end{equation}
    
    \item
    \begin{equation}
        (n + 1)! \leq c n!
    \end{equation}
    Da $(n + 1)! = (n + 1)n!$, erhält man bei Division durch $n+1$ auf beiden Seiten
    \begin{equation}
        \label{eq:b}
        n + 1 \leq c
    \end{equation}
    Ungleichung~\ref{eq:b} kann für große $n$ und konstantes $c$ nicht zutreffen.
    

    % \item 
    % \begin{align}
    %     \sum\limits_{i = 2}^{n - 1} \left( n - i + 1 \right) &= (n - 1) + (n - 2) + \ldots + 4 + 3 + 2 = \sum\limits_{i=2}^{n - 1} i \\
    %     &= \left(\sum\limits_{i = 1}^{n - 1} i\right) - 1\overset{\text{Gaußsumme}}{=}\frac{n(n - 1)}{2} - 1 \\
    %     &= \frac{1}{2}n^2 - \frac{1}{2}n - 1 \leq \frac{1}{2} n^2 = O(n^2)
    % \end{align}


    \item
    \begin{equation}
        -1 \leq \cos(\pi n) \leq 1
    \end{equation}
    \begin{equation}
        n^{2 + \cos(\pi n)} \geq n^{2 - 1} = n = \Omega(n)
    \end{equation}
    \begin{equation}
        n^{2 + \cos(\pi n)} \leq n^{2 + 1} = n^3 = O(n^3)
    \end{equation}

    \item
    \begin{align}
        \sum\limits_{i=1}^{n} \frac{n + 1}{2^i - 1} &= (n+1)\cdot\sum\limits_{i=1}^{n} \frac{1}{2^i - 1} \geq
        n\cdot\sum\limits_{i=1}^{n} \frac{1}{2^i - 1} 
        \geq n \cdot \sum\limits_{i=1}^{n} \frac{1}{2^i} \\
        &= n \cdot \left(\sum\limits_{i=0}^{n} \frac{1}{2^i} - 1\right)
        \overset{\text{geometr. Reihe}}{=}
        n \cdot  \left(\frac{1 - \left(\frac{1}{2}\right)^{n+1}}{1 - \frac{1}{2}} - 1\right) \\
        &=n \cdot  \left(2\left(1 - \left(\frac{1}{2}\right)^{n + 1}\right) - 1\right)
        \overset{\text{für }n \geq 1}{\geq} n \cdot  \left(2 \left(1 - \frac{1}{4}\right) - 1\right) \\
        &= n \cdot \frac{1}{2}= \Omega(n)
    \end{align}

    \item
    Um zu zeigen, dass 
    \begin{equation}
     f(n) = O\big(g(n)\big) \Leftrightarrow g(n) = \Omega\big(f(n)\big),
    \end{equation}
    reicht es, zu zeigen, dass $f(n) = O\big(g(n)\big) \Rightarrow g(n) = \Omega\big(f(n)\big)$ und $f(n) = O\big(g(n)\big) \Leftarrow g(n) = \Omega\big(f(n)\big)$, also dass jeweils die eine Aussage aus der anderen folgt:
    \begin{itemize}
        \item $\Rightarrow$: Sei $f(n) = O\big(g(n)\big)$, dann gibt es positive Konstanten $n_0$ und $ c$, sodass $f(n) \leq c \cdot g(n)$ für alle $n \geq n_0$.
        Daraus folgt, dass $g(n) \geq \frac{1}{c} f(n)$ für $n \geq n_0$.
        Damit ist per Definition $g(n) = \Omega\big(f(n)\big)$ mit Konstanten $n_0' = n_0$ und $c' = \frac{1}{c}$
        \item $\Leftarrow$: Die Rückrichtung erfolgt analog.
    \end{itemize}

    \item
    Sei $f(n) = O\big(g(n)\big)$ und $g(n) = O\big(h(n)\big)$.
    Dann gilt $f(n) \leq c_f \cdot g(n)$ für $n \geq n_{0,f}$ sowie $g(n) \leq c_g \cdot h(n)$ für $n \geq n_{0,g}$.
    Daraus folgt, dass für jedes $n$ mit $n \geq n_{0,f}$ und $n \geq n_{0,g}$ gilt:
    \begin{equation}
        f(n) \leq c_f \cdot g(n) \leq c_f \cdot c_g \cdot h(n)
    \end{equation}
    Damit ist per Definition $f(n) = O\big(h(n)\big)$ mit Konstanten $n_0 = \max(n_{0,f}, n_{0,g})$ und $c = c_f \cdot c_g$.

    


\end{enumerate}
\end{loesung}


\begin{aufgabe}{2}{Analyse von Algorithmen}
\begin{enumerate}
\item
Implementieren Sie eine Funktion \texttt{duplicates} in einer Programmiersprache Ihrer Wahl, welche ein Array von ganzen Zahlen als Eingabe erhält und als Boolean zurückgibt, ob das Array Duplikate (zwei oder mehr Elemente des gleichen Werts) enthält.

Beispiele: $\operatorname{duplicates}(\left[4, 1, 2, 4\right]) = \texttt{true}$;  $\operatorname{duplicates}(\left[4, 3, 2, 1\right]) = \texttt{false}$; $\operatorname{duplicates}(\left[3, 3, 2, 3\right]) = \texttt{true}$

Achten Sie bei Ihrer Implementierung darauf, dass Sie keine zwei Werte des Arrays mehr als einmal vergleichen.

\item
Wie viele Vergleiche (\texttt{==}, \texttt{<}, \texttt{>=}, $\ldots$) benötigt Ihr Programm im Worst Case, abhängig von der Anzahl der Elemente $n$?

\item Schätzen Sie anhand der Anzahl von Vergleichen die Worst-Case-Laufzeit Ihrer Funktion in $O$-Notation ab.
\item Angenommen, Sie wissen, dass die Werte des Eingabearrays aufsteigend sortiert sind.
Wie können Sie Ihr bestehendes Programm verbessern? Welche Laufzeit hat es nun?

\end{enumerate}
\end{aufgabe}

\begin{loesung}
    
\begin{enumerate}
    \item Beispielcode:
    \begin{lstlisting}
bool duplicates(int[] numbers) {
    for (int i = 0; i < numbers.length; i++) {
        for (int j = i + 1; j < numbers.length; j++) {
            if (numbers[i] == numbers[j]) {
                return true;
            }
        }
    }
    return false;
}
    \end{lstlisting}

    \item
    Der Vergleich \texttt{i < numbers.length} der äußeren \texttt{for}-Schleife wird $n + 1$ mal durchgeführt.
    Die innere Schleife durchläuft für jeden Durchlauf der äußeren Schleife $n - i - 1$ Iterationen. Macht nochmal $n - i$ Vergleiche im \glqq{}\texttt{for}-Kopf\grqq{} und $n - i - 1$ Vergleiche für die \texttt{if}-Bedingung. Die Anzahl der Vergleiche $C(n)$ ist im Worst Case also:
    \begin{align}
        C(n) &= n + 1 + \sum\limits_{i = 0}^{n - 1} \big(2(n - i) - 1 \big)
        = n + 1 + 2\sum\limits_{i = 0}^{n - 1} (n - i) - n \\
        &=1 + 2\sum\limits_{i = 0}^{n - 1} (n - i)
        = 1 + 2\big(n + (n-1) + \ldots + 2 + 1\big) = 1 + 2\sum\limits_{i = 1}^{n} i \\
        &= 1 + 2\left( \frac{n(n + 1)}{2} \right) = 1 + n(n+1) \\
        &= n^2 + n + 1
    \end{align}

    \item 
    \begin{equation}
        T(n) = n^2 + n + 1 \overset{\text{für }n \geq 1}{\leq} n^2 + n^2 + n^2 = 3 n^2 = O(n^2)
    \end{equation}
    
    \item Wenn das Array aufsteigend sortiert ist, können Duplikate nur benachbarte Elemente sein:
\begin{lstlisting}
bool duplicatesSorted(int[] numbers) {
    for (int i = 0; i < numbers.length - 1; i++) {
        if (numbers[i] == numbers[i + 1]) {
            return true;
        }
    }
    return false;
}
\end{lstlisting}
Diese Funktion benötigt im Worst Case $n$ Vergleiche für die \texttt{for}-Schleife und $n - 1$ Vergleiche für die \texttt{if}-Bedingung.
Die Worst-Base-Laufzeit, abgeschätzt anhand der Anzahl von Vergleichen, entspricht also
\begin{equation}
    T(n) = 2n - 1 = O(n).
\end{equation}

\end{enumerate}
\end{loesung}


\begin{aufgabe}{3}{Binäre Suche}
Mittels binärer Suche kann effizient überprüft werden, ob ein Array von Zahlen einen bestimmten Wert enthält.
Vorraussetzung ist, dass die Zahlen aufsteigend sortiert sind.
Dafür wird der gesuchte Wert $x$ mit dem Wert $m$ in der Mitte des Arrays verglichen. Hier gibt es drei Fälle:
\vspace{-3mm}
\begin{itemize}
    \item $x = m$: Der gesuchte Wert wurde gefunden. Fertig!
    \item $x < m$: Der gesuchte Wert befindet sich, falls er existiert, links von $m$. Führe die binäre Suche rekursiv auf der linken Hälfte des Arrays aus.
    \item $x > m$: Analog zu Fall 2, nur auf der rechten Hälfte
\end{itemize}
\begin{enumerate}
    \item Implementieren Sie binäre Suche als rekursiven Algorithmus, vergessen Sie nicht den Basisfall.
    \item Konvertieren Sie Ihre Implementierung in einen iterativen Algorithmus.
    \item Stellen Sie für den Worst Case die Rekursionsgleichung auf. Sie müssen diese Gleichung nicht lösen.
\end{enumerate}
    
\end{aufgabe}

\ifcsdef{showLoesung}{\newpage}{}
\begin{loesung}
\begin{enumerate}
    \item rekursive Implementierung (Parameter \texttt{start} ist inklusiv, \texttt{end} exklusiv):
    \begin{lstlisting}
bool binarySearchRec(int[] numbers, int value) {
    return _binarySearchRec(numbers, 0, numbers.length, value);
}
bool _binarySearchRec(int[] numbers, int start, int end, int value) {
    if (end - start <= 0) {
        return false;
    }
    int middle = (start + end) / 2;
    if (value == numbers[middle]) {
        return true;
    }
    if (value < numbers[middle]) {
        return _binarySearchRec(numbers, start, middle, value);
    } else {
        return _binarySearchRec(numbers, middle + 1, end, value);
    }
}
    \end{lstlisting}
    \item Da sich der rekursive Aufruf in Tail-Position befindet (letzte Anweisung vor dem \texttt{return}), reicht es für die iterative Implementierung, Parameter \texttt{start} und \texttt{end} jeweils für den nächsten Schleifendurchlauf zu aktualisieren.
    Verwalten eines Stapels für die Aufrufe ist nicht notwendig.
    \begin{lstlisting}
bool binarySearchIt(int[] numbers, int value) {
    int start = 0;
    int end = numbers.length;
    while (true) {
        if (end - start <= 0) {
            return false;
        }
        int middle = (start + end) / 2;
        if (value == numbers[middle]) {
            return true;
        }
        if (value < numbers[middle]) {
            end = middle;
        } else {
            start = middle + 1;
        }
    }
}
    \end{lstlisting}
    \item 
    \begin{equation}
        T(n) = \begin{cases}
            1 & \text{für $n \leq 1$} \\
            T(\left\lfloor \frac{n}{2} \right\rfloor) + 1  & \text{für $n > 1$} \\
        \end{cases}
    \end{equation}
\end{enumerate}

\end{loesung}


\begin{aufgabe}{4}{Registermaschinensimulator}
\begin{enumerate}[label=\alph*)]
    \item Implementieren Sie ein Registermaschinenprogramm, welches eine Zahl $n \in \mathbb{N}$ einliest und alle Fibonacci-Zahlen $\operatorname{fib}(i)$ mit $\operatorname{fib}(i) \leq n$ ausgibt. Bei $n = 20$ soll das Programm also die Zahlen $(1, 1, 2, 3, 5, 8, 13)$ aus\-ge\-ben; Bei Eingabe 1 nur $(1, 1)$.
    Zur Definition der Fibonacci-Zahlen siehe Vorlesung 4, Folie 12.

    \item \textbf{Bonusaufgabe (schwer):} Erweitern Sie Ihren Registermaschinensimulator, sodass dieser auch indirekte Adressierung untersützt.
    Fügen Sie dafür folgende Befehle hinzu:
    \begin{table}[h!]
        \centering
        \begin{tabular}{|l|l|}
        \hline
        \textbf{Befehl} & \textbf{Semantik} \\ \hline
        \texttt{LDI} & $f(0) = f\big(f(\mathrm{adresse})\big)$ \\ \hline
        \texttt{STI} & $f\big(f(\mathrm{adresse})\big) = f(0)$ \\ \hline
        \end{tabular}
    \end{table}

    Implementieren Sie anschließend das Sieb des Eratosthenes auf der Registermaschine.
    Das heißt, schreiben Sie ein Programm, welches eine Zahl $n \geq 2$ einliest und alle Primzahlen $p_i$ mit $p_i \leq n$ ausgibt.
\end{enumerate}
\end{aufgabe}

\begin{loesung}
\begin{enumerate}[label=\alph*)]
\item Speicherbelegung: $1 \rightarrow n, 2 \rightarrow a, 3 \rightarrow b, 4 \rightarrow c$
\begin{lstlisting}
01 INP 01 // Eingabe: n
02 LDK 01
03 STA 02 // a = 1
04 STA 03 // b = 1
05 LDA 02 
06 SUB 01
07 JLE 09 // while (a <= n)
08 HLT 00 // Schleifenabbruch
09 OUT 02 // print(a)
10 LDA 02
11 ADD 03
12 STA 04 // c = a + b
14 LDA 03
15 STA 02 // a = b
16 LDA 04
17 STA 03 // b = c
18 JMP 05 // Schleifenende
\end{lstlisting}
\item Speicherzellen 1 bis 3 werden für Eingabe und Zählervariablen verwendet ($n$, $i$ und $j$).
Zelle 4 wird temporär für den indirekten Speicherzugriff verwendet.
Das \glqq{}Boolean-Array\grqq{} für den Sieb beginnt ab Zelle 5 beginnend mit dem Wert für 2. 
Der Wert für Zahl $i$ steht also an Adresse $i + 3$.
0 steht dabei für prim, 1 für nicht-prim.
\begin{lstlisting}
01 INP 01 // Eingabe: n
02 LDK 02
03 STA 02 // i = 2
04 LDK 03 // while (true) {
05 ADD 02
06 STA 04
07 LDI 04
08 JEZ 16 // if (arr[i] == 0) subroutine()
09 LDK 01
10 ADD 02
11 STA 02 // i = i + 1
12 LDA 02
13 SUB 01 
14 JLE 04 // if (i <= n) continue
15 HLT 00 // return }
16 OUT 02 // function subroutine() { print(i)
17 LDA 02
18 ADD 02
19 STA 03 // j = i + i
20 LDA 03 // while (true) {
21 SUB 01
22 JGZ 09 // if (j > n) return
23 LDK 03
24 ADD 03
25 STA 04
26 LDK 01
27 STI 04 // arr[j] = 1
28 LDA 03
29 ADD 02
30 STA 03 // j = j + i
31 JMP 20 // } }
\end{lstlisting}
\end{enumerate}
\end{loesung}

\end{document}
