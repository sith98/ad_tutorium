\documentclass[11pt,a4paper]{article}

\usepackage{gastex}
\usepackage{etoolbox}
% \newcommand{\showLoesung}{2} %<---als Schalter
% \newcommand{\showInhalt}{1} %<---als Schalter

\usepackage{alltt,moreverb,amsmath,enumerate}
\usepackage[normalem]{ulem}
\usepackage[T1]{fontenc}
\usepackage{ae,aecompl} %helvet,mathptm
%\usepackage[left=15mm,right=15mm,top=20mm,bottom=20mm]{geometry}
\usepackage[margin=.5in]{geometry}
%\usepackage[latin1]{inputenc} % f�r Linux
\usepackage[utf8]{inputenc} % Umlaute etc. direkt schreiben (unter Windows)
\usepackage[german]{babel}
\usepackage[url]{oth-logoPNG}
%\usepackage{i2sym,i2ams}

\usepackage{tikz}
\usetikzlibrary{arrows,shapes,trees,positioning,automata,decorations.pathreplacing,decorations.pathmorphing}
\usepackage{tkz-graph}
\usepackage{color}

\usepackage{longtable}
\usepackage{tabularx}

%\usepackage{epic}
%\usepackage{eepic}
\usepackage{comment,ifthen}
\usepackage{../include/todo}

\usepackage[T1]{fontenc}
\usepackage{textcomp}

\usepackage{listings}                   % Listings in Core-Erlang und Maude
\usepackage{lstmisc}

\usepackage{epic}                       % Bildbefehle (picture)
%\usepackage{eepic}                      % erweiterte Bildbefehle

\usepackage{bbm}                        % Mengensymbole (N,C,R,B)
\usepackage{latexsym}                   % zusaetzliche Mathesymbole
\usepackage{amsmath}                    % Mathepaket von der AMS
\usepackage{amstext}
\usepackage{amsfonts}
\usepackage{stmaryrd}                   % zusaetzliche Mathesymbole
\usepackage{mathtools}
\usepackage{amsthm}
\usepackage{cancel}

\usepackage{hyperref}
\usepackage{url}                        % Zum Setzen von URLs in typewriter-face

\pagestyle{empty}

\let\epsilon=\varepsilon
\let\phi=\varphi

\frenchspacing

\setlength{\parindent}{0pt}
\setlength{\textwidth}{18.6cm}
\setlength{\textheight}{26.5cm}
\setlength{\hfuzz}{1mm}

%%% Read dates of assignments from file
\usepackage{xparse}
\ExplSyntaxOn
\ior_new:N \g_hringriin_file_stream

\NewDocumentCommand{\ReadFile}{mm}
 {
  \hringriin_read_file:nn { #1 } { #2 }
  \cs_new:Npn #1 ##1
   {
    \str_if_eq:nnTF { ##1 } { * }
      { \seq_count:c { g_hringriin_file_ \cs_to_str:N #1 _seq } }
      { \seq_item:cn { g_hringriin_file_ \cs_to_str:N #1 _seq } { ##1 } }
   }
 }

\cs_new_protected:Nn \hringriin_read_file:nn
 {
  \ior_open:Nn \g_hringriin_file_stream { #2 }
  \seq_gclear_new:c { g_hringriin_file_ \cs_to_str:N #1 _seq }
  \ior_map_inline:Nn \g_hringriin_file_stream
   {
    \seq_gput_right:cx 
     { g_hringriin_file_ \cs_to_str:N #1 _seq }
     { \tl_trim_spaces:n { ##1 } }
   }
  \ior_close:N \g_hringriin_file_stream
 }

\ExplSyntaxOff

\ReadFile{\uebungsabgabe}{../skel/UEBUNGSABGABE.def}

%%% Read subject info from file
\newcommand{\dozent}[1]{\def\DOZENT{#1}}
\newcommand{\tutoren}[1]{\def\TUTOREN{#1}}
\newcommand{\vorlesung}[1]{\def\VORLESUNG{#1}}
\newcommand{\semester}[1]{\def\SEMESTER{#1}}

\InputIfFileExists{../skel/VORLESUNG.def}{\providecommand{\TUTOREN}{}}%
{\typeout{***********}
 \typeout{Warnung: Kein File vorhanden, das die Vorlesung spezifiziert!}
 \typeout{Spezifikation muss daher im Text des Blattes oder ueber die
          Tastatur erfolgen.}
 \typeout{***********}}

\def\Uebung#1#2#3{
  \othLehrstuhlLogo[\DOZENT]
  \begin{center}
	{~\\[-2em]\Large\bf \VORLESUNG}\\[0.5em]
    \LARGE --~Tutorium #1 (Übung #2)~--\\[4mm]
  \
  \normalsize
  \textbf{#3}
    \rule{\textwidth}{0.1pt}\\[1cm]
  \end{center}
}

\def\Hinweis#1{
	{~\\[-3em]\bf Hinweis: }
	\begin{minipage}[t]{16.5cm}
	#1
	\end{minipage}\\[1em]
    \rule{\textwidth}{0.1pt}
}

\def\Tipps#1{
	{~\\[-3em]\bf Tipps: }
	\begin{minipage}[t]{16.5cm}
	#1
	\end{minipage}\\[1em]
    \rule{\textwidth}{0.1pt}
}
  
\def\MyHeader{
  \othLehrstuhlLogo[Prof.~Dr.~rer.~nat.~Carsten~Kern]%[Carsten~Kern,~Stefan~Rieger]
}

\newcommand{\sem}[1]{[\![#1\,]\!]}

\def\aufgabe#1#2{\subsection*{Aufgabe #1 (#2)}\par}
\def\endaufgabe{}

\newenvironment{loesung}{\subsection*{L\"osungsvorschlag:}}{}
\newenvironment{hinweis}{}{}
\ifthenelse{\isundefined{\showLoesung}}{\excludecomment{loesung}}{\pagestyle{plain}\excludecomment{hinweis}}

\newenvironment{tipps}{}{}
\ifthenelse{\isundefined{\showTipps}}{\excludecomment{tipps}}{\excludecomment{hinweis}}

\newenvironment{inhalt}{\subsection*{Kommentar:}}{}
\ifthenelse{\isundefined{\showInhalt}}{\excludecomment{inhalt}}{}

\long\def\Exercise#1#2{\begin{exercise}{#1}#2\end{exercise}}

\def\underbar#1{%
  \setbox0=\hbox{#1}%
  \dimen0=\dp0\relax%
  \dp0=0pt%
  \setbox0=\hbox{\underline{\box0}}%
  \dp0=\dimen0\relax%
  \box0%
  }

\makeatletter
\def\@makeunderbar[#1]#2{\expandafter\def\csname#1\endcsname{\underbar{#2}}}
\def\makeunderbar{\@ifnextchar[{\@makeunderbar}{\@makeunderbar[]}}
\makeatother

\def\T{\mathrm{T}}
\def\P{\mathrm{P}}
\def\CT{\mathrm{CT}}
\def\COp{\mathrm{COp}}

\makeunderbar{Comp}
\makeunderbar{Ops}
\makeunderbar{trans}
\makeunderbar[strans]{s-trans}
\makeunderbar[ntrans]{n-trans}
\makeunderbar{fix}

\def\labelenumi{\alph{enumi})}
\let\<=\langle
\let\>=\rangle

\parindent=0pt
\parskip=1ex

\definecolor{javared}{rgb}{0.6,0,0} % for strings
\definecolor{javagreen}{rgb}{0.25,0.5,0.35} % comments
\definecolor{javapurple}{rgb}{0.5,0,0.35} % keywords
\definecolor{javadocblue}{rgb}{0.25,0.35,0.75} % javadoc
 
\lstset{language=C++,
basicstyle=\ttfamily\footnotesize,
keywordstyle=\color{javapurple}\bf,
stringstyle=\color{javared},
commentstyle=\color{javagreen}\it\bf,
morecomment=[s][\color{javadocblue}]{/**}{*/},
numbers=left,
numberstyle=\tiny\color{gray},
stepnumber=1,
numbersep=10pt,
tabsize=3,
showspaces=false,
showstringspaces=false}

\usepackage{enumitem}
\usepackage{algpseudocode}
\usepackage{caption}
\usepackage{subcaption}
\usepackage{placeins}
\usepackage{multicol}
\usepackage{slashbox}
\usepackage{fancyvrb}
\usepackage{ulem}
\usepackage{amssymb}

\begin{document}
\thispagestyle{empty}
\DeclareRobustCommand{\ttfamily}{\fontencoding{T1}\fontfamily{lmtt}\selectfont}

\newcommand{\quotes}[1]{\glqq{}#1\grqq{}}

\Uebung{12}{13}{Simon Thelen}{20. Januar 2021}  % FIXME: Blattnummer, Datum, Zeit

%%%%%%%%%%%%%%%%%%%%%%%%%%%%%%%%%%%%%%%%%%%%%%%%%%%%%%%%%%%%%%%%%%%%%%

\ifcsdef{showLoesung}{
\textbf{Bitte beachten Sie:} Die Lösungen können trotz sorgfältiger Prüfung Fehler enthalten.
Bei Fragen oder Unklarheiten kontaktieren Sie bitte den Tutor oder Dozenten in Tutorien, Übungen oder nach Vorlesungen.
}{}

\begin{aufgabe}{3}{Dynamische Programmierung}
    Gegeben sei ein Array $(a_1, a_2, \ldots, a_n)$ mit $a_i \in \mathbb{Z}$ für $1 \leq i \leq n$.
    Eine aufsteigende Subsequenz des Arrays ist definiert als eine Folge $s_1, s_2, \ldots, s_{k - 1}, s_k$ mit $1 \leq s_1 < s_2 < \ldots < s_k \leq n$ und $a_{s_1} < a_{s_2} < \ldots < a_{s_k}$.
    Gesucht ist die längste, aufsteigende Subsequenz.
    \begin{description}
        \item[Beispiel:] Eingabe: $(7, 12, 4, 8, 6, 8, 9, 8, 11)$ $\rightarrow$ Ausgabe: $(3, 5, 6, 7, 9)$ mit $k = 5$. Dies entspricht den Elementen~$(4, 6, 8, 9, 11)$ des Arrays.
    \end{description}
    \begin{enumerate}[label=\alph*)]
        \item\label{optimal_substructure} Beschreiben Sie kurz, inwiefern die längste, aufsteigende Subsequenz die Eigenschaften einer optimalen Substruktur besitzt.
        \item\label{dynamic} Implementieren Sie einen Algorithmus in Pseudocode oder einer Programmiersprache Ihrer Wahl nach dem Prinzip der dynamischen Programmierung mit Laufzeit $O(n^2)$, welcher $k$, die Länge der längsten, aufsteigenden Subsequenz $s_1, \ldots, s_k$ ausgibt.
        \begin{description}
            \item[Tipp:] Berechnen Sie für jedes $i$ ($1 \leq i \leq n$) die Länge der längsten, aufsteigenden Subsequenz die bei $a_i$ endet ($s_k = i$) und geben Sie abschießend das Maximum dieser Werte zurück.
        \end{description}
        \item Erweitern Sie Ihren Algorithmus aus Teilaufgabe~\ref*{dynamic}, sodass dieser neben der Länge $k$ der Sequenz $a_{s_1}, a_{s_2}, \ldots, a_{s_k}$ auch die Sequenz selbst ausgibt, ohne die Laufzeit im $O$-Kalkül zu verschlechtern.
        \item Wie können Sie das obige Problem als Graphproblem formulieren, wenn Sie jeden Wert $a_i$ des Arrays als Knoten interpretieren und sie zwischen $a_i$ und $a_j$ immer dann eine Kante ziehen, wenn $a_i < a_j$.
        Nennen Sie einen Algorithmus aus der Vorlesung, mit dem Sie das entsprechende Graphproblem lösen können.
    \end{enumerate}
\end{aufgabe}
\begin{loesung}
    \begin{enumerate}
        \item Es sei eine längste, aufsteigende Subsequenz $s_1, s_2, \ldots, s_i, \ldots, s_j, \ldots, s_{k - 1}, s_k$ gegeben.
        Nun wird die Teilsequenz $s_i, s_{i + 1}, \ldots s_{j - 1}, s_j$ betrachtet.
        Diese Teilsequenz ist eine längste, aufsteigende Subsequenz des Teilarrays $a_{s_i}, a_{s_i + 1}, \ldots, a_{s_j - 1}, a_{s_j}$ unter den Nebenbedingungen, dass $a_{s_{i - 1}} < a_{s_i}$ und $a_{s_j} < a_{s_{j + 1}}$.
        Gäbe es nämlich eine längere, aufsteigende Subsequenz in diesem Teilarray unter obigen Nebenbedingungen, könnte man diese (analog zum Beweis der optimalen Substruktur kürzester Wege) in die ursprüngliche, längste, aufsteigende Subsequenz $s_1, \ldots, s_k$ einsetzen und diese damit verlängern.

        Man kann das betrachtete Teilarray sogar auf $a_{s_{i - 1} + 1}, a_{s_{i - 1} + 2}, \ldots a_{s_i}, \ldots, a_{s_j}, \ldots, a_{s_{j + 1} - 2}, a_{s_{j + 1} - 1}$ ausweiten.
        $s_i, \ldots, s_j$ ist im erweitereten Teilarray weiterhin eine längste, aufsteigende Subsequenz (unter besagten Nebenbedigungen).

        \item 
        Das Array $a$ wird von links nach rechts durchlaufen. Für jedes Element $a_i$ wird die Länge $l_i$ der längsten, aufsteigenden Subsequenz bestimmt, die bei $a_i$ endet (siehe Tipp).
        Diese Längen werden in einem temporären Array gespeichert.
        Um $l_i$ zu bestimmen, können so die Längen bei allen vorherigen Elemente ($l_1, l_2, \ldots, l_{i-1}$) verwendet werden.
        
        Für die längste, aufsteigende Subsequenz, die bei $a_i$ endet, gibt es zwei Möglichkeiten:
        Entweder enthält sie nur das Element $a_i$ und besitzt somit Länge 1.
        Oder sie ist länger.
        In diesem Fall enthält sie zusätzlich weitere Elemente aus der Menge $\{a_1, a_2, \ldots, a_{i - 1}\}$.
        Seien dies die Elemente $(a_{s_1}, a_{s_2}, \ldots, a_{s_l})$.
        Aufgrund der Eigentschaften aufsteigender Subsequenzen gilt dann $1 \leq s_1 < s_2 < \ldots < s_l < i$ sowie außerdem $a_{s_1} < a_{s_2} < \ldots < a_{s_l} < a_i$.
        Da längste, aufsteigende Subsequenzen eine optimale Substruktur bilden (siehe Teilaufgabe \ref*{optimal_substructure}), muss $s_1, s_2, \ldots, s_l$ die längste, aufsteigende Subsequenz im Bereich $[1, i - 1]$ sein unter der zusätzlichen Einschränkung, dass $a_{s_l} < a_i$.

        Die Länge $l$ dieser Subsequenz entspricht $\max\{l_j \mid j \in \{1, 2, \ldots, i - 1\}, a_j < a_i\}$.
        Es gilt dann also $l_i = l + 1$.
        Für jedes $i$ kann man also $l_i$ leicht bestimmen, indem man über alle $l_1, l_2, \ldots, l_{i - 1}$ iteriert und das Maximium aller $l_j$ bestimmt, für die $a_j < a_i$ gilt.
        Schließlich muss man nur noch $k = \max\{l_i | i \in \{1, 2, \ldots, n\}\}$ bestimmen.
        \begin{lstlisting}[language=c++]
int longestIncreasingSubsequence(int a[], int n) {
    int *seqLen = new int[n];
    int k = 1;
    for (int i = 0; i < n; i++) {
        seqLen[i] = 1;
    }
    for (int i = 1; i < n; i++) {
        for (int j = 0; j < i; j++) {
            if (a[j] < a[i]) {
                seqLen[i] = max(seqLen[i], seqLen[j] + 1);
                k = max(k, seqLen[i]);
            }
        }
    }
    return k;
} 
        \end{lstlisting}
        Die Laufzeit wird dominiert durch die beiden verschachtelten \texttt{for}-Schleifen.
        Deren Laufzeit ergibt sich durch die Gaußsumme: $\Theta(n^2)$.
        \item Jedes Mal, wenn ein $l_i$ aktualisiert wird, muss zusätzlich gespeichert werden welches $a_j$ ($1 \leq j < i$) vor $a_i$ in der längsten, bei $a_i$ endenden, aufsteigenden Subsequenz steht.
        Dies geschiet im Beispeilcode unten mithilfe des Hilfsarrays \texttt{prev}.
        Jedes Mal, wenn man eine neue, längste, aufsteigende Subsequenz findet ($l_i > k$), aktualisiert man nicht nur $k$, sondern merkt sich auch das $i$, bei dem das $k$ erreicht wurde.
        Auf diese Weise kennt man, nachdem alle $l_i$ bestimmt wurden, die Endposition der längsten, aufsteigenden Subsequenz.
        Außerdem kann man mittels \texttt{prev} die gesamte Sequenz von hinten nach vorne rekonstruieren.
        \begin{lstlisting}[language=c++]
int longestIncreasingSubsequenceExt(int a[], int n, int **solP) {
    int *seqLen = new int[n];
    int *prev = new int[n];
    int k = 1;
    int kIndex = 0;
    for (int i = 0; i < n; i++) {
        seqLen[i] = 1;
        prev[i] = -1;
    }
    for (int i = 1; i < n; i++) {
        for (int j = 0; j < i; j++) {
            if (a[j] < a[i] && seqLen[j] + 1 > seqLen[i]) {
                seqLen[i] = seqLen[j] + 1;
                prev[i] = j;
                if (seqLen[i] > k) {
                    k = seqLen[i];
                    kIndex = i;
                }
            }
        }
    }
    int *sol = new int[k];
    int j = kIndex;
    for (int i = k - 1; i >= 0; i--) {
        sol[i] = a[j];
        j = prev[j];
    }
    *solP = sol;
    return k;
}
        \end{lstlisting}
        Im Wesentlichen kommt im Vergleich zum Ursprungsalgorithmus eine weitere \texttt{for}-Schleife hinzu, die $k \leq n = O(n)$ Iterationen durchläuft.
        Die Laufzeit $O(n^2)$ ändert sich also nicht.

        \item Wird ein Graph gemäß der Aufgabenstellung erstellt, entspricht die längste, aufsteigende Subsequenz dem längsten Pfad im Graphen.
        Diesen längsten Pfad kann man beispielsweise mittels Floyd-Warshall finden.
        Dafür legt man für alle $(u, v) \in E$ das Kantengewicht $w(u, v) = -1$ fest.
        Anschließend sucht man alle kürzesten Wege im Graphen mittels Floyd-Warshall. Der kürzeste Weg gemäß obiger Gewichte entspricht dem längsten Weg nach Kantenanzahl und somit der längsten, aufsteigenden Subsequenz.
        Die Laufzeit wäre aber mit $O(n^3)$ schlechter als die des in den letzten beiden Teilaufgaben entwickelten Algorithmus ($O(n^2)$).
    \end{enumerate}
\end{loesung}

\end{document}