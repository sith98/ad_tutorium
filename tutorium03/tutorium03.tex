\documentclass[11pt,a4paper]{article}

\usepackage{gastex}
\usepackage{etoolbox}

% \newcommand{\showLoesung}{2} %<---als Schalter
%\newcommand{\showInhalt}{1} %<---als Schalter

\usepackage{alltt,moreverb,amsmath,enumerate}
\usepackage[normalem]{ulem}
\usepackage[T1]{fontenc}
\usepackage{ae,aecompl} %helvet,mathptm
%\usepackage[left=15mm,right=15mm,top=20mm,bottom=20mm]{geometry}
\usepackage[margin=.5in]{geometry}
%\usepackage[latin1]{inputenc} % f�r Linux
\usepackage[utf8]{inputenc} % Umlaute etc. direkt schreiben (unter Windows)
\usepackage[german]{babel}
\usepackage[url]{oth-logoPNG}
%\usepackage{i2sym,i2ams}

\usepackage{tikz}
\usetikzlibrary{arrows,shapes,trees,positioning,automata,decorations.pathreplacing,decorations.pathmorphing}
\usepackage{tkz-graph}
\usepackage{color}

\usepackage{longtable}
\usepackage{tabularx}

%\usepackage{epic}
%\usepackage{eepic}
\usepackage{comment,ifthen}
\usepackage{../include/todo}

\usepackage[T1]{fontenc}
\usepackage{textcomp}

\usepackage{listings}                   % Listings in Core-Erlang und Maude
\usepackage{lstmisc}

\usepackage{epic}                       % Bildbefehle (picture)
%\usepackage{eepic}                      % erweiterte Bildbefehle

\usepackage{bbm}                        % Mengensymbole (N,C,R,B)
\usepackage{latexsym}                   % zusaetzliche Mathesymbole
\usepackage{amsmath}                    % Mathepaket von der AMS
\usepackage{amstext}
\usepackage{amsfonts}
\usepackage{stmaryrd}                   % zusaetzliche Mathesymbole
\usepackage{mathtools}
\usepackage{amsthm}
\usepackage{cancel}

\usepackage{hyperref}
\usepackage{url}                        % Zum Setzen von URLs in typewriter-face

\pagestyle{empty}

\let\epsilon=\varepsilon
\let\phi=\varphi

\frenchspacing

\setlength{\parindent}{0pt}
\setlength{\textwidth}{18.6cm}
\setlength{\textheight}{26.5cm}
\setlength{\hfuzz}{1mm}

%%% Read dates of assignments from file
\usepackage{xparse}
\ExplSyntaxOn
\ior_new:N \g_hringriin_file_stream

\NewDocumentCommand{\ReadFile}{mm}
 {
  \hringriin_read_file:nn { #1 } { #2 }
  \cs_new:Npn #1 ##1
   {
    \str_if_eq:nnTF { ##1 } { * }
      { \seq_count:c { g_hringriin_file_ \cs_to_str:N #1 _seq } }
      { \seq_item:cn { g_hringriin_file_ \cs_to_str:N #1 _seq } { ##1 } }
   }
 }

\cs_new_protected:Nn \hringriin_read_file:nn
 {
  \ior_open:Nn \g_hringriin_file_stream { #2 }
  \seq_gclear_new:c { g_hringriin_file_ \cs_to_str:N #1 _seq }
  \ior_map_inline:Nn \g_hringriin_file_stream
   {
    \seq_gput_right:cx 
     { g_hringriin_file_ \cs_to_str:N #1 _seq }
     { \tl_trim_spaces:n { ##1 } }
   }
  \ior_close:N \g_hringriin_file_stream
 }

\ExplSyntaxOff

\ReadFile{\uebungsabgabe}{../skel/UEBUNGSABGABE.def}

%%% Read subject info from file
\newcommand{\dozent}[1]{\def\DOZENT{#1}}
\newcommand{\tutoren}[1]{\def\TUTOREN{#1}}
\newcommand{\vorlesung}[1]{\def\VORLESUNG{#1}}
\newcommand{\semester}[1]{\def\SEMESTER{#1}}

\InputIfFileExists{../skel/VORLESUNG.def}{\providecommand{\TUTOREN}{}}%
{\typeout{***********}
 \typeout{Warnung: Kein File vorhanden, das die Vorlesung spezifiziert!}
 \typeout{Spezifikation muss daher im Text des Blattes oder ueber die
          Tastatur erfolgen.}
 \typeout{***********}}

\def\Uebung#1#2#3{
  \othLehrstuhlLogo[\DOZENT]
  \begin{center}
	{~\\[-2em]\Large\bf \VORLESUNG}\\[0.5em]
    \LARGE --~Tutorium #1 (Übung #2)~--\\[4mm]
  \
  \normalsize
  \textbf{#3}
    \rule{\textwidth}{0.1pt}\\[1cm]
  \end{center}
}

\def\Hinweis#1{
	{~\\[-3em]\bf Hinweis: }
	\begin{minipage}[t]{16.5cm}
	#1
	\end{minipage}\\[1em]
    \rule{\textwidth}{0.1pt}
}

\def\Tipps#1{
	{~\\[-3em]\bf Tipps: }
	\begin{minipage}[t]{16.5cm}
	#1
	\end{minipage}\\[1em]
    \rule{\textwidth}{0.1pt}
}
  
\def\MyHeader{
  \othLehrstuhlLogo[Prof.~Dr.~rer.~nat.~Carsten~Kern]%[Carsten~Kern,~Stefan~Rieger]
}

\newcommand{\sem}[1]{[\![#1\,]\!]}

\def\aufgabe#1#2{\subsection*{Aufgabe #1 (#2)}\par}
\def\endaufgabe{}

\newenvironment{loesung}{\subsection*{L\"osungsvorschlag:}}{}
\newenvironment{hinweis}{}{}
\ifthenelse{\isundefined{\showLoesung}}{\excludecomment{loesung}}{\pagestyle{plain}\excludecomment{hinweis}}

\newenvironment{tipps}{}{}
\ifthenelse{\isundefined{\showTipps}}{\excludecomment{tipps}}{\excludecomment{hinweis}}

\newenvironment{inhalt}{\subsection*{Kommentar:}}{}
\ifthenelse{\isundefined{\showInhalt}}{\excludecomment{inhalt}}{}

\long\def\Exercise#1#2{\begin{exercise}{#1}#2\end{exercise}}

\def\underbar#1{%
  \setbox0=\hbox{#1}%
  \dimen0=\dp0\relax%
  \dp0=0pt%
  \setbox0=\hbox{\underline{\box0}}%
  \dp0=\dimen0\relax%
  \box0%
  }

\makeatletter
\def\@makeunderbar[#1]#2{\expandafter\def\csname#1\endcsname{\underbar{#2}}}
\def\makeunderbar{\@ifnextchar[{\@makeunderbar}{\@makeunderbar[]}}
\makeatother

\def\T{\mathrm{T}}
\def\P{\mathrm{P}}
\def\CT{\mathrm{CT}}
\def\COp{\mathrm{COp}}

\makeunderbar{Comp}
\makeunderbar{Ops}
\makeunderbar{trans}
\makeunderbar[strans]{s-trans}
\makeunderbar[ntrans]{n-trans}
\makeunderbar{fix}

\def\labelenumi{\alph{enumi})}
\let\<=\langle
\let\>=\rangle

\parindent=0pt
\parskip=1ex

\definecolor{javared}{rgb}{0.6,0,0} % for strings
\definecolor{javagreen}{rgb}{0.25,0.5,0.35} % comments
\definecolor{javapurple}{rgb}{0.5,0,0.35} % keywords
\definecolor{javadocblue}{rgb}{0.25,0.35,0.75} % javadoc
 
\lstset{language=Java,
basicstyle=\ttfamily\footnotesize,
keywordstyle=\color{javapurple}\bf,
stringstyle=\color{javared},
commentstyle=\color{javagreen}\it\bf,
morecomment=[s][\color{javadocblue}]{/**}{*/},
numbers=left,
numberstyle=\tiny\color{gray},
stepnumber=1,
numbersep=10pt,
tabsize=3,
showspaces=false,
showstringspaces=false}

\usepackage{enumitem}

\begin{document}
\thispagestyle{empty}

\Uebung{3}{4}{Simon Thelen}{28. Oktober 2021}  % FIXME: Blattnummer, Datum, Zeit
\ifcsdef{showLoesung}{
\textbf{Bitte beachten Sie:} Die Lösungen können trotz sorgfältiger Prüfung Fehler enthalten.
Bei Fragen oder Unklarheiten kontaktieren Sie bitte den Tutor oder Dozenten in Tutorien, Übungen oder nach Vorlesungen.
}{}



%%%%%%%%%%%%%%%%%%%%%%%%%%%%%%%%%%%%%%%%%%%%%%%%%%%%%%%%%%%%%%%%%%%%%%
\begin{aufgabe}{1}{Anpassen von Sortieralgorithmen}
    \begin{enumerate}
        \item Gegeben sei folgende Funktion:
        \begin{lstlisting}
void swap(int[] arr, int i, int j) {
    int temp = arr[i];
    arr[i] = arr[j];
    arr[j] = temp;
} 
        \end{lstlisting}
        Verändern Sie Ihre Implementierung von \textbf{Insertionsort}, sodass Ihre neue Implementierung ausschließlich die Funktion \texttt{swap} aufruft, um das Array zu modifizieren. Leseopeartionen sind natürlich weiterhin erlaubt.
        Zeigen Sie, dass der modifizierte Algorithmus weiterhin korrekt ist und sich die Laufzeit asymptotisch nicht ändert.
        Warum ist die neue Variante wahrscheinlich trotzdem langsamer als die alte?

        \item Betrachten Sie folgendes Array: $(2, 4, 5, 7, 8, 1, 9)$. Zeigen Sie, dass diese Zahlen beim Aufruf von \textbf{Bubble\-sort} bereits nach einer Iteration der äußeren Schleife sortiert sein werden und sich die Reihenfolge danach nicht mehr ändern wird.
        Modifizieren Sie Ihre Bubblesort-Implementierung, sodass der Algorithmus in diesem Fall und in ähnlichen Fällen vorzeitig abbricht.
        Wie lautet die Best-Case- sowie die Worst-Case-Laufzeit des modifizierten Algorithmus.

        \item Verändern Sie die \texttt{preparePartion}-Funktion Ihrer \textbf{Quicksort}-Implementierung, indem sie nun von rechts nach links über das Array laufen und die beiden Partionen von rechts nach links aufbauen.
    \end{enumerate}
\end{aufgabe}

\begin{aufgabe}{2}{Qucksort und Duplikate}
    \begin{enumerate}
        \item Angenommen, im Eingabearray für Quicksort sind viele Zahlen mehrfach enthalten.
        Weshalb kann dies ein Problem darstellen?
        Zeigen Sie anhand eines Beispiels, dass sich die Laufzeit bei vielen gleichen Werten stark verschlechtern kann.
        \item Schlagen Sie eine mögliche Lösung für dieses Problem vor.
        \item Modifizieren Sie Ihre Quicksort-Implementierung, sodass die Laufzeit der neuen Implementierung bei einem Eingabearray mit vielen gleichen Werten mindestens genau so gut ist wie bei einem Array, in dem alle Werte verschieden sind.
    \end{enumerate}
\end{aufgabe}

\ifcsdef{showLoesung}{}{\newpage}

\begin{aufgabe}{3}{Inversionen}
    Gegeben ein Array $(a_1, a_2, a_3, \ldots a_{n - 1}, a_n)$ mit $n$ Elementen ist eine Inversion definiert als ein Paar $(a_i, a_j)$ von zwei Elementen des Arrays, die in absteigender Reihenfolge im Array liegen.
    Die Menge aller Inversionen eines Arrays ist also gegeben durch $I = \{(a_i, a_j) \mid 1 \leq i < j \leq n \wedge a_j  < a_i \}$
    \begin{enumerate}
        \item Wie viele Inversionen enthalten die Arrays $(1, 2, 4, 3, 5, 6)$, $(2, 4, 5, 6, 8, 1)$ und $(6, 5, 4, 3, 2, 1)$?

        \item Wie viele Inversionen kann ein Array mit $n$ Elmenten maximal enthalten?

        \item Wie viele Inversionen enthält ein Array mit $n$ paarweise verschiedenen Elementen im Durchschnitt unter der Voraussetzung, dass jede der $n!$ Permutationen gleich wahrscheinlich ist?
        \begin{description}
            \item[Tipp:] Sei $p_{ij}$, die Wahrscheinlichkeit, dass $(a_i, a_j)$ eine Inversion ist. Dann ist die durchschnittliche Anzahl der Inversionen gegeben durch die Summe über alle $p_{ij}$.
        \end{description}

        \item Sei in einem Array das Paar $(a_i, a_{i + 1})$ eine Inversion. Zeigen Sie, dass das Vertauschen von $a_i$ und $a_{i + 1}$ die Anzahl der Inversionen um genau 1 reduziert.
        
        \item Gegeben ein Array mit $n$ Elementen und $k$ Inversionen. Zeigen Sie, dass die Laufzeit von Insertionsort $\Theta(n + k)$ ist.

        \item Gegeben ein Array mit $n$ Elementen und $k$ Inversionen. Zeigen Sie, dass die Laufzeit des modifizierten Bubblesort-Algorithmus aus Aufgabe 1b $\Omega(n + k)$ ist.
    \end{enumerate}
\end{aufgabe}

\begin{aufgabe}{4}{Quickselect}
    Gegeben ein Array mit $n$ Zahlen, sei das $k$-kleinste Element gesucht, wobei $k = 1$ dem kleinsten Element entspricht, $k = 2$ dem zweitkleinsten und so weiter.
    $k = n$ meint dementsprechend also das größte Element.
    % Das $k$-kleinste Element entspricht also dem Wert, welcher nach dem Sortieren des Arrays an Index $k$ stünde (mit $k \in \{0, 1, \ldots, n - 1\}$).
    \begin{enumerate}
        \item Geben Sie einen Algorithmus an, der das $k$-kleinste Element mittels Quicksort findet. Welches Laufzeitverhalten hat der Algorithmus.

        \item Beim Aufruf von Quicksort wird Zeit aufgewendet, um Teile des Arrays zu sortieren, die zum Finden des $k$-kleinsten Elements gar nicht nötig sind.
        Wird etwa nach dem zweitkleinsten Element gesucht, macht es keinen Sinn, die rechte Hälfte des Arrays zu sortieren.
        Entwickeln und Implementieren Sie den Algorithmus \textsc{QuickSelect}, indem Sie Quicksort modifizieren, sodass der neue Algorithmus nur den Teil des Arrays sortiert, der zum Finden des $k$-kleinsten Elements nötig ist.
        \begin{description}
            \item[Tipp:] Benötigen Sie beide rekursiven Aufrufe von Quicksort?
        \end{description}

        \item Was ist die Worst- und Best-Case-Laufzeit von \textsc{QuickSelect}?

        \item Angenommen, das Pivot-Element läge nach dem Partitionieren immer in der Mitte des aktuellen Teil\-arrays. Welche Laufzeit hätte \textsc{QuickSelect} dann im Worst Case?

        \item Stellen Sie eine Vermutung darüber an, welche Durchschnittslaufzeit \textsc{QuickSelect} in der Praxis hat.
    \end{enumerate}
\end{aufgabe}


\end{document}