\documentclass[11pt,a4paper]{article}

\usepackage{gastex}
\usepackage{etoolbox}
% \newcommand{\showLoesung}{2} %<---als Schalter
%\newcommand{\showInhalt}{1} %<---als Schalter

\usepackage{alltt,moreverb,amsmath,enumerate}
\usepackage[normalem]{ulem}
\usepackage[T1]{fontenc}
\usepackage{ae,aecompl} %helvet,mathptm
%\usepackage[left=15mm,right=15mm,top=20mm,bottom=20mm]{geometry}
\usepackage[margin=.5in]{geometry}
%\usepackage[latin1]{inputenc} % f�r Linux
\usepackage[utf8]{inputenc} % Umlaute etc. direkt schreiben (unter Windows)
\usepackage[german]{babel}
\usepackage[url]{oth-logoPNG}
%\usepackage{i2sym,i2ams}

\usepackage{tikz}
\usetikzlibrary{arrows,shapes,trees,positioning,automata,decorations.pathreplacing,decorations.pathmorphing}
\usepackage{tkz-graph}
\usepackage{color}

\usepackage{longtable}
\usepackage{tabularx}

%\usepackage{epic}
%\usepackage{eepic}
\usepackage{comment,ifthen}
\usepackage{../include/todo}

\usepackage[T1]{fontenc}
\usepackage{textcomp}

\usepackage{listings}                   % Listings in Core-Erlang und Maude
\usepackage{lstmisc}

\usepackage{epic}                       % Bildbefehle (picture)
%\usepackage{eepic}                      % erweiterte Bildbefehle

\usepackage{bbm}                        % Mengensymbole (N,C,R,B)
\usepackage{latexsym}                   % zusaetzliche Mathesymbole
\usepackage{amsmath}                    % Mathepaket von der AMS
\usepackage{amstext}
\usepackage{amsfonts}
\usepackage{stmaryrd}                   % zusaetzliche Mathesymbole
\usepackage{mathtools}
\usepackage{amsthm}
\usepackage{cancel}

\usepackage{hyperref}
\usepackage{url}                        % Zum Setzen von URLs in typewriter-face

\pagestyle{empty}

\let\epsilon=\varepsilon
\let\phi=\varphi

\frenchspacing

\setlength{\parindent}{0pt}
\setlength{\textwidth}{18.6cm}
\setlength{\textheight}{26.5cm}
\setlength{\hfuzz}{1mm}

%%% Read dates of assignments from file
\usepackage{xparse}
\ExplSyntaxOn
\ior_new:N \g_hringriin_file_stream

\NewDocumentCommand{\ReadFile}{mm}
 {
  \hringriin_read_file:nn { #1 } { #2 }
  \cs_new:Npn #1 ##1
   {
    \str_if_eq:nnTF { ##1 } { * }
      { \seq_count:c { g_hringriin_file_ \cs_to_str:N #1 _seq } }
      { \seq_item:cn { g_hringriin_file_ \cs_to_str:N #1 _seq } { ##1 } }
   }
 }

\cs_new_protected:Nn \hringriin_read_file:nn
 {
  \ior_open:Nn \g_hringriin_file_stream { #2 }
  \seq_gclear_new:c { g_hringriin_file_ \cs_to_str:N #1 _seq }
  \ior_map_inline:Nn \g_hringriin_file_stream
   {
    \seq_gput_right:cx 
     { g_hringriin_file_ \cs_to_str:N #1 _seq }
     { \tl_trim_spaces:n { ##1 } }
   }
  \ior_close:N \g_hringriin_file_stream
 }

\ExplSyntaxOff

\ReadFile{\uebungsabgabe}{../skel/UEBUNGSABGABE.def}

%%% Read subject info from file
\newcommand{\dozent}[1]{\def\DOZENT{#1}}
\newcommand{\tutoren}[1]{\def\TUTOREN{#1}}
\newcommand{\vorlesung}[1]{\def\VORLESUNG{#1}}
\newcommand{\semester}[1]{\def\SEMESTER{#1}}

\InputIfFileExists{../skel/VORLESUNG.def}{\providecommand{\TUTOREN}{}}%
{\typeout{***********}
 \typeout{Warnung: Kein File vorhanden, das die Vorlesung spezifiziert!}
 \typeout{Spezifikation muss daher im Text des Blattes oder ueber die
          Tastatur erfolgen.}
 \typeout{***********}}

\def\Uebung#1#2#3{
  \othLehrstuhlLogo[\DOZENT]
  \begin{center}
	{~\\[-2em]\Large\bf \VORLESUNG}\\[0.5em]
    \LARGE --~Tutorium #1 (Übung #2)~--\\[4mm]
  \
  \normalsize
  \textbf{#3}
    \rule{\textwidth}{0.1pt}\\[1cm]
  \end{center}
}

\def\Hinweis#1{
	{~\\[-3em]\bf Hinweis: }
	\begin{minipage}[t]{16.5cm}
	#1
	\end{minipage}\\[1em]
    \rule{\textwidth}{0.1pt}
}

\def\Tipps#1{
	{~\\[-3em]\bf Tipps: }
	\begin{minipage}[t]{16.5cm}
	#1
	\end{minipage}\\[1em]
    \rule{\textwidth}{0.1pt}
}
  
\def\MyHeader{
  \othLehrstuhlLogo[Prof.~Dr.~rer.~nat.~Carsten~Kern]%[Carsten~Kern,~Stefan~Rieger]
}

\newcommand{\sem}[1]{[\![#1\,]\!]}

\def\aufgabe#1#2{\subsection*{Aufgabe #1 (#2)}\par}
\def\endaufgabe{}

\newenvironment{loesung}{\subsection*{L\"osungsvorschlag:}}{}
\newenvironment{hinweis}{}{}
\ifthenelse{\isundefined{\showLoesung}}{\excludecomment{loesung}}{\pagestyle{plain}\excludecomment{hinweis}}

\newenvironment{tipps}{}{}
\ifthenelse{\isundefined{\showTipps}}{\excludecomment{tipps}}{\excludecomment{hinweis}}

\newenvironment{inhalt}{\subsection*{Kommentar:}}{}
\ifthenelse{\isundefined{\showInhalt}}{\excludecomment{inhalt}}{}

\long\def\Exercise#1#2{\begin{exercise}{#1}#2\end{exercise}}

\def\underbar#1{%
  \setbox0=\hbox{#1}%
  \dimen0=\dp0\relax%
  \dp0=0pt%
  \setbox0=\hbox{\underline{\box0}}%
  \dp0=\dimen0\relax%
  \box0%
  }

\makeatletter
\def\@makeunderbar[#1]#2{\expandafter\def\csname#1\endcsname{\underbar{#2}}}
\def\makeunderbar{\@ifnextchar[{\@makeunderbar}{\@makeunderbar[]}}
\makeatother

\def\T{\mathrm{T}}
\def\P{\mathrm{P}}
\def\CT{\mathrm{CT}}
\def\COp{\mathrm{COp}}

\makeunderbar{Comp}
\makeunderbar{Ops}
\makeunderbar{trans}
\makeunderbar[strans]{s-trans}
\makeunderbar[ntrans]{n-trans}
\makeunderbar{fix}

\def\labelenumi{\alph{enumi})}
\let\<=\langle
\let\>=\rangle

\parindent=0pt
\parskip=1ex

\definecolor{javared}{rgb}{0.6,0,0} % for strings
\definecolor{javagreen}{rgb}{0.25,0.5,0.35} % comments
\definecolor{javapurple}{rgb}{0.5,0,0.35} % keywords
\definecolor{javadocblue}{rgb}{0.25,0.35,0.75} % javadoc
 
\lstset{language=C++,
basicstyle=\ttfamily\footnotesize,
keywordstyle=\color{javapurple}\bf,
stringstyle=\color{javared},
commentstyle=\color{javagreen}\it\bf,
morecomment=[s][\color{javadocblue}]{/**}{*/},
numbers=left,
numberstyle=\tiny\color{gray},
stepnumber=1,
numbersep=10pt,
tabsize=3,
showspaces=false,
showstringspaces=false}

\usepackage{enumitem}
\usepackage{algpseudocode}
\usepackage{caption}
\usepackage{subcaption}
\usepackage{placeins}
\usepackage{multicol}
\usepackage{slashbox}
\usepackage{fancyvrb}
\usepackage{ulem}

\begin{document}
\thispagestyle{empty}
\DeclareRobustCommand{\ttfamily}{\fontencoding{T1}\fontfamily{lmtt}\selectfont}

\newcommand{\quotes}[1]{\glqq{}#1\grqq{}}

\Uebung{10}{11}{Simon Thelen}{16. Dezember 2021}  % FIXME: Blattnummer, Datum, Zeit

%%%%%%%%%%%%%%%%%%%%%%%%%%%%%%%%%%%%%%%%%%%%%%%%%%%%%%%%%%%%%%%%%%%%%%

\ifcsdef{showLoesung}{
\textbf{Bitte beachten Sie:} Die Lösungen können trotz sorgfältiger Prüfung Fehler enthalten.
Bei Fragen oder Unklarheiten kontaktieren Sie bitte den Tutor oder Dozenten in Tutorien, Übungen oder nach Vorlesungen.
}{}

\begin{aufgabe}{1}{Repräsentation von Graphen}
    Gegeben sei Graph $G_1$:
    \begin{figure}[h!]
        \centering
        \begin{tikzpicture}[scale=0.15]
            tikzstyle{every node}+=[inner sep=0pt]
            \tikzstyle{every node}+=[inner sep=0pt]
            \draw [black] (14.5,-16) circle (3);
\draw (14.5,-16) node {$1$};
\draw [black] (30.4,-16) circle (3);
\draw (30.4,-16) node {$3$};
\draw [black] (45.3,-16) circle (3);
\draw (45.3,-16) node {$4$};
\draw [black] (30.4,-4.1) circle (3);
\draw (30.4,-4.1) node {$2$};
\draw [black] (14.5,-31.4) circle (3);
\draw (14.5,-31.4) node {$7$};
\draw [black] (30.4,-31.4) circle (3);
\draw (30.4,-31.4) node {$8$};
\draw [black] (45.3,-31.4) circle (3);
\draw (45.3,-31.4) node {$9$};
\draw [black] (38.1,-23.8) circle (3);
\draw (38.1,-23.8) node {$6$};
\draw [black] (59.1,-16) circle (3);
\draw (59.1,-16) node {$5$};
\draw [black] (59.1,-31.4) circle (3);
\draw (59.1,-31.4) node {$10$};
\draw [black] (17.5,-16) -- (27.4,-16);
\fill [black] (27.4,-16) -- (26.6,-15.5) -- (26.6,-16.5);
\draw [black] (33.374,-15.614) arc (95.54669:84.45331:46.303);
\fill [black] (42.33,-15.61) -- (41.58,-15.04) -- (41.48,-16.03);
\draw [black] (42.96,-14.13) -- (32.74,-5.97);
\fill [black] (32.74,-5.97) -- (33.06,-6.86) -- (33.68,-6.08);
\draw [black] (28,-5.9) -- (16.9,-14.2);
\fill [black] (16.9,-14.2) -- (17.84,-14.12) -- (17.24,-13.32);
\draw [black] (14.5,-19) -- (14.5,-28.4);
\fill [black] (14.5,-28.4) -- (15,-27.6) -- (14,-27.6);
\draw [black] (17.5,-31.4) -- (27.4,-31.4);
\fill [black] (27.4,-31.4) -- (26.6,-30.9) -- (26.6,-31.9);
\draw [black] (30.4,-28.4) -- (30.4,-19);
\fill [black] (30.4,-19) -- (29.9,-19.8) -- (30.9,-19.8);
\draw [black] (16.65,-29.31) -- (28.25,-18.09);
\fill [black] (28.25,-18.09) -- (27.32,-18.28) -- (28.02,-19);
\draw [black] (42.3,-31.4) -- (33.4,-31.4);
\fill [black] (33.4,-31.4) -- (34.2,-31.9) -- (34.2,-30.9);
\draw [black] (45.3,-28.4) -- (45.3,-19);
\fill [black] (45.3,-19) -- (44.8,-19.8) -- (45.8,-19.8);
\draw [black] (32.51,-18.13) -- (35.99,-21.67);
\fill [black] (35.99,-21.67) -- (35.79,-20.74) -- (35.07,-21.45);
\draw [black] (40.13,-21.6) -- (43.27,-18.2);
\fill [black] (43.27,-18.2) -- (42.36,-18.45) -- (43.09,-19.13);
\draw [black] (28.25,-29.31) -- (16.65,-18.09);
\fill [black] (16.65,-18.09) -- (16.88,-19) -- (17.58,-18.28);
\draw [black] (42.334,-16.448) arc (-83.55508:-96.44492:39.951);
\fill [black] (33.37,-16.45) -- (34.1,-17.03) -- (34.22,-16.04);
\draw [black] (56.1,-16) -- (48.3,-16);
\fill [black] (48.3,-16) -- (49.1,-16.5) -- (49.1,-15.5);
\draw [black] (57.1,-18.23) -- (47.3,-29.17);
\fill [black] (47.3,-29.17) -- (48.21,-28.9) -- (47.46,-28.24);
\draw [black] (59.1,-28.4) -- (59.1,-19);
\fill [black] (59.1,-19) -- (58.6,-19.8) -- (59.6,-19.8);
\draw [black] (48.3,-31.4) -- (56.1,-31.4);
\fill [black] (56.1,-31.4) -- (55.3,-30.9) -- (55.3,-31.9);
\draw [black] (29.077,-13.32) arc (234:-54:2.25);
\fill [black] (31.72,-13.32) -- (32.6,-12.97) -- (31.79,-12.38);

        \end{tikzpicture}
    \end{figure}
    \begin{enumerate}
        \item Geben Sie für $G_1$ die Adjazenzlistendarstellung (gestreute und dichte Speicherung) und die Adjazenzmatrixdarstellung an.
        \item Nennen Sie zwei Vor- und zwei Nachteile der Repräsentation von Graphen als Adjazenzmatrix im Vergleich zu Adjazenzlisten.
        \item Sie möchten zu einem gegebenen Knoten $v$ eines gerichteten Graphen $\operatorname{deg}^+(v)$ und $\operatorname{deg}^-(v)$, also Eingangs- und Ausgangsgrad, bestimmen.
        Wie schnell (in $O$-Notation) schaffen Sie dies, wenn Sie den Graphen als Adjazenzliste gegeben haben?
        Wie lange brauchen Sie, wenn Sie ihn als Adjazenzliste gegeben haben.
        \item 
        Geben Sie zu folgender Adjazenzmatrix an, ob der Graph gerichtet oder ungerichtet ist.
        Zeichnen Sie anschließend den Graphen.
        Nummerieren Sie die Knoten anhand der Zeilen und Spalten der Matrix.
        \begin{equation*}
            \begin{pmatrix}
                0 & 1 & 1 & 0 & 0 \\
                1 & 0 & 1 & 1 & 1 \\
                1 & 1 & 0 & 1 & 1 \\
                0 & 1 & 1 & 0 & 1 \\
                0 & 1 & 1 & 1 & 0 
            \end{pmatrix}
        \end{equation*}
        \item Zeichnen Sie den Graph zu folgender Inzidenzmatrix.
        Nummerieren Sie die Knoten anhand der Zeilen der Matrix.
        \begin{equation*}
            \begin{pmatrix}
                -1 & 0 & 0 & 0 & 1 & 0 & -1 & 0 \\
                1 & -1 & 0 & 0 & 0 & 1 & 0 & 0 \\
                0 & 1 & -1 & 1 & 0 & 0 & 0 & 0 \\
                0 & 0 & 1 & -1 & -1 & 0 & 0 & 1 \\
                0 & 0 & 0 & 0 & 0 & -1 & 1 & -1
            \end{pmatrix}
        \end{equation*}
    \end{enumerate}
\end{aufgabe}

\begin{aufgabe}{2}{Breiten- und Tiefensuche}
    Gegeben sei den Graph aus Aufgabe 1 als Adjazenzliste.
    Die Kanten, die von einem Knoten aus beginnen, liegen in der entsprechenden Liste in numerisch aufsteigend sortierter Reihenfolge vor.
    \begin{enumerate}
        \item Führen Sie eine Breitensuche auf dem Graphen durch.
        Geben Sie an, in welcher Reihenfolge Knoten ent\-deckt und abgeschlossen werden.
        \item Führen Sie eine Tiefensuche auf dem Graphen durch.
        Geben Sie an, in welcher Reihenfolge Knoten entdeckt und abgeschlossen werden. Vergessen Sie nicht, dass die Tiefensuche von allen noch nicht besuchten Knoten aus gestartet werden muss.
        \item Angenommen, der Graph sei ungerichtet (alle Kanten gehen in beide Richtungen, Selbstschleifen werden entfernt).
        Führen Sie eine Breitensuche und Tiefensuche auf dem ungerichteten Graphen durch.
        Geben Sie an, in welcher Reihenfolge Knoten entdeckt und abgeschlossen werden.
        \item Welche Laufzeit hätten Breiten- und Tiefensuche, wenn der Eingabegraph in Adjazenzmatrixform vorliegen würde.
        \item
        Gegeben sei ein Baum als gerichteter Graph (Die Wurzel hat Eingangsgrad 0, Blätter Ausgangsgrad 0).
        Angenommen, Sie beginnnen bei der Wurzel des Baumes einen Tiefensuchendurchlauf und geben alle Knoten in der Reihenfolge aus, in der Sie sie entdecken.
        Welcher Form von Baumtraversion entspricht dies?
        Wie sieht es aus, wenn Sie die Knoten in der Reihenfolge ausgeben, in der Sie sie abschließen?
        \item
        Begründen Sie, warum die gestrichelten Linien in folgendem Graphen keinen Breitensuchenbaum bilden können.
        \begin{figure}[h!]
            \centering
            \begin{tikzpicture}[scale=0.1]
\tikzstyle{every node}+=[inner sep=0pt]
\draw [black] (24.3,-12.9) circle (3);
\draw (24.3,-12.9) node {$a$};
\draw [black] (40.9,-12.9) circle (3);
\draw (40.9,-12.9) node {$b$};
\draw [black] (24.3,-30) circle (3);
\draw (24.3,-30) node {$c$};
\draw [black] (32.7,-21.6) circle (3);
\draw (32.7,-21.6) node {$d$};
\draw [black] (40.9,-30) circle (3);
\draw (40.9,-30) node {$e$};
\draw [black, dashed] (27.3,-12.9) -- (37.9,-12.9);
\fill [black] (37.9,-12.9) -- (37.1,-12.4) -- (37.1,-13.4);
\draw [black, dashed] (24.3,-15.9) -- (24.3,-27);
\fill [black] (24.3,-27) -- (24.8,-26.2) -- (23.8,-26.2);
\draw [black, dashed] (38.84,-15.08) -- (34.76,-19.42);
\fill [black] (34.76,-19.42) -- (35.67,-19.18) -- (34.94,-18.49);
\draw [black, dashed] (27.3,-30) -- (37.9,-30);
\fill [black] (37.9,-30) -- (37.1,-29.5) -- (37.1,-30.5);
\draw [black] (40.9,-15.9) -- (40.9,-27);
\fill [black] (40.9,-27) -- (41.4,-26.2) -- (40.4,-26.2);
\draw [black] (26.42,-27.88) -- (30.58,-23.72);
\fill [black] (30.58,-23.72) -- (29.66,-23.93) -- (30.37,-24.64);
\end{tikzpicture}
        \end{figure}
    \end{enumerate}
\end{aufgabe}

\begin{aufgabe}{3}{Anwendungen von Tiefensuche}
    \begin{enumerate}[label=\alph*)]
        \item Ein ungerichteter Graph ist zusammenhängend, wenn von jedem Knoten aus jeder andere Knoten erreichbar ist.
        Gegeben sei ein ungerichteter Graph in Adjazenzlistenform.
        Implementieren Sie einen Algorithmus mit Laufzeit $O(|V| + |E|)$ in Pseudocode oder einer Programmiersprache Ihrer Wahl, der überprüft, ob der Graph zusammenhängend ist.
        \item \label{transpose} Gegeben sei ein gerichteter Graph $G = (V, E)$. Dann ist die Transposition dieses Graphen $G^T$ definiert als der Graph $G^T = (V, \{(v, u) \mid (u, v) \in E\})$.
        Es handelt sich also um den gleichen Graphen, nur das alle seine Kanten in die andere Richtung verlaufen.
        Implementieren Sie einen Algorithmus mit Laufzeit $\Theta(|V| + |E|)$ in Pseudocode oder einer Programmiersprache Ihrer Wahl, welcher einen gerichteten Graphen $G$ in Adjazenzlistenform als Eingabe erhält und $G^T$ zurückgibt.
        \item Ein gerichteter Graph ist stark zusammenhängend, wenn von jedem Knoten aus jeder andere Knoten er\-reichbar ist.
        Gegeben sei ein gerichteter Graph in Adjazenzlistenform.
        Implementieren Sie einen Algorithmus mit Laufzeit $O(|V| + |E|)$ in Pseudocode oder einer Programmiersprache Ihrer Wahl, der überprüft, ob der Graph stark zusammenhängend ist.
        \begin{description}
            \item[Tipp:] Verwenden Sie den Algorithmus aus Teilaufgabe \ref*{transpose} als Subroutine.
        \end{description}
        \item Gegeben sein ein zusammenhängender, ungerichteter Graph.
        Implementieren Sie einen Algorithmus mit Laufzeit $O(|V| + |E|)$ in Pseudocode oder einer Programmiersprache Ihrer, welcher zusammenhängenden Pfad durch den Graphen ausgibt, der jede Kante des Graphen genau zweimal passiert, nämlich einmal in jede Richtung.
    \end{enumerate}
\end{aufgabe}


\end{document}