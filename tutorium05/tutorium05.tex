\documentclass[11pt,a4paper]{article}

\usepackage{gastex}
\usepackage{etoolbox}
% \newcommand{\showLoesung}{2} %<---als Schalter
%\newcommand{\showInhalt}{1} %<---als Schalter

\usepackage{alltt,moreverb,amsmath,enumerate}
\usepackage[normalem]{ulem}
\usepackage[T1]{fontenc}
\usepackage{ae,aecompl} %helvet,mathptm
%\usepackage[left=15mm,right=15mm,top=20mm,bottom=20mm]{geometry}
\usepackage[margin=.5in]{geometry}
%\usepackage[latin1]{inputenc} % f�r Linux
\usepackage[utf8]{inputenc} % Umlaute etc. direkt schreiben (unter Windows)
\usepackage[german]{babel}
\usepackage[url]{oth-logoPNG}
%\usepackage{i2sym,i2ams}

\usepackage{tikz}
\usetikzlibrary{arrows,shapes,trees,positioning,automata,decorations.pathreplacing,decorations.pathmorphing}
\usepackage{tkz-graph}
\usepackage{color}

\usepackage{longtable}
\usepackage{tabularx}

%\usepackage{epic}
%\usepackage{eepic}
\usepackage{comment,ifthen}
\usepackage{../include/todo}

\usepackage[T1]{fontenc}
\usepackage{textcomp}

\usepackage{listings}                   % Listings in Core-Erlang und Maude
\usepackage{lstmisc}

\usepackage{epic}                       % Bildbefehle (picture)
%\usepackage{eepic}                      % erweiterte Bildbefehle

\usepackage{bbm}                        % Mengensymbole (N,C,R,B)
\usepackage{latexsym}                   % zusaetzliche Mathesymbole
\usepackage{amsmath}                    % Mathepaket von der AMS
\usepackage{amstext}
\usepackage{amsfonts}
\usepackage{stmaryrd}                   % zusaetzliche Mathesymbole
\usepackage{mathtools}
\usepackage{amsthm}
\usepackage{cancel}

\usepackage{hyperref}
\usepackage{url}                        % Zum Setzen von URLs in typewriter-face

\pagestyle{empty}

\let\epsilon=\varepsilon
\let\phi=\varphi

\frenchspacing

\setlength{\parindent}{0pt}
\setlength{\textwidth}{18.6cm}
\setlength{\textheight}{26.5cm}
\setlength{\hfuzz}{1mm}

%%% Read dates of assignments from file
\usepackage{xparse}
\ExplSyntaxOn
\ior_new:N \g_hringriin_file_stream

\NewDocumentCommand{\ReadFile}{mm}
 {
  \hringriin_read_file:nn { #1 } { #2 }
  \cs_new:Npn #1 ##1
   {
    \str_if_eq:nnTF { ##1 } { * }
      { \seq_count:c { g_hringriin_file_ \cs_to_str:N #1 _seq } }
      { \seq_item:cn { g_hringriin_file_ \cs_to_str:N #1 _seq } { ##1 } }
   }
 }

\cs_new_protected:Nn \hringriin_read_file:nn
 {
  \ior_open:Nn \g_hringriin_file_stream { #2 }
  \seq_gclear_new:c { g_hringriin_file_ \cs_to_str:N #1 _seq }
  \ior_map_inline:Nn \g_hringriin_file_stream
   {
    \seq_gput_right:cx 
     { g_hringriin_file_ \cs_to_str:N #1 _seq }
     { \tl_trim_spaces:n { ##1 } }
   }
  \ior_close:N \g_hringriin_file_stream
 }

\ExplSyntaxOff

\ReadFile{\uebungsabgabe}{../skel/UEBUNGSABGABE.def}

%%% Read subject info from file
\newcommand{\dozent}[1]{\def\DOZENT{#1}}
\newcommand{\tutoren}[1]{\def\TUTOREN{#1}}
\newcommand{\vorlesung}[1]{\def\VORLESUNG{#1}}
\newcommand{\semester}[1]{\def\SEMESTER{#1}}

\InputIfFileExists{../skel/VORLESUNG.def}{\providecommand{\TUTOREN}{}}%
{\typeout{***********}
 \typeout{Warnung: Kein File vorhanden, das die Vorlesung spezifiziert!}
 \typeout{Spezifikation muss daher im Text des Blattes oder ueber die
          Tastatur erfolgen.}
 \typeout{***********}}

\def\Uebung#1#2#3{
  \othLehrstuhlLogo[\DOZENT]
  \begin{center}
	{~\\[-2em]\Large\bf \VORLESUNG}\\[0.5em]
    \LARGE --~Tutorium #1 (Übung #2)~--\\[4mm]
  \
  \normalsize
  \textbf{#3}
    \rule{\textwidth}{0.1pt}\\[1cm]
  \end{center}
}

\def\Hinweis#1{
	{~\\[-3em]\bf Hinweis: }
	\begin{minipage}[t]{16.5cm}
	#1
	\end{minipage}\\[1em]
    \rule{\textwidth}{0.1pt}
}

\def\Tipps#1{
	{~\\[-3em]\bf Tipps: }
	\begin{minipage}[t]{16.5cm}
	#1
	\end{minipage}\\[1em]
    \rule{\textwidth}{0.1pt}
}
  
\def\MyHeader{
  \othLehrstuhlLogo[Prof.~Dr.~rer.~nat.~Carsten~Kern]%[Carsten~Kern,~Stefan~Rieger]
}

\newcommand{\sem}[1]{[\![#1\,]\!]}

\def\aufgabe#1#2{\subsection*{Aufgabe #1 (#2)}\par}
\def\endaufgabe{}

\newenvironment{loesung}{\subsection*{L\"osungsvorschlag:}}{}
\newenvironment{hinweis}{}{}
\ifthenelse{\isundefined{\showLoesung}}{\excludecomment{loesung}}{\pagestyle{plain}\excludecomment{hinweis}}

\newenvironment{tipps}{}{}
\ifthenelse{\isundefined{\showTipps}}{\excludecomment{tipps}}{\excludecomment{hinweis}}

\newenvironment{inhalt}{\subsection*{Kommentar:}}{}
\ifthenelse{\isundefined{\showInhalt}}{\excludecomment{inhalt}}{}

\long\def\Exercise#1#2{\begin{exercise}{#1}#2\end{exercise}}

\def\underbar#1{%
  \setbox0=\hbox{#1}%
  \dimen0=\dp0\relax%
  \dp0=0pt%
  \setbox0=\hbox{\underline{\box0}}%
  \dp0=\dimen0\relax%
  \box0%
  }

\makeatletter
\def\@makeunderbar[#1]#2{\expandafter\def\csname#1\endcsname{\underbar{#2}}}
\def\makeunderbar{\@ifnextchar[{\@makeunderbar}{\@makeunderbar[]}}
\makeatother

\def\T{\mathrm{T}}
\def\P{\mathrm{P}}
\def\CT{\mathrm{CT}}
\def\COp{\mathrm{COp}}

\makeunderbar{Comp}
\makeunderbar{Ops}
\makeunderbar{trans}
\makeunderbar[strans]{s-trans}
\makeunderbar[ntrans]{n-trans}
\makeunderbar{fix}

\def\labelenumi{\alph{enumi})}
\let\<=\langle
\let\>=\rangle

\parindent=0pt
\parskip=1ex

\definecolor{javared}{rgb}{0.6,0,0} % for strings
\definecolor{javagreen}{rgb}{0.25,0.5,0.35} % comments
\definecolor{javapurple}{rgb}{0.5,0,0.35} % keywords
\definecolor{javadocblue}{rgb}{0.25,0.35,0.75} % javadoc
 
\lstset{language=Java,
basicstyle=\ttfamily\footnotesize,
keywordstyle=\color{javapurple}\bf,
stringstyle=\color{javared},
commentstyle=\color{javagreen}\it\bf,
morecomment=[s][\color{javadocblue}]{/**}{*/},
numbers=left,
numberstyle=\tiny\color{gray},
stepnumber=1,
numbersep=10pt,
tabsize=3,
showspaces=false,
showstringspaces=false}

\usepackage{enumitem}
\usepackage{algpseudocode}
\usepackage{caption}
\usepackage{subcaption}
\usepackage{placeins}
\usepackage{multicol}

\begin{document}
\thispagestyle{empty}

\Uebung{5}{6}{Simon Thelen}{11. November 2021}  % FIXME: Blattnummer, Datum, Zeit

%%%%%%%%%%%%%%%%%%%%%%%%%%%%%%%%%%%%%%%%%%%%%%%%%%%%%%%%%%%%%%%%%%%%%%

\ifcsdef{showLoesung}{
\textbf{Bitte beachten Sie:} Die Lösungen können trotz sorgfältiger Prüfung Fehler enthalten.
Bei Fragen oder Unklarheiten kontaktieren Sie bitte den Tutor oder Dozenten in Tutorien, Übungen oder nach Vorlesungen.
}{}

\begin{aufgabe}{1}{Countsort}
    \begin{enumerate}
        \item
        13 Schülerinnen und Schüler haben an einer Prüfung teilgenommen.
        Hier die Notenergebnisse:
        \begin{table}[h!]
            \centering
            \begin{tabular}{|c|c|c|c|c|c|c|c|c|c|c|c|c|}
                \hline
                \textbf{A} & \textbf{B} & \textbf{C} & \textbf{D} & \textbf{E} & \textbf{F} & \textbf{G} & \textbf{H} & \textbf{I} & \textbf{J} & \textbf{K} & \textbf{L} & \textbf{M} \\
                \hline
                3 & 2 & 4 & 2 & 4 & 3 & 3 & 1 & 2 & 2 & 4 & 6 & 3 \\
                \hline
            \end{tabular}
        \end{table}
        \ \\
        Sortieren sie von Hand die Einträge aufsteigend nach Note mittels Countsort (mögliche Schulnoten sind 1, 2, 3, 4, 5 und 6).
        Geben Sie für das Ausgabearray an, in welcher Reihenfolge die einzelnen Werte eingefügt wurden.
        Geben Sie außerdem, jeweils nach den drei \texttt{for}-Schleifen, den Inhalt des Hilfsarrays an.
        \item
        Ist Countsort stabil?
        Falls ja, begründen Sie dies.
        Falls nein, wie könnte man den Algorithmus anpassen, damit er stabil ist?
    \end{enumerate}
\end{aufgabe}

\begin{aufgabe}{2}{Mapsort}
    \begin{enumerate}
        \item 
        Sortieren sie das Array (45, 13, 8, 0, 91, 93, 100, 60, 41, 69, 32) von Hand mittels Mapsort ($c = 1.0$).
        Geben Sie nach dem Einfügen jedes neuen Wertes den Inhalt des temporären Arrays an.
        \item
        Betrachten Sie folgende Variante von Mapsort:
        Statt konkreten Werten werden im temporären Array (doppelt) verkettete Listen verwaltet.
        Wenn sich beim Einfügen an der entsprechenden Stelle bereits ein Wert befindet, wird der neue Wert an die zugehörige Liste angehängt.

        An welchen Stellen muss die Implementierung aus der Vorlesung dann angepasst werden, um auf diese Weise Zahlen sortieren zu können?
        
        \emph{Zusatzaufgabe:} Wie muss der neue Algorithmus angepasst werden, um eine Worst-Case-Laufzeit von $O\big(n \log(n)\big)$ zu garantieren.
    \end{enumerate}
\end{aufgabe}

\begin{aufgabe}{3}{Verkettete Listen}
    \begin{enumerate}
        \item 
        Welche der folgenden Algorithmen sind auch auf doppelt verketteten Listen ohne Laufzeiteinbußen im $O$-Kalkül umsetzbar?
        Begründen Sie jeweils Ihre Entscheidung.
        \begin{multicols}{3}
        \begin{itemize}
            \item Binäre Suche
            \item Insertionsort
            \item Bubblesort
            \item Mergesort
            \item Heapsort
            \item Countsort
        \end{itemize}
        \end{multicols}
        \item Implementieren Sie die Operationen \texttt{pop} und \texttt{push} der Datenstruktur \glqq{}Stack\grqq{} mithilfe einer einfach verketteten Liste in Pseudocode oder einer Programmiersprache Ihrer Wahl.
        Jede Operation soll nur konstante Laufzeit benötigen.
        \item Implementieren Sie die Operationen \texttt{enqueue} und \texttt{dequeue} der Datenstruktur \glqq{}Queue\grqq{} mithilfe einer einfach verketteten Liste in Pseudocode oder einer Programmiersprache Ihrer Wahl.
        Jede Operation soll nur konstante Laufzeit benötigen.
    \end{enumerate}
\end{aufgabe}

\begin{aufgabe}{4}{Binäre, verkettete Bäume}
    \begin{enumerate}
        \item Gegeben sei ein binärer, verketteter Baum mit Preorder-Darstellung (a, b, c, d, e, f, g, h, i) und Inorder-Darstellung (c, d, b, e, a, h, g, i, f).
        Wie sieht der Baum aus?
        \item Gegeben sei ein binärer, verketteter Suchbaum mit Postorder-Darstellung (15, 28, 12, 33, 41, 37, 30, 50, 42).
        Wie sieht der Baum aus?
        \item Gegeben sei ein linksvoller, binärer, verketteter Baum mit Inorder-Darstellung (9, 8, 7, 6, 5, 4, 3, 2, 1).
        Wie sieht der Baum aus?
        \item Sei ein Knoten eines binären, verketteten Suchbaums durch folgende Datenstruktur definiert:
        \begin{lstlisting}[language=c++]
class Node {
    int value;
    Node *left, *right, *parent;
} 
        \end{lstlisting}
        Geben Sie einen Algorithmus in Pseudocode oder einer Programmiersprache Ihrer Wahl an, welcher Ihnen bei einem gegebenen Knoten den nächsten Knoten in Inorder-Reihenfolge liefert.
        \item
        Gegeben sei ein aufsteigend sortiertes Array mit $n$ ganzen Zahlen.
        Geben Sie einen Algorithmus mit Laufzeit $O(n)$ in Pseudocode oder einer Programmiersprache Ihrer Wahl an, welcher einen binär verketteten Suchbaum zurückgibt, der die gleichen Elemente wie das Array enthält.
    \end{enumerate}
\end{aufgabe}

\begin{aufgabe}{5}{Stacks}
    \begin{enumerate}
        \item
        Gegeben sei ein Array mit $n$ ganzen Zahlen.
        Geben Sie einen Algorithmus in Pseudocode oder einer Programmiersprache Ihrer Wahl an, der mithilfe eines Stacks die Reihenfolge der Werte im Array umkehrt.

        Beispiel: \texttt{reverse(\{4, 2, 5, 3, 6, 1\}) = \{1, 6, 3, 5, 2, 4\})}

        \emph{Zusatzaufgabe:} Schaffen Sie es auch ohne einen Stack und mit nur konstantem Speicherbedarf.

        \item Gegeben sei folgende kontextfreie Grammatik:
        \begin{equation*}
            S \rightarrow
            \epsilon \mid
            \text{\texttt{(}}S\text{\texttt{)}} \mid
            \text{\texttt{[}}S\text{\texttt{]}} \mid
            \text{\texttt{\{}}S\text{\texttt{\}}}
        \end{equation*}
        Sei $L$ die durch diese Grammatik definierte Sprache.
        Geben Sie einen effizienten Algorithmus an (in Pseudocode oder einer Programmiersprache Ihrer Wahl), welcher bei einem gegebenen Wort $w \in \{\text{\texttt{()[]\{\}}}\}^n$ in Laufzeit $O(n)$ entscheidet, ob $w \in L$.

        Beispiele: $\text{\texttt{"([]\{[()]\})"}} \in L$, $\text{\texttt{"[\{(\})]"}} \not\in L$

        \item
        Manchmal wissen Sie im Vorhinein nicht, wie groß das zugrundeliegende Array eines Stacks sein soll.
        Deshalb beginnnt man häufig mit einem \glqq{}kleinen\grqq{} Array und vergrößert es dynamisch, wenn kein Platz für neue Elemente übrig ist.
        Vergrößern bedeutet hierbei: Allokiere Speicherplatz für ein größeres Array und kopiere die bestehenden Werte in das neue Array.
        Wir nehmen im folgenden an, dass ein Vergrößern lineare Laufzeit in der Größe des neuen Array benötigt.
        Ein Push ohne Vergrößern benötigt dagegen nur konstante Laufzeit.

        Vergleichen wir zwei mögliche Vorgehensweisen für das Vergrößern.
        \begin{enumerate}[label=(\roman*)]
            \item Das Array wird mit Länge 8 initialisiert.
            Jedes Mal, wenn bei einem \texttt{push} kein Platz für das neue Element vorhanden ist, wird das Array um 8 Elemente vergrößert.
            \item Das Array wird mit Länge 8 initialisiert.
            Jedes Mal, wenn bei einem \texttt{push} kein Platz für das neue Element vorhanden ist, wird die Länge des Array verdoppelt.
        \end{enumerate}

        Angenommen, es werden solange Elemente auf den Stack gepusht, bis der Stack $2^n$ Elemente enthält ($n \geq 3$).
        Welche Laufzeit benötigen diese $2^n$ Push-Operationen bei Vorgehen a) und bei Vorgehen b) insgesamt?
        Welche Laufzeit hat jeweils eine der $2^n$ Operationen im Durchschnitt?
    \end{enumerate}
\end{aufgabe}

\end{document}