\documentclass[11pt,a4paper]{article}

\usepackage{gastex}
\usepackage{etoolbox}
% \newcommand{\showLoesung}{2} %<---als Schalter
% \newcommand{\showInhalt}{1} %<---als Schalter

\usepackage{alltt,moreverb,amsmath,enumerate}
\usepackage[normalem]{ulem}
\usepackage[T1]{fontenc}
\usepackage{ae,aecompl} %helvet,mathptm
%\usepackage[left=15mm,right=15mm,top=20mm,bottom=20mm]{geometry}
\usepackage[margin=.5in]{geometry}
%\usepackage[latin1]{inputenc} % f�r Linux
\usepackage[utf8]{inputenc} % Umlaute etc. direkt schreiben (unter Windows)
\usepackage[german]{babel}
\usepackage[url]{oth-logoPNG}
%\usepackage{i2sym,i2ams}

\usepackage{tikz}
\usetikzlibrary{arrows,shapes,trees,positioning,automata,decorations.pathreplacing,decorations.pathmorphing}
\usepackage{tkz-graph}
\usepackage{color}

\usepackage{longtable}
\usepackage{tabularx}

%\usepackage{epic}
%\usepackage{eepic}
\usepackage{comment,ifthen}
\usepackage{../include/todo}

\usepackage[T1]{fontenc}
\usepackage{textcomp}

\usepackage{listings}                   % Listings in Core-Erlang und Maude
\usepackage{lstmisc}

\usepackage{epic}                       % Bildbefehle (picture)
%\usepackage{eepic}                      % erweiterte Bildbefehle

\usepackage{bbm}                        % Mengensymbole (N,C,R,B)
\usepackage{latexsym}                   % zusaetzliche Mathesymbole
\usepackage{amsmath}                    % Mathepaket von der AMS
\usepackage{amstext}
\usepackage{amsfonts}
\usepackage{stmaryrd}                   % zusaetzliche Mathesymbole
\usepackage{mathtools}
\usepackage{amsthm}
\usepackage{cancel}

\usepackage{hyperref}
\usepackage{url}                        % Zum Setzen von URLs in typewriter-face

\pagestyle{empty}

\let\epsilon=\varepsilon
\let\phi=\varphi

\frenchspacing

\setlength{\parindent}{0pt}
\setlength{\textwidth}{18.6cm}
\setlength{\textheight}{26.5cm}
\setlength{\hfuzz}{1mm}

%%% Read dates of assignments from file
\usepackage{xparse}
\ExplSyntaxOn
\ior_new:N \g_hringriin_file_stream

\NewDocumentCommand{\ReadFile}{mm}
 {
  \hringriin_read_file:nn { #1 } { #2 }
  \cs_new:Npn #1 ##1
   {
    \str_if_eq:nnTF { ##1 } { * }
      { \seq_count:c { g_hringriin_file_ \cs_to_str:N #1 _seq } }
      { \seq_item:cn { g_hringriin_file_ \cs_to_str:N #1 _seq } { ##1 } }
   }
 }

\cs_new_protected:Nn \hringriin_read_file:nn
 {
  \ior_open:Nn \g_hringriin_file_stream { #2 }
  \seq_gclear_new:c { g_hringriin_file_ \cs_to_str:N #1 _seq }
  \ior_map_inline:Nn \g_hringriin_file_stream
   {
    \seq_gput_right:cx 
     { g_hringriin_file_ \cs_to_str:N #1 _seq }
     { \tl_trim_spaces:n { ##1 } }
   }
  \ior_close:N \g_hringriin_file_stream
 }

\ExplSyntaxOff

\ReadFile{\uebungsabgabe}{../skel/UEBUNGSABGABE.def}

%%% Read subject info from file
\newcommand{\dozent}[1]{\def\DOZENT{#1}}
\newcommand{\tutoren}[1]{\def\TUTOREN{#1}}
\newcommand{\vorlesung}[1]{\def\VORLESUNG{#1}}
\newcommand{\semester}[1]{\def\SEMESTER{#1}}

\InputIfFileExists{../skel/VORLESUNG.def}{\providecommand{\TUTOREN}{}}%
{\typeout{***********}
 \typeout{Warnung: Kein File vorhanden, das die Vorlesung spezifiziert!}
 \typeout{Spezifikation muss daher im Text des Blattes oder ueber die
          Tastatur erfolgen.}
 \typeout{***********}}

\def\Uebung#1#2#3{
  \othLehrstuhlLogo[\DOZENT]
  \begin{center}
	{~\\[-2em]\Large\bf \VORLESUNG}\\[0.5em]
    \LARGE --~Tutorium #1 (Übung #2)~--\\[4mm]
  \
  \normalsize
  \textbf{#3}
    \rule{\textwidth}{0.1pt}\\[1cm]
  \end{center}
}

\def\Hinweis#1{
	{~\\[-3em]\bf Hinweis: }
	\begin{minipage}[t]{16.5cm}
	#1
	\end{minipage}\\[1em]
    \rule{\textwidth}{0.1pt}
}

\def\Tipps#1{
	{~\\[-3em]\bf Tipps: }
	\begin{minipage}[t]{16.5cm}
	#1
	\end{minipage}\\[1em]
    \rule{\textwidth}{0.1pt}
}
  
\def\MyHeader{
  \othLehrstuhlLogo[Prof.~Dr.~rer.~nat.~Carsten~Kern]%[Carsten~Kern,~Stefan~Rieger]
}

\newcommand{\sem}[1]{[\![#1\,]\!]}

\def\aufgabe#1#2{\subsection*{Aufgabe #1 (#2)}\par}
\def\endaufgabe{}

\newenvironment{loesung}{\subsection*{L\"osungsvorschlag:}}{}
\newenvironment{hinweis}{}{}
\ifthenelse{\isundefined{\showLoesung}}{\excludecomment{loesung}}{\pagestyle{plain}\excludecomment{hinweis}}

\newenvironment{tipps}{}{}
\ifthenelse{\isundefined{\showTipps}}{\excludecomment{tipps}}{\excludecomment{hinweis}}

\newenvironment{inhalt}{\subsection*{Kommentar:}}{}
\ifthenelse{\isundefined{\showInhalt}}{\excludecomment{inhalt}}{}

\long\def\Exercise#1#2{\begin{exercise}{#1}#2\end{exercise}}

\def\underbar#1{%
  \setbox0=\hbox{#1}%
  \dimen0=\dp0\relax%
  \dp0=0pt%
  \setbox0=\hbox{\underline{\box0}}%
  \dp0=\dimen0\relax%
  \box0%
  }

\makeatletter
\def\@makeunderbar[#1]#2{\expandafter\def\csname#1\endcsname{\underbar{#2}}}
\def\makeunderbar{\@ifnextchar[{\@makeunderbar}{\@makeunderbar[]}}
\makeatother

\def\T{\mathrm{T}}
\def\P{\mathrm{P}}
\def\CT{\mathrm{CT}}
\def\COp{\mathrm{COp}}

\makeunderbar{Comp}
\makeunderbar{Ops}
\makeunderbar{trans}
\makeunderbar[strans]{s-trans}
\makeunderbar[ntrans]{n-trans}
\makeunderbar{fix}

\def\labelenumi{\alph{enumi})}
\let\<=\langle
\let\>=\rangle

\parindent=0pt
\parskip=1ex

\definecolor{javared}{rgb}{0.6,0,0} % for strings
\definecolor{javagreen}{rgb}{0.25,0.5,0.35} % comments
\definecolor{javapurple}{rgb}{0.5,0,0.35} % keywords
\definecolor{javadocblue}{rgb}{0.25,0.35,0.75} % javadoc
 
\lstset{language=C++,
basicstyle=\ttfamily\footnotesize,
keywordstyle=\color{javapurple}\bf,
stringstyle=\color{javared},
commentstyle=\color{javagreen}\it\bf,
morecomment=[s][\color{javadocblue}]{/**}{*/},
numbers=left,
numberstyle=\tiny\color{gray},
stepnumber=1,
numbersep=10pt,
tabsize=3,
showspaces=false,
showstringspaces=false}

\usepackage{enumitem}
\usepackage{algpseudocode}
\usepackage{caption}
\usepackage{subcaption}
\usepackage{placeins}
\usepackage{multicol}
\usepackage{slashbox}
\usepackage{fancyvrb}
\usepackage{ulem}
\usepackage{amssymb}

\begin{document}
\thispagestyle{empty}
\DeclareRobustCommand{\ttfamily}{\fontencoding{T1}\fontfamily{lmtt}\selectfont}

\newcommand{\quotes}[1]{\glqq{}#1\grqq{}}

\Uebung{11}{12}{Simon Thelen}{13. Januar 2021}  % FIXME: Blattnummer, Datum, Zeit

%%%%%%%%%%%%%%%%%%%%%%%%%%%%%%%%%%%%%%%%%%%%%%%%%%%%%%%%%%%%%%%%%%%%%%

\ifcsdef{showLoesung}{
\textbf{Bitte beachten Sie:} Die Lösungen können trotz sorgfältiger Prüfung Fehler enthalten.
Bei Fragen oder Unklarheiten kontaktieren Sie bitte den Tutor oder Dozenten in Tutorien, Übungen oder nach Vorlesungen.
}{}

\begin{aufgabe}{1}{Tiefensuche}
    \begin{enumerate}
        \item Zeigen oder widerlegen Sie folgende Aussage:
        \emph{Wenn während einer Tiefensuche auf einem ungerichteten Graphen eine Kante von $u$ nach $v$ betrachtet wird, ist $v$ niemals schwarz.}
        \item Zeigen oder widerlegen Sie folgende Aussage:
        \emph{Wenn während einer Tiefensuche auf einem gerichteten, azyklischen Graphen eine Kante von $u$ nach $v$ betrachtet wird, ist $v$ niemals grau.}
    \end{enumerate}
\end{aufgabe}

\begin{loesung}
    \begin{enumerate}
        \item Die Aussage ist falsch.
        \begin{proof}
            Beweis durch Gegenbeispiel:
            \begin{figure}[h!]
                \centering
                \begin{tikzpicture}[scale=0.12]
                    \tikzstyle{every node}+=[inner sep=0pt]
                    \draw [black] (26.7,-19.9) circle (3);
                    \draw (26.7,-19.9) node {$1$};
                    \draw [black] (37.9,-19.9) circle (3);
                    \draw (37.9,-19.9) node {$2$};
                    \draw [black] (48.8,-19.9) circle (3);
                    \draw (48.8,-19.9) node {$3$};
                    \draw [black] (29.7,-19.9) -- (34.9,-19.9);
                    \draw [black] (40.9,-19.9) -- (45.8,-19.9);
                    \draw [black] (28.753,-17.72) arc (130.51772:49.48228:13.849);
                \end{tikzpicture}
            \end{figure}
            \FloatBarrier
            Es wird eine Tiefensuche bei Knoten 1 gestartet und die Kante $(1, 2)$ zuerst traversiert und als nächstes Knoten 2 untersucht.
            Anschließend muss die Kante $(2, 3)$ traversiert werden, also wird nun Knoten 3 untersucht.
            Zu diesem Zeitpunkt sind alle Knoten grau.
            Daher wird Knoten 3 abgeschlossen.
            Anschließend wird Knoten 2 abgeschlossen.
            Bevor auch Knoten 1 abgeschlossen werden kann, muss noch Kante $(1, 3)$ überprüft werden.
            Zu diesem Zeitpunkt ist jedoch Knoten 3 bereits abgeschlossen und damit schwarz.
        \end{proof}
        \item Die Aussage ist richtig.
        \begin{proof}
            Beweis durch Widerspruch:
            Angenommen, eine Tiefensuche wird auf einem gerichteten, azyklischen Graphen ausgeführt und dabei wird eine Kante $(u, v)$ überprüft, wobei Knoten $v$ grau ist.
            Also wird gerade $u$ untersucht, während $v$ noch nicht abgeschlossen ist.
            Demnach gibt es einen direkten Pfad von $v$ nach $u$.
            Die Kante $(u, v)$ schließt somit einen Zyklus.
            Der Graph ist also nicht azyklisch, was einen Widerspruch zur Annahme darstellt.
        \end{proof}
    \end{enumerate}
\end{loesung}

\begin{aufgabe}{2}{Minimale Spannbäume}
    Gegeben seien folgende, ungerichtete Graphen:
    \begin{figure}[h!]
        \centering
        $G_1$:
        \begin{subfigure}{0.34\textwidth}
            \centering
            \begin{tikzpicture}[scale=0.15]
                \tikzstyle{every node}+=[inner sep=0pt]
                \draw [black] (20.9,-19.3) circle (3);
                \draw (20.9,-19.3) node {$1$};
                \draw [black] (34.9,-19.3) circle (3);
                \draw (34.9,-19.3) node {$2$};
                \draw [black] (21.1,-32.2) circle (3);
                \draw (21.1,-32.2) node {$4$};
                \draw [black] (48.6,-19.3) circle (3);
                \draw (48.6,-19.3) node {$3$};
                \draw [black] (48.6,-32.2) circle (3);
                \draw (48.6,-32.2) node {$6$};
                \draw [black] (34.9,-32.2) circle (3);
                \draw (34.9,-32.2) node {$5$};
                \draw [black] (23.9,-19.3) -- (31.9,-19.3);
                \draw (27.9,-18.8) node [above] {$1$};
                \draw [black] (20.95,-22.3) -- (21.05,-29.2);
                \draw (20.48,-25.75) node [left] {$1$};
                \draw [black] (48.6,-22.3) -- (48.6,-29.2);
                \draw (48.1,-25.75) node [left] {$2$};
                \draw [black] (32.71,-21.35) -- (23.29,-30.15);
                \draw (26.98,-25.27) node [above] {$2$};
                \draw [black] (37.08,-21.36) -- (46.42,-30.14);
                \draw (42.77,-25.27) node [above] {$3$};
                \draw [black] (24.1,-32.2) -- (31.9,-32.2);
                \draw (28,-31.7) node [above] {$3$};
                \draw [black] (34.9,-29.2) -- (34.9,-22.3);
                \draw (35.4,-25.75) node [right] {$4$};
                \draw [black] (37.9,-19.3) -- (45.6,-19.3);
                \draw (41.75,-18.8) node [above] {$4$};
                \draw [black] (37.9,-32.2) -- (45.6,-32.2);
                \draw (41.75,-31.7) node [above] {$4$};
            \end{tikzpicture}
        \end{subfigure}
        $G_2$:
        \begin{subfigure}{0.34\textwidth}
            \centering
            \begin{tikzpicture}[scale=0.13]
                \tikzstyle{every node}+=[inner sep=0pt]
                \draw [black] (16.1,-16.8) circle (3);
                \draw (16.1,-16.8) node {$1$};
                \draw [black] (32.1,-16.8) circle (3);
                \draw (32.1,-16.8) node {$2$};
                \draw [black] (16.1,-31.8) circle (3);
                \draw (16.1,-31.8) node {$4$};
                \draw [black] (32.1,-31.8) circle (3);
                \draw (32.1,-31.8) node {$5$};
                \draw [black] (47.8,-31.8) circle (3);
                \draw (47.8,-31.8) node {$6$};
                \draw [black] (47.8,-16.8) circle (3);
                \draw (47.8,-16.8) node {$3$};
                \draw [black] (16.1,-47.2) circle (3);
                \draw (16.1,-47.2) node {$7$};
                \draw [black] (32.1,-47.2) circle (3);
                \draw (32.1,-47.2) node {$8$};
                \draw [black] (47.8,-47.2) circle (3);
                \draw (47.8,-47.2) node {$9$};
                \draw [black] (34.27,-29.73) -- (45.63,-18.87);
                \draw (38.93,-23.82) node [above] {$1$};
                \draw [black] (44.8,-31.8) -- (35.1,-31.8);
                \draw (39.95,-31.3) node [above] {$2$};
                \draw [black] (47.8,-28.8) -- (47.8,-19.8);
                \draw (47.3,-24.3) node [left] {$3$};
                \draw [black] (19.1,-47.2) -- (29.1,-47.2);
                \draw (24.1,-46.7) node [above] {$4$};
                \draw [black] (18.29,-29.75) -- (29.91,-18.85);
                \draw (23.08,-23.82) node [above] {$5$};
                \draw [black] (19.1,-31.8) -- (29.1,-31.8);
                \draw (24.1,-31.3) node [above] {$6$};
                \draw [black] (16.1,-34.8) -- (16.1,-44.2);
                \draw (15.6,-39.5) node [left] {$7$};
                \draw [black] (32.1,-34.8) -- (32.1,-44.2);
                \draw (31.6,-39.5) node [left] {$8$};
                \draw [black] (35.1,-16.8) -- (44.8,-16.8);
                \draw (39.95,-16.3) node [above] {$9$};
                \draw [black] (19.1,-16.8) -- (29.1,-16.8);
                \draw (24.1,-16.3) node [above] {$10$};
                \draw [black] (16.1,-19.8) -- (16.1,-28.8);
                \draw (15.6,-24.3) node [left] {$11$};
                \draw [black] (47.8,-34.8) -- (47.8,-44.2);
                \draw (47.3,-39.5) node [left] {$13$};
                \draw [black] (35.1,-47.2) -- (44.8,-47.2);
                \draw (39.95,-46.7) node [above] {$14$};
                \draw [black] (34.24,-45.1) -- (45.66,-33.9);
                \draw (38.43,-39.02) node [above] {$12$};
                \draw [black] (18.26,-45.12) -- (29.94,-33.88);
                \draw (22.58,-39.02) node [above] {$15$};
                \draw [black] (32.1,-28.8) -- (32.1,-19.8);
                \draw (31.6,-24.3) node [left] {$16$};
            \end{tikzpicture}
        \end{subfigure}
    \end{figure}
    \FloatBarrier
    \begin{enumerate}[label=\alph*)]
        \item Demonstieren Sie den Algorithmus von Kruskal zum Finden von minimalen Spannbäumen an $G_1$ (und optional $G_2$) nach dem Schema aus der Vorlesung.
        Geben Sie für jeden Zwischenschritt den bisherigen Teil-MST sowie den aktuellen Zustand der Union-Find-Datenstruktur an.
        \item Demonstieren Sie den Algorithmus von Prim zum Finden von minimalen Spannbäumen an $G_1$ (und optional $G_2$) nach dem Schema aus der Vorlesung.
        Starten Sie jeweils bei Knoten 1.
        Geben Sie für jeden Zwischenschritt den bisherigen Teil-MST sowie den aktuellen Zustand des Min-Heaps an.
        \item Angenommen, für einen Graph $G=(V,E)$ gilt $\forall \{u, v\} \in E$: $w(u, v) \in \{1, 2, \ldots, |E|\}$.
        Wie kann man den Algorithmus von Kruskal in einem solchen Fall mithilfe eines anderen Sortieralgorithmus beschleunigen?
        Welche Laufzeit hat Kruskal dann?
        \item\label{connected_components}Ein ungerichteter (nicht unbedingt zusammenhängender) Graph lässt sich in disjunkte \emph{Zusammenhangskomponenten} aufteilen, wobei sich in der Komponente, in der der Knoten $v$ liegt, genau die von $v$ aus erreichbaren Knoten befinden.
        Implementieren Sie einen Algorithmus in Pseudocode oder einer Programmiersprache Ihrer Wahl auf Basis der Union-Find-Datenstruktur aus der Vorlesung, der alle Zusammenhangskomponenten eines ungerichteten Graphen findet.
        Welche Worst-Case-Laufzeit hat Ihr Algorithmus?
        \item
        Wie können Sie das Problem aus Teilaufgabe \ref*{connected_components} schneller, nämlich in $O(|V| + |E|)$, mit einem Ihnen aus der Vorlesung bekannten Graphalgorithmus lösen?
    \end{enumerate}
\end{aufgabe}

\begin{loesung}
    \begin{enumerate}
        \item 
        \item
        \item Man kann die Kantengewichte mittels Countsort sortieren.
        Alle zu sortierenden Werte liegen im Bereich $\{1, 2, \ldots, |E|\}$.
        Also gilt: $k = |E|$.
        Die Laufzeit von Countsort beträgt somit $O(n + k) = O(|E| + |E|) = O(|E|)$.
        Zum Bestimmen des minimalen Spannbaumes sind, zusätzlich zum Sortieren, im Worst Case noch $2|V|$ \textsc{Menge}-Operationen und $|V|$ \textsc{Vereinige}-Operationen nötig.
        Die Gesamtlaufzeit des Algorithmus beträgt also $O(|E| + |E| \log |V|) = O(|E| \log |V|)$.
        \item
        Es wird über alle Kanten $(u, v) \in E$ iteriert und die Mengen von $u$ und $v$ vereinigt.
        Schließlich sind alle Knoten einer Komponente in derselben Menge.
        \begin{algorithmic}[1]
            \Procedure{FindComponents}{$V, E$}
                \For{$v \in V$}
                    \State \textsc{ErzeugeMenge}$(v)$
                \EndFor
                \For{$(u, v) \in E$}
                    \If{$\text{\textsc{Menge}}(u) \neq \text{\textsc{Menge}}(v)$}
                        \State \textsc{Vereinige}$(u, v)$
                    \EndIf
                \EndFor
                \For{$v \in V$}
                    \State $v.\mathrm{component} \gets \text{\textsc{Menge}}(v)$
                \EndFor
            \EndProcedure
        \end{algorithmic}
        Sowohl das Erzeugen der Mengen als auch das Zuweisen der Komponenten benötigt Laufzeit $O(|V|)$.
        Für jede der $|E|$ Kanten wird zweimal \textsc{Menge} und einmal \textsc{Vereinige} aufgerufen (jeweils Laufzeit $O(\log |V|)$.
        Die Gesamtlaufzeit beträgt also $O(|V| + |E| \log |V|)$.

        \item Das Problem lässt sich mittels einer Tiefensuche lösen, indem man ähnlich vorgeht wie in Aufgabe 4a von Tutoriumsblatt 10.
        Sei $v$ der Knoten einer Komponente, der in der Tiefensuche als erstes entdeckt wird.
        Dann sind alle anderen Knoten der Komponente noch weiß.
        Außerdem sind sie erreichbar, da sie sich in der gleichen Komponente befinden.
        Also werden alle Knoten der Komponente im selben Tiefensuchendurchlauf wie $v$ entdeckt.
        Danach wird kein weiterer Tiefensuchendurchlauf auf einem Knoten der Komponente gestartet, da bereits alle Knoten entdeckt wurden.

        Der Algorithmus iteriert also wie gewohnt über alle Knoten.
        Ist ein Knoten noch nicht entdeckt, ist er der erste Knoten einer noch unetdeckten Komponente.
        Ein Tiefensuchendurchlauf wird von diesem Knoten aus gestartet und alle in diesem Durchlauf erreichten Knoten werden derselben Komponente zugewiesen.
        \begin{algorithmic}[1]
            \Procedure{FindComponentsDfs}{$V, E$}
                \For{$v \in V$}
                    \State $v.\mathrm{visited} \gets \text{\textbf{false}}$
                \EndFor
                \For{$v \in V$}
                    \If{$v.\mathrm{visited} = \text{\textbf{false}}$}
                        \State \textsc{Dfs}$(V, E, v, v)$
                    \EndIf
                \EndFor
            \EndProcedure
            \Procedure{Dfs}{$V, E, v, c$}
                \State $v.\mathrm{visited} \gets \text{\textbf{true}}$
                \State $v.\mathrm{component} \gets c$
                \For{$w \in v.\mathrm{adj}$}
                    \If{$w.\mathrm{visited} = \text{\textbf{false}}$}
                        \State \textsc{Dfs}$(V, E, w, c)$
                    \EndIf
                \EndFor
            \EndProcedure
        \end{algorithmic}
        Die Laufzeit ist die gleiche wie bei jeder Tiefensuche: $O(|V| + |E|)$.
    \end{enumerate}
\end{loesung}

\begin{aufgabe}{3}{Kürzeste Wege}
    Gegeben sei folgender, gerichteter Graph $G_3$:
    \begin{figure}[h!]
        \centering
        \begin{tikzpicture}[scale=0.15]
            \tikzstyle{every node}+=[inner sep=0pt]
            \draw [black] (18.6,-17.6) circle (3);
            \draw (18.6,-17.6) node {$a$};
            \draw [black] (31.8,-17.4) circle (3);
            \draw (31.8,-17.4) node {$b$};
            \draw [black] (45.4,-17.4) circle (3);
            \draw (45.4,-17.4) node {$c$};
            \draw [black] (18.6,-29.1) circle (3);
            \draw (18.6,-29.1) node {$d$};
            \draw [black] (31.8,-29.1) circle (3);
            \draw (31.8,-29.1) node {$e$};
            \draw [black] (45.4,-29.1) circle (3);
            \draw (45.4,-29.1) node {$f$};
            \draw [black] (21.6,-17.55) -- (28.8,-17.45);
            \fill [black] (28.8,-17.45) -- (27.99,-16.96) -- (28.01,-17.96);
            \draw (25.2,-16.98) node [above] {$4$};
            \draw [black] (18.6,-20.6) -- (18.6,-26.1);
            \fill [black] (18.6,-26.1) -- (19.1,-25.3) -- (18.1,-25.3);
            \draw (18.1,-23.35) node [left] {$2$};
            \draw [black] (20.85,-27.11) -- (29.55,-19.39);
            \fill [black] (29.55,-19.39) -- (28.62,-19.55) -- (29.29,-20.29);
            \draw (24.19,-22.76) node [above] {$1$};
            \draw [black] (31.8,-20.4) -- (31.8,-26.1);
            \fill [black] (31.8,-26.1) -- (32.3,-25.3) -- (31.3,-25.3);
            \draw (31.3,-23.25) node [left] {$2$};
            \draw [black] (21.6,-29.1) -- (28.8,-29.1);
            \fill [black] (28.8,-29.1) -- (28,-28.6) -- (28,-29.6);
            \draw (25.2,-29.6) node [below] {$4$};
            \draw [black] (30.372,-31.724) arc (-37.69112:-224.43443:9.416);
            \fill [black] (16.2,-19.37) -- (15.28,-19.6) -- (15.99,-20.3);
            \draw (15.72,-33.56) node [below] {$1$};
            \draw [black] (34.8,-17.4) -- (42.4,-17.4);
            \fill [black] (42.4,-17.4) -- (41.6,-16.9) -- (41.6,-17.9);
            \draw (38.6,-16.9) node [above] {$3$};
            \draw [black] (43.789,-19.929) arc (-35.78379:-62.80578:26.108);
            \fill [black] (43.79,-19.93) -- (42.92,-20.29) -- (43.73,-20.87);
            \draw (40.65,-24.95) node [below] {$1$};
            \draw [black] (33.471,-26.61) arc (143.10748:118.30295:28.307);
            \fill [black] (33.47,-26.61) -- (34.35,-26.27) -- (33.55,-25.67);
            \draw (36.64,-21.65) node [above] {$1$};
            \draw [black] (34.8,-29.1) -- (42.4,-29.1);
            \fill [black] (42.4,-29.1) -- (41.6,-28.6) -- (41.6,-29.6);
            \draw (38.6,-29.6) node [below] {$3$};
            \draw [black] (45.4,-20.4) -- (45.4,-26.1);
            \fill [black] (45.4,-26.1) -- (45.9,-25.3) -- (44.9,-25.3);
            \draw (45.9,-23.25) node [right] {$1$};
        \end{tikzpicture}
    \end{figure}
    \FloatBarrier
    \begin{enumerate}[label=\alph*)]
        \item Demonstieren Sie den Algorithmus von Bellman und Ford nach dem Schema aus der Vorlesung, indem Sie in $G_3$ alle kürzesten Wege, beginnend bei Knoten $a$, suchen.
        Durchlaufen Sie Kanten $(u, v)$ alphabetisch~(nach $u$ sortiert, bei gleichem $u$ nach $v$).
        Geben Sie alle Zwischenschritte an.
        \item Demonstieren Sie den Algorithmus von Dijkstra nach dem Schema aus der Vorlesung, indem Sie in $G_3$ alle kürzesten Wege, beginnend bei Knoten $a$, suchen.
        Geben Sie alle Zwischenschritte an, inklusive Min-Heaps.
        \item Suchen Sie nach demselben Schema mittels Dijkstra die kürzesten Wege in $G_2$, beginnend bei Knoten 1.
        \item Bei welchen Arten von Graphen ist Dijkstra mit einem Array statt einem Min-Heap potentiell effizienter?
        \item Zeigen Sie anhand eines Beispielgraphen mit 3 Knoten, dass Dijkstra bei negativen Kantengewichten nicht immer kürzeste Wege findet.
        \item Angenommen, die Kantengewichte eines Graphen werden angepasst: $w'(u, v) = w(u, v) - \min\limits_{(a, b) \in E} \{w(a, b) \}$.
        Die neuen Gewichte $w'(u, v)$ sind stets nicht-negativ, sodass prinzipiell Dijkstra auf den Graphen mit angepassten Gewichten angewendet werden könnte.
        Zeigen Sie anhand eines Beispiels, dass sich kürzeste Wege durch das Anpassen der Gewichte ändern können und somit die obige Anpassung nicht sinnvoll ist.
        \item\label{shortest_paths_dag} Implementieren Sie einen Algorithmus in Pseudocode oder einer Programmiersprache Ihrer Wahl, der alle kürzesten Wege in einem gerichteten, azyklischen Graphen beginnend bei einem Knoten $s$ in Laufzeit $O(|V| + |E|)$ findet.
        \begin{description}
            \item[Tipp:] Wenn Sie alle Kanten entlang eines kürzesten Weges in der Reihenfolge relaxieren, in der sie auf dem Weg vorkommen, haben Sie den kürzesten Weg gefunden.
        \end{description}
    \end{enumerate}
\end{aufgabe}
\begin{loesung}
    \begin{enumerate}
        \item 
        \item
        \item
        \item
        Bei dichten Graphen ($|E| = \Theta(|V|^2)$) ist Dijkstra mit einem Feld potentiell schneller als mittels Min-Heap.
        Dijkstra mittels Min-Heap hat Laufzeit $O(|V| \log |V| + |E| \log |V|)$.
        Setzt man $|V|^2$ für $|E|$ ein, erhält man $O(|V|^2 \log\left(|V|^2\right) = O(|V|^2 \log |V|)$.
        Bei einem Feld ist die Laufzeit von Dijkstra $O(|V|^2 + |E)$, was für dichte Graphen $O(|V|^2)$ entspricht.
        Die asymptotische Laufzeit ist also bei einem Feld um den Faktor $\log |V|$ schneller als bei einem Heap.
        Da der Logarithmus aber sehr langsam wächst, muss in der Praxis $|V|$ vermutlich sehr groß sein, damit dies eine wesentliche Laufzeitverbesserung bedeutet.
        \item \ \\
        \begin{figure}[h!]
            \centering
            \begin{tikzpicture}[scale=0.15]
                \tikzstyle{every node}+=[inner sep=0pt]
                \draw [black] (17.3,-17.2) circle (3);
                \draw (17.3,-17.2) node {$a$};
                \draw [black] (23.7,-8.5) circle (3);
                \draw (23.7,-8.5) node {$b$};
                \draw [black] (30.4,-17) circle (3);
                \draw (30.4,-17) node {$c$};
                \draw [black] (19.08,-14.78) -- (21.92,-10.92);
                \fill [black] (21.92,-10.92) -- (21.05,-11.26) -- (21.85,-11.86);
                \draw (19.92,-11.46) node [left] {$1$};
                \draw [black] (20.3,-17.15) -- (27.4,-17.05);
                \fill [black] (27.4,-17.05) -- (26.59,-16.56) -- (26.61,-17.56);
                \draw (23.85,-17.62) node [below] {$2$};
                \draw [black] (28.54,-14.64) -- (25.56,-10.86);
                \fill [black] (25.56,-10.86) -- (25.66,-11.79) -- (26.45,-11.17);
                \draw (27.62,-11.33) node [right] {$-2$};
            \end{tikzpicture}
        \end{figure}
        \FloatBarrier
        Wird in obigem Graphen mittels Dijkstra nach dem kürzesten Weg von $a$ nach $b$, werden die Knoten in der Reihenfolge $a$, $b$, $c$ abgearbeitet.
        Das bedeutet, $b$ wurde bereits aus dem Heap entfernt, wenn $c$ untersucht wird.
        Somit kann $b.\mathbf{dist}$ zu diesem Zeitpunkt nicht mehr aktualisiert werden.
        Dijkstra liefert also als kürzesten Pfad $(a, b)$ mit Länge 1, obwohl $(a, c, b)$ mit Länge 0 kürzer wäre.
        \item
        Betrachten wir folgenden Graphen, einmal mit den ursprünglichen und einmal mit den angepassten Gewichten:
        \begin{figure}[h!]
            \centering
            \begin{subfigure}{0.2\textwidth}
                \centering
                \begin{tikzpicture}[scale=0.15]
                    \tikzstyle{every node}+=[inner sep=0pt]
                    \draw [black] (17.3,-17.2) circle (3);
                    \draw (17.3,-17.2) node {$a$};
                    \draw [black] (23.7,-8.5) circle (3);
                    \draw (23.7,-8.5) node {$b$};
                    \draw [black] (30.4,-17) circle (3);
                    \draw (30.4,-17) node {$c$};
                    \draw [black] (19.08,-14.78) -- (21.92,-10.92);
                    \fill [black] (21.92,-10.92) -- (21.05,-11.26) -- (21.85,-11.86);
                    \draw (19.92,-11.46) node [left] {$-3$};
                    \draw [black] (20.3,-17.15) -- (27.4,-17.05);
                    \fill [black] (27.4,-17.05) -- (26.59,-16.56) -- (26.61,-17.56);
                    \draw (23.85,-17.62) node [below] {$-2$};
                    \draw [black] (28.54,-14.64) -- (25.56,-10.86);
                    \fill [black] (25.56,-10.86) -- (25.66,-11.79) -- (26.45,-11.17);
                    \draw (27.62,-11.33) node [right] {$-2$};
                \end{tikzpicture}
            \end{subfigure}
            $\rightarrow$
            \begin{subfigure}{0.2\textwidth}
                \centering
                \begin{tikzpicture}[scale=0.15]
                    \tikzstyle{every node}+=[inner sep=0pt]
                    \draw [black] (17.3,-17.2) circle (3);
                    \draw (17.3,-17.2) node {$a$};
                    \draw [black] (23.7,-8.5) circle (3);
                    \draw (23.7,-8.5) node {$b$};
                    \draw [black] (30.4,-17) circle (3);
                    \draw (30.4,-17) node {$c$};
                    \draw [black] (19.08,-14.78) -- (21.92,-10.92);
                    \fill [black] (21.92,-10.92) -- (21.05,-11.26) -- (21.85,-11.86);
                    \draw (19.92,-11.46) node [left] {$0$};
                    \draw [black] (20.3,-17.15) -- (27.4,-17.05);
                    \fill [black] (27.4,-17.05) -- (26.59,-16.56) -- (26.61,-17.56);
                    \draw (23.85,-17.62) node [below] {$1$};
                    \draw [black] (28.54,-14.64) -- (25.56,-10.86);
                    \fill [black] (25.56,-10.86) -- (25.66,-11.79) -- (26.45,-11.17);
                    \draw (27.62,-11.33) node [right] {$1$};
                \end{tikzpicture}
            \end{subfigure}
        \end{figure}
        \FloatBarrier
        Der kürzeste Weg von $a$ nach $b$ nach den ursprünglichen Gewichten ist $(a, c, b)$.
        Nach den angepassten Gewichten ist es jedoch $(a, b)$.

        \item
        Wir durchlaufen alle Knoten $u \in V$ in topologischer Sortierung und relaxieren für jeden Knoten $u$ die Kanten $(u, v) \in u.\mathrm{adj}$.
        So garantieren wir, dass wir alle Kanten des kürzesten Weges in der Reihenfolge relaxieren, in der sie im kürzesten Weg auftauchen (siehe Tipp), da in topologischer Reihenfolge keine Kante \quotes{zurück} führt.
        \begin{algorithmic}[1]
            \Procedure{ShortestPathsDag}{V, E, s}
                \For{$v \in V$}
                    \State $v.\mathrm{dist} \gets \infty$
                    \State $v.\mathrm{pred} \gets \mathrm{NULL}$
                \EndFor
                \State $s.\mathrm{dist} \gets 0$
                \State $S \gets \text{\textsc{TopologicalSort}}(V, E)$
                \For{$u \in S$}
                    \For{$v \in u.\mathrm{adj}$}
                        \If{$u.\mathrm{dist} + w(u, v) < v.\mathrm{dist}$}
                            \State $v.\mathrm{dist} = u.\mathrm{dist} + w(u, v)$
                            \State $v.\mathrm{pred} = u$
                        \EndIf
                    \EndFor
                \EndFor
            \EndProcedure
        \end{algorithmic}
        Die erste Schleife hat Laufzeit $O(|V|)$.
        Der Algorithmus zum Finden einer topologischen Sortierung benötigt Laufzeit $O(|V| + |E|)$.
        In der letzten Schleife durchlaufen wir alle Kanten des Graphen.
        Daher beträgt die Laufzeit hierfür ebenfalls $O(|V| + |E|)$ (vergl. Vorlesungsunterlagen).
        Die Gesamtlaufzeit des Algorithmus ist demnach $O(|V| + |E|)$, was schneller ist als Dijkstra oder Bellman-Ford.
    \end{enumerate}
\end{loesung}
\begin{aufgabe}{4}{Längste Wege}
    Der \emph{längste Weg} zwischen zwei Knoten $v$ und $w$ ist der längste, zyklenfreie (!) Pfad von $v$ nach $w$.
    \begin{enumerate}[label=\alph*)]
        \item Widerlegen Sie folgende Aussage, zum Beispiel mithilfe eines Gegenbeispiels:
        \emph{Längste Wege bilden eine optimale Substruktur. Das heißt, längste Wege enhalten stets längste Wege.}
        \item\label{longest_path_dag}Zeigen Sie folgende Aussage:
        \emph{Längste Wege in einem gerichteten, azyklischen Graphen bilden eine optimale Substruktur.}
        \item Beschreiben Sie kurz, wie Sie mit einem Algorithmus aus der Vorlesung längste Wege beginnend bei einem Knoten $s$ in einem gerichteten, azyklischen Graphen finden können.
        Geben Sie die Laufzeit des Algorithmus an.
    \end{enumerate}
\end{aufgabe}
\begin{loesung}
    \begin{enumerate}
        \item 
        \begin{proof}
            Beweis durch Gegenbeispiel:
            \begin{figure}[h!]
                \centering
                \begin{tikzpicture}[scale=0.15]
                    \tikzstyle{every node}+=[inner sep=0pt]
                    \draw [black] (24,-10.7) circle (3);
                    \draw (24,-10.7) node {$a$};
                    \draw [black] (34.1,-10.7) circle (3);
                    \draw (34.1,-10.7) node {$b$};
                    \draw [black] (34.1,-20.7) circle (3);
                    \draw (34.1,-20.7) node {$c$};
                    \draw [black] (24,-20.7) circle (3);
                    \draw (24,-20.7) node {$d$};
                    \draw [black] (27,-10.7) -- (31.1,-10.7);
                    \draw (29.05,-10.2) node [above] {$1$};
                    \draw [black] (34.1,-13.7) -- (34.1,-17.7);
                    \draw (34.6,-15.7) node [right] {$1$};
                    \draw [black] (31.1,-20.7) -- (27,-20.7);
                    \draw (29.05,-21.2) node [below] {$1$};
                    \draw [black] (24,-17.7) -- (24,-13.7);
                    \draw (23.5,-15.7) node [left] {$1$};
                \end{tikzpicture}
            \end{figure}
            \FloatBarrier
            Der längste Weg von $a$ nach $b$ ist $(a, d, c, b)$ mit Länge 3.
            Würden längste Wege stets längste Wege enthalten müsste $(d, c)$ mit Länge 1 der längste Weg von $d$ nach $c$ sein.
            Der längste Weg von $d$ nach $c$ ist jedoch $(d, a, b, c)$ mit Länge 3.
            Würde man diesen längsten Weg in den ursprünglichen Weg von $a$ nach $b$ einsetzen, wäre der resultierende Pfad $(a, d, a, b, c, b)$ nicht mehr zyklenfrei und somit per Definition kein kürzester Weg mehr.
        \end{proof}
        \item
        Der Beweis verläuft analog zum Beweis der optimalen Substruktur kürzester Wege, nur das zusätzlich gezeigt werden musst, dass der \quotes{verlängerte} Pfad zyklenfrei ist, also einen Kandidat für einen längsten Weg ist.
        Da jedoch gerichtete, azyklische Graphen per Definition keine Zyklen besitzen, ist dieser zusätzliche Schritt nicht schwierig.
        \begin{proof}
            Beweis durch Widerspruch:
            Sei $P = (v_1, v_2, \ldots, v_i, v_{i + 1} \ldots, v_{j - 1}, v_j, \ldots, v_{n - 1}, v_n)$ ein längster Weg von $v_1$ nach $v_n$ ($1 \leq i < j \leq n$) und sei $P_{i, j} = (v_i, v_{i + 1}, \ldots, v_{j - 1}, v_j)$ \emph{kein} längster Weg von $v_i$ nach $v_j$.
            Sei stattdessen $P' = (v_i, v_1', v_2', \ldots, v_{m - 1}', v_m', v_j)$ ein längster Weg von $v_i$ nach $v_j$.
            Es gilt also: $w(P_{i, j}) < w(P')$.

            Betrachten wir den Pfad $\hat{P} = (v_1, v_2, \ldots, v_i, v_1' \ldots, v_m', v_j, \ldots, v_{n - 1}, v_n)$.
            $\hat{P}$ muss zyklenfrei sein, da es sich um einen zusammenhängenden Pfad handelt und der Graph zyklenfrei ist.
            Es kann also kein Knoten doppelt vorkommen.
            Außerdem gilt:
            \begin{align*}
                w(P) = w(P_{1,i}) + w(P_{i, j}) + w(P(j, n))
                < w(P_{1,i}) + w(P') + w(P(j, n)) = w(\hat{P})
            \end{align*}
            Das Gewicht von $\hat{P}$ ist also größer als das von $P$.
            Das ist ein Widerspruch zur Annahme, dass $P$ ein längster Pfad ist.
        \end{proof}
        \item
        Wir negieren alle Kantengewichte ($w'(u, v) = -w(u, v)$) und wenden den Algorithmus nach Bellman und Ford an.
        Da nach Teilaufgabe \ref*{longest_path_dag} längste Pfade in einem gerichteteten, azyklischen Graphen eine optimale Substruktur bilden, können wir den Korrektheitsbeweis von Bellman-Ford (siehe Cormen 24.1) anwenden, um zu zeigen, dass wir so längste Wege finden.
        Dijkstra können wir nicht anwenden, da $w'(u, v)$ negativ sein kann und vermutlich auch ist.
        Alternativ können wir den Algorithmus aus Aufgabe 3\ref*{shortest_paths_dag} verwenden.
        Damit erreichen wir sogar Laufzeit $O(|V| + |E|)$, was deutlich besser ist als $O(|V| \cdot |E|)$ bei Bellman-Ford.
    \end{enumerate}
\end{loesung}

\end{document}