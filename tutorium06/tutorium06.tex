\documentclass[11pt,a4paper]{article}

\usepackage{gastex}
\usepackage{etoolbox}
% \newcommand{\showLoesung}{2} %<---als Schalter
%\newcommand{\showInhalt}{1} %<---als Schalter

\usepackage{alltt,moreverb,amsmath,enumerate}
\usepackage[normalem]{ulem}
\usepackage[T1]{fontenc}
\usepackage{ae,aecompl} %helvet,mathptm
%\usepackage[left=15mm,right=15mm,top=20mm,bottom=20mm]{geometry}
\usepackage[margin=.5in]{geometry}
%\usepackage[latin1]{inputenc} % f�r Linux
\usepackage[utf8]{inputenc} % Umlaute etc. direkt schreiben (unter Windows)
\usepackage[german]{babel}
\usepackage[url]{oth-logoPNG}
%\usepackage{i2sym,i2ams}

\usepackage{tikz}
\usetikzlibrary{arrows,shapes,trees,positioning,automata,decorations.pathreplacing,decorations.pathmorphing}
\usepackage{tkz-graph}
\usepackage{color}

\usepackage{longtable}
\usepackage{tabularx}

%\usepackage{epic}
%\usepackage{eepic}
\usepackage{comment,ifthen}
\usepackage{../include/todo}

\usepackage[T1]{fontenc}
\usepackage{textcomp}

\usepackage{listings}                   % Listings in Core-Erlang und Maude
\usepackage{lstmisc}

\usepackage{epic}                       % Bildbefehle (picture)
%\usepackage{eepic}                      % erweiterte Bildbefehle

\usepackage{bbm}                        % Mengensymbole (N,C,R,B)
\usepackage{latexsym}                   % zusaetzliche Mathesymbole
\usepackage{amsmath}                    % Mathepaket von der AMS
\usepackage{amstext}
\usepackage{amsfonts}
\usepackage{stmaryrd}                   % zusaetzliche Mathesymbole
\usepackage{mathtools}
\usepackage{amsthm}
\usepackage{cancel}

\usepackage{hyperref}
\usepackage{url}                        % Zum Setzen von URLs in typewriter-face

\pagestyle{empty}

\let\epsilon=\varepsilon
\let\phi=\varphi

\frenchspacing

\setlength{\parindent}{0pt}
\setlength{\textwidth}{18.6cm}
\setlength{\textheight}{26.5cm}
\setlength{\hfuzz}{1mm}

%%% Read dates of assignments from file
\usepackage{xparse}
\ExplSyntaxOn
\ior_new:N \g_hringriin_file_stream

\NewDocumentCommand{\ReadFile}{mm}
 {
  \hringriin_read_file:nn { #1 } { #2 }
  \cs_new:Npn #1 ##1
   {
    \str_if_eq:nnTF { ##1 } { * }
      { \seq_count:c { g_hringriin_file_ \cs_to_str:N #1 _seq } }
      { \seq_item:cn { g_hringriin_file_ \cs_to_str:N #1 _seq } { ##1 } }
   }
 }

\cs_new_protected:Nn \hringriin_read_file:nn
 {
  \ior_open:Nn \g_hringriin_file_stream { #2 }
  \seq_gclear_new:c { g_hringriin_file_ \cs_to_str:N #1 _seq }
  \ior_map_inline:Nn \g_hringriin_file_stream
   {
    \seq_gput_right:cx 
     { g_hringriin_file_ \cs_to_str:N #1 _seq }
     { \tl_trim_spaces:n { ##1 } }
   }
  \ior_close:N \g_hringriin_file_stream
 }

\ExplSyntaxOff

\ReadFile{\uebungsabgabe}{../skel/UEBUNGSABGABE.def}

%%% Read subject info from file
\newcommand{\dozent}[1]{\def\DOZENT{#1}}
\newcommand{\tutoren}[1]{\def\TUTOREN{#1}}
\newcommand{\vorlesung}[1]{\def\VORLESUNG{#1}}
\newcommand{\semester}[1]{\def\SEMESTER{#1}}

\InputIfFileExists{../skel/VORLESUNG.def}{\providecommand{\TUTOREN}{}}%
{\typeout{***********}
 \typeout{Warnung: Kein File vorhanden, das die Vorlesung spezifiziert!}
 \typeout{Spezifikation muss daher im Text des Blattes oder ueber die
          Tastatur erfolgen.}
 \typeout{***********}}

\def\Uebung#1#2#3{
  \othLehrstuhlLogo[\DOZENT]
  \begin{center}
	{~\\[-2em]\Large\bf \VORLESUNG}\\[0.5em]
    \LARGE --~Tutorium #1 (Übung #2)~--\\[4mm]
  \
  \normalsize
  \textbf{#3}
    \rule{\textwidth}{0.1pt}\\[1cm]
  \end{center}
}

\def\Hinweis#1{
	{~\\[-3em]\bf Hinweis: }
	\begin{minipage}[t]{16.5cm}
	#1
	\end{minipage}\\[1em]
    \rule{\textwidth}{0.1pt}
}

\def\Tipps#1{
	{~\\[-3em]\bf Tipps: }
	\begin{minipage}[t]{16.5cm}
	#1
	\end{minipage}\\[1em]
    \rule{\textwidth}{0.1pt}
}
  
\def\MyHeader{
  \othLehrstuhlLogo[Prof.~Dr.~rer.~nat.~Carsten~Kern]%[Carsten~Kern,~Stefan~Rieger]
}

\newcommand{\sem}[1]{[\![#1\,]\!]}

\def\aufgabe#1#2{\subsection*{Aufgabe #1 (#2)}\par}
\def\endaufgabe{}

\newenvironment{loesung}{\subsection*{L\"osungsvorschlag:}}{}
\newenvironment{hinweis}{}{}
\ifthenelse{\isundefined{\showLoesung}}{\excludecomment{loesung}}{\pagestyle{plain}\excludecomment{hinweis}}

\newenvironment{tipps}{}{}
\ifthenelse{\isundefined{\showTipps}}{\excludecomment{tipps}}{\excludecomment{hinweis}}

\newenvironment{inhalt}{\subsection*{Kommentar:}}{}
\ifthenelse{\isundefined{\showInhalt}}{\excludecomment{inhalt}}{}

\long\def\Exercise#1#2{\begin{exercise}{#1}#2\end{exercise}}

\def\underbar#1{%
  \setbox0=\hbox{#1}%
  \dimen0=\dp0\relax%
  \dp0=0pt%
  \setbox0=\hbox{\underline{\box0}}%
  \dp0=\dimen0\relax%
  \box0%
  }

\makeatletter
\def\@makeunderbar[#1]#2{\expandafter\def\csname#1\endcsname{\underbar{#2}}}
\def\makeunderbar{\@ifnextchar[{\@makeunderbar}{\@makeunderbar[]}}
\makeatother

\def\T{\mathrm{T}}
\def\P{\mathrm{P}}
\def\CT{\mathrm{CT}}
\def\COp{\mathrm{COp}}

\makeunderbar{Comp}
\makeunderbar{Ops}
\makeunderbar{trans}
\makeunderbar[strans]{s-trans}
\makeunderbar[ntrans]{n-trans}
\makeunderbar{fix}

\def\labelenumi{\alph{enumi})}
\let\<=\langle
\let\>=\rangle

\parindent=0pt
\parskip=1ex

\definecolor{javared}{rgb}{0.6,0,0} % for strings
\definecolor{javagreen}{rgb}{0.25,0.5,0.35} % comments
\definecolor{javapurple}{rgb}{0.5,0,0.35} % keywords
\definecolor{javadocblue}{rgb}{0.25,0.35,0.75} % javadoc
 
\lstset{language=Java,
basicstyle=\ttfamily\footnotesize,
keywordstyle=\color{javapurple}\bf,
stringstyle=\color{javared},
commentstyle=\color{javagreen}\it\bf,
morecomment=[s][\color{javadocblue}]{/**}{*/},
numbers=left,
numberstyle=\tiny\color{gray},
stepnumber=1,
numbersep=10pt,
tabsize=3,
showspaces=false,
showstringspaces=false}

\usepackage{enumitem}
\usepackage{algpseudocode}
\usepackage{caption}
\usepackage{subcaption}
\usepackage{placeins}
\usepackage{multicol}

\begin{document}
\thispagestyle{empty}

\Uebung{6}{7}{Simon Thelen}{18. November 2021}  % FIXME: Blattnummer, Datum, Zeit

%%%%%%%%%%%%%%%%%%%%%%%%%%%%%%%%%%%%%%%%%%%%%%%%%%%%%%%%%%%%%%%%%%%%%%

\ifcsdef{showLoesung}{
\textbf{Bitte beachten Sie:} Die Lösungen können trotz sorgfältiger Prüfung Fehler enthalten.
Bei Fragen oder Unklarheiten kontaktieren Sie bitte den Tutor oder Dozenten in Tutorien, Übungen oder nach Vorlesungen.
}{}


\begin{aufgabe}{1}{Binäre, verkettete Bäume}
    \begin{enumerate}
        \item Gegeben sei ein binärer, verketteter Baum mit Preorder-Darstellung (a, b, c, d, e, f, g, h, i) und Inorder-Darstellung (c, d, b, e, a, h, g, i, f).
        Wie sieht der Baum aus?
        \item Angenommen Sie haben die Preorder- und die Postorder-Darstellung eines binären Baumes gegeben.
        Können Sie (wie bei der vorherigen Teilaufgabe) den Baum eindeutig rekonstruieren?
        Falls ja, begründen Sie dies.
        Falls nein, geben Sie ein Gegenbeispiel an.
        \item Betrachten Sie folgende Implementierung eines Inorder-Durchlaufs über einen binären, verketteten Baum:
        \begin{lstlisting}[language=c++]
void inorder(Node *node) {
    if (node->left != NULL) inorder(node->left);
    cout << node->value << " ";
    if (node->right != NULL) inorder(node->right);
}
        \end{lstlisting}
        Begründen Sie, dass die Laufzeit dieser Funktion für einen Baum mit $n$ Elementen $O(n)$ beträgt, indem Sie einen Zusammenhang zwischen Kanten des Baumes und den rekursiven Aufrufen des Algorithmus herstellen.   
        % \item
        % Implementieren Sie einen Preorder-Durchlauf über einen binären, verketten Baum (in Pseudocode oder einer Programmiersprache Ihrer Wahl), der auf Rekursion verzichtet, aber stattdessen einen Stack verwendet.
        \item 
        Geben Sie einen Algorithmus in Pseudocode oder einer Programmiersprache Ihrer Wahl an, welcher einen binären, verketteten Baum schichtweise ausgibt.
        Also zunächst den Wurzelknoten, dann die Kindknoten der Wurzel von links nach rechts, dann alle Enkelknoten der Wurzel von links nach rechts und so weiter.
        \begin{description}
            \item[Tipp:] Verwenden Sie eine Queue.
        \end{description}
        \item
        Der Abstand von zwei Knoten in einem Baum ist die Anzahl an Kanten, die mindestens traversiert werden müssen, um von einem Knoten zum anderen zu gelangen.
        \emph{Beispiel:} Die Knoten 3 und 13 im Baum aus Aufgabe 2\ref*{insert_delete} haben Abstand 6.

        Geben Sie einen Algorithmus in Pseudocode oder einer Programmiersprache Ihrer Wahl an, welcher einen binären, verketteten Baum als Eingabe erhält und den größten Abstand von zwei Knoten im Baum liefert.
        Eine einfache Lösung hat Laufzeit $O(n^2)$.
        Schaffen Sie es sogar in $O(n)$?
    \end{enumerate}
\end{aufgabe}
\begin{loesung}
    \begin{enumerate}
        \item \ \\
        \begin{figure}[h!]
            \centering
            \includegraphics[width=0.3\textwidth]{img/1a}
        \end{figure}
        \FloatBarrier
        \item Eine eindeutige Rekonstruktion ist allein bei gegebener Preorder- und Postorder-Darstellung im Allgemeinen nicht möglich.
        Das kleinste Gegenbeispiel ist Preorder (1, 2) und Postorder (2, 1).
        Beide der folgenden Bäume haben genau diese Darstellung:
        \begin{figure}[h!]
            \centering
            \begin{subfigure}[b]{0.23\textwidth}
                \centering
                \includegraphics[width=\textwidth]{img/a1}
            \end{subfigure}
            \begin{subfigure}[b]{0.23\textwidth}
                \centering
                \includegraphics[width=\textwidth]{img/a2}
            \end{subfigure}
        \end{figure}
        \FloatBarrier
        Das kleinste Gegenbeispiel ist Preorder (1, 2) und Postorder (2, 1).
        Ein Beispiel ist Preorder (1, 2, 3, 4) und Postorder (4, 3, 2, 1).
        Sämtliche entarteten Bäume (Bäume, bei denen jeder Knoten maximal einen Nachfolger hat) mit 4 Knoten passen auf diese Darstellung.
        So etwa die 
    \end{enumerate}
\end{loesung}

\begin{aufgabe}{2}{Binäre, verkettete Suchbäume}
    \begin{enumerate}[label=\alph*)]
        \item Gegeben sei ein binärer, verketteter Suchbaum mit Postorder-Darstellung (15, 28, 12, 33, 41, 37, 30, 50, 42).
        Wie sieht der Baum aus?
        % Quelle: Cormen 12.2-1
        \item Angenommen, Sie suchen den Wert 75 in einem binären, verketteten Suchbaum.
        Welche der folgenden Pfade von der Wurzel bis zum gesuchten Knoten ist nicht möglich: (51, 52, 97, 80, 72, 76, 75) oder (89, 80, 24, 63, 81, 69, 75)?
        Begründen Sie Ihre Entscheidung.
        \item Gegeben seien die Werte (29, 33, 45, 50, 77, 86, 96).
        Geben Sie binäre, verkette Suchbäume mit Höhe 2, 3, 4, 5 und 6 an, die jeweils genau diese Elemente enthalten.
        \item\label{insert_delete} Gegeben sei folgender binärer, verketteter Suchbaum:
        \begin{figure}[h!]
            \centering
            \includegraphics[width=0.3\textwidth]{img/2d}
        \end{figure}
        \FloatBarrier
        \ \\
        Führen Sie nacheinander folgende Operationen durch: \textsc{Delete(15), Delete(1), Delete(9), Insert(1), Insert(12)}.
        Geben Sie alle Zwischenschritte an.
        \item 
        Geben Sie einen Algorithmus in Pseudocode oder einer Programmiersprache Ihrer Wahl an, der einen binären, verketteten Suchbaum sowie zwei Werte $l$ und $h$ als Eingabe erhält und anschließend alle Werte $x$ des Baumes ausgibt, für die $l \leq x \leq h$ gilt.
        Versuchen Sie so gut wie möglich, nur selektiert über den Teil des Baumes zu iterieren, der für die Ausgabe relevant ist.
    \end{enumerate}
\end{aufgabe}

\begin{aufgabe}{3}{AVL-Bäume}
    \begin{enumerate}
        \item Gegeben sei folgender AVL-Baum:
        \begin{figure}[h!]
            \centering
            \includegraphics[width=0.3\textwidth]{img/3a}
        \end{figure}
        \FloatBarrier
        \ \\
        Führen Sie nacheinander folgende Operationen durch: \textsc{Insert(5), Insert(3), Insert(38), Insert(2), Delete(3), Delete(38), Delete(42)}.
        Geben Sie alle Zwischenschritte und Rotationen an.
        % Quelle: Cormen Aufgabe 12.1-1
        \item Begründen Sie oder widerlegen Sie die folgende Aussage:
        \begin{center}
            \emph{Bei jedem }\textsc{Insert}\emph{-Aufruf ist maximal eine (Doppel-)Rotation notwendig}
        \end{center}
        \item Geben Sie ein Beispiel für das Löschen eines Knotens an, bei dem mehr als eine (Doppel-)Rotation benötigt wird.
        \begin{description}
            \item[Tipp:] Sie können zum Beispiel den minimalen AVL-Baum der Höhe 4 verwenden.
        \end{description}
        \item Implementieren Sie die folgenden Operationen, die Sie im Rahmen der Heap-Datenstruktur kennen gelernt haben, mithilfe eines AVL-Baumes (in Pseudocode oder einer Programmiersprache Ihrer Wahl).
        Geben Sie außerdem an, wie Sie die AVL-Datenstruktur aus der Vorlesung erweitern müssen, um die angegebenen Laufzeiten zu erreichen.
        \begin{table}[h!]
            \centering
            \begin{tabular}{|l|l|l|}
            \hline
            \multicolumn{1}{|c|}{\textbf{Operation}}  & \multicolumn{1}{c|}{\textbf{Laufzeit}} & \multicolumn{1}{c|}{\textbf{Beschreibung}}\\ \hline
            \textsc{GetMaximum} & $O(1)$      & Gibt den größten Wert im Baum zurück.\\ \hline
            \textsc{ExtractMax} & $O(\log n)$ & Entfernt den größten Wert aus dem Baum. \\ \hline
            \textsc{Insert}     & $O(\log n)$ & Fügt einen neuen Wert in den Baum ein. \\ \hline
            \end{tabular}
        \end{table}
    \end{enumerate}
\end{aufgabe}

\begin{loesung}
    \begin{enumerate}
        \item \ \\
        \begin{figure}[h!]
            \centering
            \includegraphics[width=0.3\textwidth]{img/2a}
        \end{figure}
        \FloatBarrier
    \end{enumerate}
\end{loesung}


\end{document}