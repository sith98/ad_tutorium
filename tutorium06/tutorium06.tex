\documentclass[11pt,a4paper]{article}

\usepackage{gastex}
\usepackage{etoolbox}
\newcommand{\showLoesung}{2} %<---als Schalter
%\newcommand{\showInhalt}{1} %<---als Schalter

\usepackage{alltt,moreverb,amsmath,enumerate}
\usepackage[normalem]{ulem}
\usepackage[T1]{fontenc}
\usepackage{ae,aecompl} %helvet,mathptm
%\usepackage[left=15mm,right=15mm,top=20mm,bottom=20mm]{geometry}
\usepackage[margin=.5in]{geometry}
%\usepackage[latin1]{inputenc} % f�r Linux
\usepackage[utf8]{inputenc} % Umlaute etc. direkt schreiben (unter Windows)
\usepackage[german]{babel}
\usepackage[url]{oth-logoPNG}
%\usepackage{i2sym,i2ams}

\usepackage{tikz}
\usetikzlibrary{arrows,shapes,trees,positioning,automata,decorations.pathreplacing,decorations.pathmorphing}
\usepackage{tkz-graph}
\usepackage{color}

\usepackage{longtable}
\usepackage{tabularx}

%\usepackage{epic}
%\usepackage{eepic}
\usepackage{comment,ifthen}
\usepackage{../include/todo}

\usepackage[T1]{fontenc}
\usepackage{textcomp}

\usepackage{listings}                   % Listings in Core-Erlang und Maude
\usepackage{lstmisc}

\usepackage{epic}                       % Bildbefehle (picture)
%\usepackage{eepic}                      % erweiterte Bildbefehle

\usepackage{bbm}                        % Mengensymbole (N,C,R,B)
\usepackage{latexsym}                   % zusaetzliche Mathesymbole
\usepackage{amsmath}                    % Mathepaket von der AMS
\usepackage{amstext}
\usepackage{amsfonts}
\usepackage{stmaryrd}                   % zusaetzliche Mathesymbole
\usepackage{mathtools}
\usepackage{amsthm}
\usepackage{cancel}

\usepackage{hyperref}
\usepackage{url}                        % Zum Setzen von URLs in typewriter-face

\pagestyle{empty}

\let\epsilon=\varepsilon
\let\phi=\varphi

\frenchspacing

\setlength{\parindent}{0pt}
\setlength{\textwidth}{18.6cm}
\setlength{\textheight}{26.5cm}
\setlength{\hfuzz}{1mm}

%%% Read dates of assignments from file
\usepackage{xparse}
\ExplSyntaxOn
\ior_new:N \g_hringriin_file_stream

\NewDocumentCommand{\ReadFile}{mm}
 {
  \hringriin_read_file:nn { #1 } { #2 }
  \cs_new:Npn #1 ##1
   {
    \str_if_eq:nnTF { ##1 } { * }
      { \seq_count:c { g_hringriin_file_ \cs_to_str:N #1 _seq } }
      { \seq_item:cn { g_hringriin_file_ \cs_to_str:N #1 _seq } { ##1 } }
   }
 }

\cs_new_protected:Nn \hringriin_read_file:nn
 {
  \ior_open:Nn \g_hringriin_file_stream { #2 }
  \seq_gclear_new:c { g_hringriin_file_ \cs_to_str:N #1 _seq }
  \ior_map_inline:Nn \g_hringriin_file_stream
   {
    \seq_gput_right:cx 
     { g_hringriin_file_ \cs_to_str:N #1 _seq }
     { \tl_trim_spaces:n { ##1 } }
   }
  \ior_close:N \g_hringriin_file_stream
 }

\ExplSyntaxOff

\ReadFile{\uebungsabgabe}{../skel/UEBUNGSABGABE.def}

%%% Read subject info from file
\newcommand{\dozent}[1]{\def\DOZENT{#1}}
\newcommand{\tutoren}[1]{\def\TUTOREN{#1}}
\newcommand{\vorlesung}[1]{\def\VORLESUNG{#1}}
\newcommand{\semester}[1]{\def\SEMESTER{#1}}

\InputIfFileExists{../skel/VORLESUNG.def}{\providecommand{\TUTOREN}{}}%
{\typeout{***********}
 \typeout{Warnung: Kein File vorhanden, das die Vorlesung spezifiziert!}
 \typeout{Spezifikation muss daher im Text des Blattes oder ueber die
          Tastatur erfolgen.}
 \typeout{***********}}

\def\Uebung#1#2#3{
  \othLehrstuhlLogo[\DOZENT]
  \begin{center}
	{~\\[-2em]\Large\bf \VORLESUNG}\\[0.5em]
    \LARGE --~Tutorium #1 (Übung #2)~--\\[4mm]
  \
  \normalsize
  \textbf{#3}
    \rule{\textwidth}{0.1pt}\\[1cm]
  \end{center}
}

\def\Hinweis#1{
	{~\\[-3em]\bf Hinweis: }
	\begin{minipage}[t]{16.5cm}
	#1
	\end{minipage}\\[1em]
    \rule{\textwidth}{0.1pt}
}

\def\Tipps#1{
	{~\\[-3em]\bf Tipps: }
	\begin{minipage}[t]{16.5cm}
	#1
	\end{minipage}\\[1em]
    \rule{\textwidth}{0.1pt}
}
  
\def\MyHeader{
  \othLehrstuhlLogo[Prof.~Dr.~rer.~nat.~Carsten~Kern]%[Carsten~Kern,~Stefan~Rieger]
}

\newcommand{\sem}[1]{[\![#1\,]\!]}

\def\aufgabe#1#2{\subsection*{Aufgabe #1 (#2)}\par}
\def\endaufgabe{}

\newenvironment{loesung}{\subsection*{L\"osungsvorschlag:}}{}
\newenvironment{hinweis}{}{}
\ifthenelse{\isundefined{\showLoesung}}{\excludecomment{loesung}}{\pagestyle{plain}\excludecomment{hinweis}}

\newenvironment{tipps}{}{}
\ifthenelse{\isundefined{\showTipps}}{\excludecomment{tipps}}{\excludecomment{hinweis}}

\newenvironment{inhalt}{\subsection*{Kommentar:}}{}
\ifthenelse{\isundefined{\showInhalt}}{\excludecomment{inhalt}}{}

\long\def\Exercise#1#2{\begin{exercise}{#1}#2\end{exercise}}

\def\underbar#1{%
  \setbox0=\hbox{#1}%
  \dimen0=\dp0\relax%
  \dp0=0pt%
  \setbox0=\hbox{\underline{\box0}}%
  \dp0=\dimen0\relax%
  \box0%
  }

\makeatletter
\def\@makeunderbar[#1]#2{\expandafter\def\csname#1\endcsname{\underbar{#2}}}
\def\makeunderbar{\@ifnextchar[{\@makeunderbar}{\@makeunderbar[]}}
\makeatother

\def\T{\mathrm{T}}
\def\P{\mathrm{P}}
\def\CT{\mathrm{CT}}
\def\COp{\mathrm{COp}}

\makeunderbar{Comp}
\makeunderbar{Ops}
\makeunderbar{trans}
\makeunderbar[strans]{s-trans}
\makeunderbar[ntrans]{n-trans}
\makeunderbar{fix}

\def\labelenumi{\alph{enumi})}
\let\<=\langle
\let\>=\rangle

\parindent=0pt
\parskip=1ex

\definecolor{javared}{rgb}{0.6,0,0} % for strings
\definecolor{javagreen}{rgb}{0.25,0.5,0.35} % comments
\definecolor{javapurple}{rgb}{0.5,0,0.35} % keywords
\definecolor{javadocblue}{rgb}{0.25,0.35,0.75} % javadoc
 
\lstset{language=Java,
basicstyle=\ttfamily\footnotesize,
keywordstyle=\color{javapurple}\bf,
stringstyle=\color{javared},
commentstyle=\color{javagreen}\it\bf,
morecomment=[s][\color{javadocblue}]{/**}{*/},
numbers=left,
numberstyle=\tiny\color{gray},
stepnumber=1,
numbersep=10pt,
tabsize=3,
showspaces=false,
showstringspaces=false}

\usepackage{enumitem}
\usepackage{algpseudocode}
\usepackage{caption}
\usepackage{subcaption}
\usepackage{placeins}
\usepackage{multicol}

\begin{document}
\thispagestyle{empty}

\Uebung{6}{7}{Simon Thelen}{18. November 2021}  % FIXME: Blattnummer, Datum, Zeit

%%%%%%%%%%%%%%%%%%%%%%%%%%%%%%%%%%%%%%%%%%%%%%%%%%%%%%%%%%%%%%%%%%%%%%

\ifcsdef{showLoesung}{
\textbf{Bitte beachten Sie:} Die Lösungen können trotz sorgfältiger Prüfung Fehler enthalten.
Bei Fragen oder Unklarheiten kontaktieren Sie bitte den Tutor oder Dozenten in Tutorien, Übungen oder nach Vorlesungen.
}{}


\begin{aufgabe}{1}{Binäre, verkettete Bäume}
    \begin{enumerate}
        \item Gegeben sei ein binärer, verketteter Baum mit Preorder-Darstellung (a, b, c, d, e, f, g, h, i) und Inorder-Darstellung (c, d, b, e, a, h, g, i, f).
        Wie sieht der Baum aus?
        \item Angenommen Sie haben die Preorder- und die Postorder-Darstellung eines binären Baumes gegeben.
        Können Sie (wie bei der vorherigen Teilaufgabe) den Baum eindeutig rekonstruieren?
        Falls ja, begründen Sie dies.
        Falls nein, geben Sie ein Gegenbeispiel an.
        \item Betrachten Sie folgende Implementierung eines Inorder-Durchlaufs über einen binären, verketteten Baum:
        \begin{lstlisting}[language=c++]
void inorder(Node *node) {
    if (node->left != NULL) inorder(node->left);
    cout << node->value << " ";
    if (node->right != NULL) inorder(node->right);
}
        \end{lstlisting}
        Begründen Sie, dass die Laufzeit dieser Funktion für einen Baum mit $n$ Elementen $O(n)$ beträgt, indem Sie einen Zusammenhang zwischen Kanten des Baumes und den rekursiven Aufrufen des Algorithmus herstellen.   
        % \item
        % Implementieren Sie einen Preorder-Durchlauf über einen binären, verketten Baum (in Pseudocode oder einer Programmiersprache Ihrer Wahl), der auf Rekursion verzichtet, aber stattdessen einen Stack verwendet.
        \item 
        Geben Sie einen Algorithmus in Pseudocode oder einer Programmiersprache Ihrer Wahl an, welcher einen binären, verketteten Baum schichtweise ausgibt.
        Also zunächst den Wurzelknoten, dann die Kindknoten der Wurzel von links nach rechts, dann alle Enkelknoten der Wurzel von links nach rechts und so weiter.
        \begin{description}
            \item[Tipp:] Verwenden Sie eine Queue.
        \end{description}
        \item
        Der Abstand von zwei Knoten in einem Baum ist die Anzahl an Kanten, die mindestens traversiert werden müssen, um von einem Knoten zum anderen zu gelangen.
        \emph{Beispiel:} Die Knoten 3 und 13 im Baum aus Aufgabe 2\ref*{insert_delete} haben Abstand 6.

        Geben Sie einen Algorithmus in Pseudocode oder einer Programmiersprache Ihrer Wahl an, welcher einen binären, verketteten Baum als Eingabe erhält und den größten Abstand von zwei Knoten im Baum liefert.
        Eine einfache Lösung hat Laufzeit $O(n^2)$.
        Schaffen Sie es sogar in $O(n)$?
    \end{enumerate}
\end{aufgabe}
\begin{loesung}
    \begin{enumerate}
        \item \ \\
        \begin{figure}[h!]
            \centering
            \includegraphics[width=0.3\textwidth]{img/1a}
        \end{figure}
        \FloatBarrier
        \item Wenn nur Preorder- und Postorder-Darstellung gegeben ist, ist eine eindeutige Rekonstruktion im Allgemeinen nicht möglich.
        Das kleinste Gegenbeispiel sind die folgenden, beiden Bäume:
        \begin{figure}[h!]
            \centering
            \begin{subfigure}[b]{0.23\textwidth}
                \centering
                \includegraphics[width=\textwidth]{img/1b_1}
            \end{subfigure}
            \begin{subfigure}[b]{0.23\textwidth}
                \centering
                \includegraphics[width=\textwidth]{img/1b_2}
            \end{subfigure}
        \end{figure}
        \FloatBarrier
        Beide haben Preorder-Darstellung (1, 2) und Postorder-Darstellung (2, 1).

        Das gleiche gilt für andere entartete Bäume (Bäume, bei denen jeder Knoten maximal einen Nachfolger hat).
        Nummeriert man die Knoten eines entarteten Baumes mit $n$ Knoten von Wurzel zum Blatt durch, hat der Baum die Preorder-Darstellung (1, 2, $\ldots$, $n - 1$, $n$) und Postorder-Darstellung ($n$, $n - 1$, $\ldots$, 2, 1).
        Das gilt für alle entarteten Bäume mit gleich vielen Knoten, obwohl es insgesamt $2^n$ solcher Bäume gibt.

        \item
        Jeder Binärbaum ist zusammenhängend.
        Das heißt, alle Knoten des Baumes werden erreicht.
        Jeder Binärbaum ist außerdem zyklenfrei.
        Das bedeutet, dass kein Knoten doppelt untersucht wird (es gibt keine Kante, die nach oben im Baum führt).
        Daraus folgt, dass für jede der $n - 1$ Kanten des Baumes genau ein rekursiver Aufruf von \texttt{inorder} durchgeführt wird.
        Dazu kommt noch der initiale Aufruf auf die Wurzel des Baumes.
        Macht insgesamt $n$ Aufrufe.
        Jeder Aufruf benötigt offensichtlich nur konstante Laufzeit.
        Die Laufzeit eines Inorder-Durchlaufs beträgt daher $O(n)$.

%         Angenommen, die Implementierung sähe aus:
%         \begin{lstlisting}[language=c++]
% void inorder(Node *node) {
%     if (node == NULL) return;
%     inorder(node->left);
%     cout << node->value << " ";
%     inorder(node->right);
% }
%         \end{lstlisting}
%         Hier werden die Prüfungen auf \texttt{NULL} sozusagen auf die nächsttiefere Rekursionsebene verschoben.
%         Die Laufzeit ist augenscheinlich größer, da mehr rekursive Aufrufe nötig sind.
%         Andererseits ist leicht zu sehen, dass der Algorithmus praktisch identisch zur ursprünglichen Variante ist.
        \item Die Idee ist, zunächst nur den Wurzelknoten in die Queue einzufügen und anschließend über alle Knoten in der Queue zu iterieren.
        Jedes Knoten, der untersucht wird, wird aus der Queue entfernt und ausgeben.
        Anschließend werden alle Nachfolger des Knotens (von links nach rechts) in die Queue eingefügt.
        \begin{lstlisting}[language=c++]
void layerTraversal(Tree *tree) {
    queue q(tree->n);
    if (tree->root != NULL) q.enqueue(tree->root);
    while (!q.isEmpty()) {
        Node *node = q.dequeue();
        cout << node->value << " ";
        if (node->left != NULL) q.enqueue(node->left);
        if (node->right != NULL) q.enqueue(node->right);
    }
}            
        \end{lstlisting}
        Man kann sich überlegen, dass in dem Moment, wenn alle Knoten aus Schicht $k$ abgearbeitet wurden, die Queue genau die Knoten aus Schicht $k + 1$ enthält und zwar von links nach rechts.
        \begin{proof}
            Beweis durch vollständige Induktion über die Anzahl der Schichten
            \begin{description}
                \item[IA] Zu Beginn ist nur der Wurzelknoten in der Queue.
                Im ersten Schleifendurchlauf wird dieser aus der Queue entfernt und seine Nachfolger von links nach rechts eingefügt.
                Die Aussage ist also für Schicht 1 korrekt. 
                \item[IV] Sei die Aussage wahr für $k - 1$.
                \item[IS]
                Dann werden durch das FIFO-Prinzip der Queue alle Knoten aus Schicht $k$ von links nach rechts abgearbeitet, bevor Knoten aus tieferen Schichten untersucht werden.
                Da außerdem für jeden Knoten seine Nachfolger von links nach rechts in die Queue eingefügt werden, liegen, nachdem alle Knoten aus Schicht $k$ untersucht wurden, alle Knoten aus Schicht $k + 1$ von links nach rechts in der Queue.
                Da zu untersuchende Knoten aus der Queue entfernt werden und aufgrund der Induktionsvoraussetzung keine anderen Knoten in der Queue liegen können, liegen, nachdem die Knoten aus Schicht $k$ untersucht wurden, genau die Knoten aus Schicht $k + 1$ in der Queue und zwar von links nach rechts.
            \end{description}
            Nach dem Prinzip der vollständigen Induktion gilt obige Aussage für alle $k$.
        \end{proof}
        Aus der gerade bewiesen Aussage folgt, dass alle Knoten des Baumes schichtweise durchlaufen und somit ausgegeben werden.

        Da jeder Knoten genau einmal in die Queue eingefügt wird und pro Knoten in der Queue konstanter Zeitaufwand besteht, ist die Laufzeit des obigen Algorithmus $O(n)$.

        Der beschriebene Algorithmus entspricht einer Breitensuche (Breadth First Search).
        Diese wird später im Laufe der Vorlesung näher behandelt.

        \item
        Der größte Abstand zweier Knoten entspricht dem längsten (einfachen, also zyklenfreien) Pfad zwischen zwei Knoten im Baum.
        Es soll über alle Knoten iteriert werden und für jeden Knoten die Länge des längsten Pfades im Teilbaum unter diesem Knoten gefunden werden.
        Die Länge des längsten Pfades im Teilbaum unter dem Wurzelknoten ist die gesuchte Antwort.
        Dabei gilt für jeden Knoten, dass der längste Pfad im entsprechenden Teilbaum
        \begin{enumerate}[label=\arabic*.]
            \item sich entweder im Teilbaum unter dem linken Nachfolger befindet,
            \item sich im Teilbaum unter dem rechten Nachfolger befindet
            \item oder durch den Knoten selbst läuft.
        \end{enumerate}
        Die ersten beiden Fällen können durch die Rekursion abgedeckt werden, die beim Durchlaufen des Baumes sowieso nötig ist.
        Bleibt noch der dritte Fall:
        Der längste Pfad durch den Knoten startet im linken Teilbaum und zwar so weit unten wie möglich, läuft hoch bis zum Knoten und dann in den rechten Teilbaum, wieder so tief wie möglich.
        Der Abstand vom einem Knoten zum \glqq{}tiefsten\grqq{} Kind ist entspricht der Höhe des Knotens:
        \begin{lstlisting}[language=c++]
int height(Node *node) {
    if (node == NULL) return -1;
    return 1 + max(height(node->left), height(node->right));
}
        \end{lstlisting}
        Es wird also die Höhe des linken und die des rechten Teilbaumes berechnet.
        Dazu kommen für den vollständigen Pfad durch den betrachten Knoten noch zwei weitere Kanten, nämlich die vom linken Nachfolger zum Knoten und die vom Knoten zu seinem rechten Nachfolger.
        Die Länge des Pfades ergibt sich also durch \texttt{2 + height(left) + height(right)}.
        \begin{lstlisting}[language=c++]
int maxDistance(Node *node) {
    if (node == NULL) return INT_MIN;
    return max(
        max(maxDistance(node->left), maxDistance(node->right)),
        2 + height(node->left) + height(node->right)
    );
}
int maxDistance(Tree *tree) {
    return maxDistance(tree->root);
}
        \end{lstlisting}
    \end{enumerate}
    Die beiden \texttt{height}-Aufrufe dominieren die Laufzeit für jeden Knoten.
    Die Wurzel benötigt den größten zeitlichen Aufwand, da die beiden \texttt{height}-Aufrufe zusammen den gesamten restlichen Baum durchlaufen, was Zeitaufwand $\Theta(n)$ allein für die Wurzel bedeutet.
    Da es insgesamt $n$ Knoten gibt, beträgt die Gesamtlaufzeit für den Algorithmus $O(n^2)$.
    Diese wird in der Praxis auch tatsächlich erreicht (Es ist leicht zu sehen, dass die Laufzeit bei einem entarteten Baum durch die Gauß-Summe gegeben ist).

    Die wesentliche Überlegung, um Gesamtlaufzeit $O(n)$ zu erreichen, ist, dass Höhe und maximale Distanz der Teilbäume parallel berechnet und anschließend zwischengespeichert werden kann, um sich unnötige, doppelte Berechnungen zu sparen.
    \begin{lstlisting}[language=c++]
int maxDistanceLinear(Node *node, int &height) {
    if (node == NULL) {
        height = -1;
        return INT_MIN;
    }
    int leftHeight, rightHeight;
    int result = max(
        maxDistanceLinear(node->left, leftHeight),
        maxDistanceLinear(node->right, rightHeight)
    );
    height = 1 + max(leftHeight, rightHeight);
    return max(result, 2 + leftHeight + rightHeight);
}
int maxDistanceLinear(Tree *tree) {
    int height;
    maxDistanceLinear(tree->root, height);
}
    \end{lstlisting}
    In dieser Lösung wird die maximale Distanz per \texttt{return} zurückgegeben und die Höhe mittels Referenz-Parameter \texttt{height}.
    Pro Knoten ist abgesehen von den rekursiven Aufrufen nur noch konstante Laufzeit nötig.
    Heißt, die Gesamtlaufzeit beträgt nun $O(n)$.
\end{loesung}

\begin{aufgabe}{2}{Binäre, verkettete Suchbäume}
    \begin{enumerate}[label=\alph*)]
        \item Gegeben sei ein binärer, verketteter Suchbaum mit Postorder-Darstellung (15, 28, 12, 33, 41, 37, 30, 50, 42).
        Wie sieht der Baum aus?
        % Quelle: Cormen 12.2-1
        \item Angenommen, Sie suchen den Wert 75 in einem binären, verketteten Suchbaum.
        Welche der folgenden Pfade von der Wurzel bis zum gesuchten Knoten ist nicht möglich: (51, 52, 97, 80, 72, 76, 75) oder (89, 80, 24, 63, 81, 69, 75)?
        Begründen Sie Ihre Entscheidung.
        \item Gegeben seien die Werte (29, 33, 45, 50, 77, 86, 96).
        Geben Sie binäre, verkette Suchbäume mit Höhe 2, 3, 4, 5 und 6 an, die jeweils genau diese Elemente enthalten.
        \item\label{insert_delete} Gegeben sei folgender binärer, verketteter Suchbaum:
        \begin{figure}[h!]
            \centering
            \includegraphics[width=0.3\textwidth]{img/2d}
        \end{figure}
        \FloatBarrier
        \ \\
        Führen Sie nacheinander folgende Operationen durch: \textsc{Delete(15), Delete(1), Delete(9), Insert(1), Insert(12)}.
        Geben Sie alle Zwischenschritte an.
        \item 
        Geben Sie einen Algorithmus in Pseudocode oder einer Programmiersprache Ihrer Wahl an, der einen binären, verketteten Suchbaum sowie zwei Werte $l$ und $h$ als Eingabe erhält und anschließend alle Werte $x$ des Baumes ausgibt, für die $l \leq x \leq h$ gilt.
        Versuchen Sie so gut wie möglich, nur selektiert über den Teil des Baumes zu iterieren, der für die Ausgabe relevant ist.
    \end{enumerate}
\end{aufgabe}

\begin{loesung}
    \begin{enumerate}
        \item \ \\
        \begin{figure}[h!]
            \centering
            \includegraphics[width=0.3\textwidth]{img/2a}
        \end{figure}
        \FloatBarrier

        \item Der zweite Pfad (89, 80, 24, 63, 81, 69, 75) ist nicht möglich.

        Der zweite besuchte Knoten ist 80, was größer als die gesuchte 75 ist.
        Deshalb wurde die Kante zum linken Nachfolger 24 gewählt.
        Da der Baum ein Suchbaum ist, müssen alle Werte im Teilbaum unter 24 kleiner als 80 sein.
        Jedoch wird später auf dem Pfad der Knoten 81 erreicht, was nicht möglich ist, wenn es sich um einen Suchbaum handelt.
        
        \item \ \\
        \begin{figure}[h!]
            \centering
            \begin{subfigure}[b]{0.3\textwidth}
                \centering
                \includegraphics[width=\textwidth]{img/2c_1}
            \end{subfigure}
            \begin{subfigure}[b]{0.3\textwidth}
                \centering
                \includegraphics[width=\textwidth]{img/2c_2}
            \end{subfigure}
            \begin{subfigure}[b]{0.3\textwidth}
                \centering
                \includegraphics[width=\textwidth]{img/2c_3}
            \end{subfigure}
            \begin{subfigure}[b]{0.3\textwidth}
                \centering
                \includegraphics[width=\textwidth]{img/2c_4}
            \end{subfigure}
            \begin{subfigure}[b]{0.3\textwidth}
                \centering
                \includegraphics[width=\textwidth]{img/2c_5}
            \end{subfigure}
        \end{figure}
        \FloatBarrier
        \item \ \\
        \begin{figure}[h!]
            \centering
            \begin{subfigure}[b]{0.3\textwidth}
                \centering
                \includegraphics[width=\textwidth]{img/2d_1}
            \end{subfigure}
            \begin{subfigure}[b]{0.3\textwidth}
                \centering
                \includegraphics[width=\textwidth]{img/2d_2}
            \end{subfigure}
            \\
            \begin{subfigure}[b]{0.3\textwidth}
                \centering
                \includegraphics[width=\textwidth]{img/2d_3}
            \end{subfigure}
            \begin{subfigure}[b]{0.3\textwidth}
                \centering
                \includegraphics[width=\textwidth]{img/2d_4}
            \end{subfigure}
            \begin{subfigure}[b]{0.3\textwidth}
                \centering
                \includegraphics[width=\textwidth]{img/2d_5}
            \end{subfigure}
            \begin{subfigure}[b]{0.3\textwidth}
                \centering
                \includegraphics[width=\textwidth]{img/2d_6}
            \end{subfigure}
            \begin{subfigure}[b]{0.3\textwidth}
                \centering
                \includegraphics[width=\textwidth]{img/2d_7}
            \end{subfigure}
        \end{figure}
        \FloatBarrier

        \item 
        Prinzipiell würde ein Preorder-, Postorder- oder Inorder-Durchlauf genügen, bei dem nur die Werte $x$ ausgegeben werden, die der Bedingung $l \leq x \leq h$ genügen.
        Jedoch wird hierbei potentiell über große Bereiche des Baumes iteriert, die für die Ausgabe irrelevant sind.
        Daher macht es Sinn, vor jedem rekursiven Aufruf zu überprüfen, ob dieser überhaupt nötig ist.
        Wird zum Beispiel im Baum aus der Aufgabenstellung von 2\ref*{insert_delete} nach allen Werten zwischen 5 und 12 gesucht, ergibt es beim Untersuchen des Knotens 14 keinen Sinn, dessen rechten Nachfolger noch zu prüfen, da alle Knoten dieses Teilbaums mindestens 14 und damit sicher größer als 12 sind.
        \begin{lstlisting}[language=c++]
void rangeQuery(Node *node, int l, int h) {
    if (node == NULL) return;
    if (l <= node->value) rangeQuery(node->left, l, h);
    if (l <= node->value && node->value <= h) {
        cout << node->value << " ";
    }
    if (node->value <= h) rangeQuery(node->right, l, h);
}
void rangeQuery(Tree *tree, int l, int h) {
    rangeQuery(tree->root, l, h);
}
        \end{lstlisting}
        In dieser Lösung wird der Baum in Inorder-Reihenfolge durchlaufen.
        Das ist nicht zwangsläufig notwendig.
        Jedoch werden die gesuchten Werte auf diese Weise in aufsteigend sortierter Reihenfolge ausgegeben.

        Ohne Beweis: Wenn der Baum Höhe $h$ hat und $k$ Werte der Such-Bedingung genügen, besitzt obiger Algorithmus Laufzeit $O(h + k)$. Bei balancierten Bäumen (etwa AVL-Bäumen) ist die Laufzeit somit $O(\log n + k)$.
    \end{enumerate}
\end{loesung}


\begin{aufgabe}{3}{AVL-Bäume}
    \begin{enumerate}
        \item Gegeben sei folgender AVL-Baum:
        \begin{figure}[h!]
            \centering
            \includegraphics[width=0.3\textwidth]{img/3a}
        \end{figure}
        \FloatBarrier
        \ \\
        Führen Sie nacheinander folgende Operationen durch: \textsc{Insert(5), Insert(3), Insert(38), Insert(2), Delete(3), Delete(38), Delete(42)}.
        Geben Sie alle Zwischenschritte und Rotationen an.
        % Quelle: Cormen Aufgabe 12.1-1
        \item Begründen Sie oder widerlegen Sie die folgende Aussage:
        \begin{center}
            \emph{Bei jedem }\textsc{Insert}\emph{-Aufruf ist maximal eine (Doppel-)Rotation notwendig}
        \end{center}
        \item Geben Sie ein Beispiel für das Löschen eines Knotens an, bei dem mehr als eine (Doppel-)Rotation benötigt wird.
        \begin{description}
            \item[Tipp:] Sie können zum Beispiel den minimalen AVL-Baum der Höhe 4 verwenden.
        \end{description}
        \item Implementieren Sie die folgenden Operationen, die Sie im Rahmen der Heap-Datenstruktur kennen gelernt haben, mithilfe eines AVL-Baumes (in Pseudocode oder einer Programmiersprache Ihrer Wahl).
        Geben Sie außerdem an, wie Sie die AVL-Datenstruktur aus der Vorlesung erweitern müssen, um die angegebenen Laufzeiten zu erreichen.
        \begin{table}[h!]
            \centering
            \begin{tabular}{|l|l|l|}
            \hline
            \multicolumn{1}{|c|}{\textbf{Operation}}  & \multicolumn{1}{c|}{\textbf{Laufzeit}} & \multicolumn{1}{c|}{\textbf{Beschreibung}}\\ \hline
            \textsc{GetMaximum} & $O(1)$      & Gibt den größten Wert im Baum zurück.\\ \hline
            \textsc{ExtractMax} & $O(\log n)$ & Entfernt den größten Wert aus dem Baum. \\ \hline
            \textsc{Insert}     & $O(\log n)$ & Fügt einen neuen Wert in den Baum ein. \\ \hline
            \end{tabular}
        \end{table}
    \end{enumerate}
\end{aufgabe}

\begin{loesung}
    \begin{enumerate}
        \item \ \\
        \begin{figure}[h!]
            \centering
            \begin{subfigure}[b]{0.23\textwidth}
                \centering
                \includegraphics[width=\textwidth]{img/3a_1}
                \caption*{}
            \end{subfigure}
            \begin{subfigure}[b]{0.23\textwidth}
                \centering
                \includegraphics[width=\textwidth]{img/3a_2}
                \caption*{}
            \end{subfigure}
            \\
            \begin{subfigure}[b]{0.23\textwidth}
                \centering
                \includegraphics[width=\textwidth]{img/3a_3}
                \caption*{}
            \end{subfigure}
            \begin{subfigure}[b]{0.23\textwidth}
                \centering
                \includegraphics[width=\textwidth]{img/3a_4}
                \caption*{Rechtsrotation}
            \end{subfigure}
            \begin{subfigure}[b]{0.23\textwidth}
                \centering
                \includegraphics[width=\textwidth]{img/3a_5}
                \caption*{}
            \end{subfigure}
            \\
            \begin{subfigure}[b]{0.23\textwidth}
                \centering
                \includegraphics[width=\textwidth]{img/3a_6}
                \caption*{}
            \end{subfigure}
            \begin{subfigure}[b]{0.23\textwidth}
                \centering
                \includegraphics[width=\textwidth]{img/3a_7}
                \caption*{Links-Rechts-Rotation}
            \end{subfigure}
            \begin{subfigure}[b]{0.23\textwidth}
                \centering
                \includegraphics[width=\textwidth]{img/3a_8}
                \caption*{}
            \end{subfigure}
            \begin{subfigure}[b]{0.23\textwidth}
                \centering
                \includegraphics[width=\textwidth]{img/3a_9}
                \caption*{}
            \end{subfigure}
            \\
            \begin{subfigure}[b]{0.23\textwidth}
                \centering
                \includegraphics[width=\textwidth]{img/3a_10}
                \caption*{}
            \end{subfigure}
            \begin{subfigure}[b]{0.23\textwidth}
                \centering
                \includegraphics[width=\textwidth]{img/3a_11}
                \caption*{Rechtsrotation}
            \end{subfigure}
            \begin{subfigure}[b]{0.23\textwidth}
                \centering
                \includegraphics[width=\textwidth]{img/3a_12}
                \caption*{}
            \end{subfigure}
            \\
            \begin{subfigure}[b]{0.23\textwidth}
                \centering
                \includegraphics[width=\textwidth]{img/3a_13}
                \caption*{}
            \end{subfigure}
            \begin{subfigure}[b]{0.23\textwidth}
                \centering
                \includegraphics[width=\textwidth]{img/3a_14}
                \caption*{}
            \end{subfigure}
            \begin{subfigure}[b]{0.23\textwidth}
                \centering
                \includegraphics[width=\textwidth]{img/3a_15}
                \caption*{}
            \end{subfigure}
            \begin{subfigure}[b]{0.23\textwidth}
                \centering
                \includegraphics[width=\textwidth]{img/3a_16}
                \caption*{}
            \end{subfigure}
            \\
            \begin{subfigure}[b]{0.23\textwidth}
                \centering
                \includegraphics[width=\textwidth]{img/3a_17}
                \caption*{}
            \end{subfigure}
            \begin{subfigure}[b]{0.23\textwidth}
                \centering
                \includegraphics[width=\textwidth]{img/3a_18}
                \caption*{Links-Rechts-Rotation}
            \end{subfigure}
            \begin{subfigure}[b]{0.23\textwidth}
                \centering
                \includegraphics[width=\textwidth]{img/3a_19}
                \caption*{}
            \end{subfigure}
            \begin{subfigure}[b]{0.23\textwidth}
                \centering
                \includegraphics[width=\textwidth]{img/3a_20}
                \caption*{}
            \end{subfigure}
        \end{figure}
        \FloatBarrier

        \item Die Aussage ist richtig.
        \begin{proof}
        Wann ist nach dem Einfügen eine Rotation nötig?
        Betrachten wir einen Knoten auf dem Pfad vom eingefügten Knoten bis zur Wurzel während der Rückrichtung der Rekursion, also dann, wenn der die AVL-Eigenschaft überprüft wird.
        % Wir nennen den Teilbaum unter dem linken Nachfolger des betrachteten Knoten \glqq{}linker Teilbaum\grqq{} und analog den Teilbaum unter dem rechten Nachfolger \glqq{}rechter Teilbaum\grqq{}.
        Allgemein gilt, dass jeder Teilbaum des Baumes durch das Einfügen eines Knotens entweder gleich hoch bleibt oder um 1 höher wird.
        Angenommen, der linke und der rechte Teilbaum des betrachteten Knotens sind gleich hoch.
        Dann ist nach dem Einfügen im schlimmsten Fall einer der beiden um 1 höher als der andere.
        Somit ist die AVL-Eigenschaft weiterhin erfüllt und keine Rotation nötig.

        Wie sieht es aus, wenn einer der beiden Teilbäume bereits um 1 höher ist der andere?
        Sei dies ohne Beschränkung der Allgemeinheit der linke Teilbaum.
        Dann hat der rechte Teilbaum die Höhe $h$ und der linke $h + 1$.
        Wird der einzufügende Knoten dann im rechten Teilbaum eingehängt, wird dieser im schlimmsten Fall um 1 höher und ist damit ebenfalls $h + 1$ hoch, genau wie der linke.
        Somit ist die AVL-Eigenschaft weiterhin erfüllt und keine Rotation nötig.

        Wenn der Knoten allerdings im linken Teilbaum eingehängt wird, kann es sein, dass dieser danach $h + 2$ hoch ist und damit 2 höher als der rechte.
        In diesem Fall muss rotiert werden.
        Durch die Rotation sinkt die Höhe des rechten Teilbaums um 1, während die des linken Teilbaums um 1 steigt.
        Dies gilt sowohl bei einer Rechts- als auch einer Links-Rechts-Rotation (vergleiche Vorlesungsunterlagen).
        Beide Teilbäume haben nach der Rotation also Höhe $h + 1$.
        Da vorher der linke Teilbaum $h + 1$ hoch war und der rechte $h$, ändert sich die gesamte Höhe des Teilbaums unter dem betrachteten Knoten nicht.
        Somit ändern sich die Höhen für alle Teilbäume unter Knoten, die auf dem Pfad vom betrachteten Knoten aus bis zur Wurzel liegen, nicht.
        Die Knoten auf diesem Pfad sind jedoch genau die, bei denen die AVL-Eigenschaft (in der Rückrichtung der Rerkursion) noch überprüft wird.
        Da die AVL-Eigenschaft vor dem Einfügen überall im Baum erfüllt war und sich die Höhen der Teilbäume durch das Einfügen nicht ändert, muss diese bei diesen Knoten auch weiterhin erfüllt sein.
        Somit ist nach der ersten Rotation keine weitere nötig.
        \end{proof}

        \item In folgendem minimalen AVL mit Höhe 4 führt das Löschen des Knotens 1 zu zwei Rotationen einmal beim Knoten 2 und einmal beim Knoten 5:
        \begin{figure}[h!]
            \centering
            \begin{subfigure}[b]{0.23\textwidth}
                \centering
                \includegraphics[width=\textwidth]{img/3c_1}
                \caption*{}
            \end{subfigure}
            \begin{subfigure}[b]{0.23\textwidth}
                \centering
                \includegraphics[width=\textwidth]{img/3c_2}
                \caption*{}
            \end{subfigure}
            \begin{subfigure}[b]{0.23\textwidth}
                \centering
                \includegraphics[width=\textwidth]{img/3c_3}
                \caption*{}
            \end{subfigure}
            \begin{subfigure}[b]{0.23\textwidth}
                \centering
                \includegraphics[width=\textwidth]{img/3c_4}
                \caption*{Rechtsrotation}
            \end{subfigure}
        \end{figure}
        \FloatBarrier
        \item
        Sowohl Einfügen als auch Löschen klappt in einem AVL-Baum in $O(\log n)$. 
        Die wesentliche, noch fehlende Funktionalität ist das Finden des Maximums, was sich jedoch leicht (ebenfalls in $O(\log n)$) implementieren lässt:
        \begin{lstlisting}[language=c++]
Node *avlTree::findMaximum() {
    Node *node = root;
    while (node != NULL && node->right != NULL) {
        node = node->right;
    }
    return node;
}
        \end{lstlisting}

        Für \textsc{ExtractMax} würde diese Laufzeit reichen, jedoch ist es für $O(1)$ bei \textsc{GetMaximum} nicht ausreichend.
        Daher muss die Baum-Datenstruktur so erweitert werden, dass sie eine Referenz auf den Knoten mit dem größten Wert vorhält (\texttt{maximum}), sodass der entsprechende Wert in konstanter Zeit zurückgegeben werden kann:
        \begin{lstlisting}[language=c++]
object avlTree::getMaximum() {
    return maximum->val;
}
        \end{lstlisting}
        \textsc{ExtractMax} und \textsc{Insert} ließen sich dann so implementieren:
        % actual language is c++, not java, but c++ highlights "delete" as a keyword
        \begin{lstlisting}[language=java]
void avlTree::extractMax() {
    delete(maximum);
    maximum = findMaximum();
}
void avlTree::insert(object o) {
    insert(root, o);
    maximum = findMaximum();
}
        \end{lstlisting}
        Somit wären die angegebenen Laufzeiten von $O(\log n)$ pro \textsc{ExtractMax} und \textsc{Insert} garantiert.
        Natürlich kann man beide Operationen noch etwas optimieren.
        So kann man bei \texttt{extractMax} den Knoten im Baum direkt ansteuern, anstatt ihn zu suchen.
        Außerdem kann man bei \texttt{insert} den eingefügten Wert mit dem aktuellen Maximum vergleichen und dieses einfach direkt anpassen, falls der neue Wert größer ist.
        Beide Optierungen haben jedoch keinen Einfluss auf die Laufzeit im $O$-Kalkül.
    \end{enumerate}

\end{loesung}


\end{document}