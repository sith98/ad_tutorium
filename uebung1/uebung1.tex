\documentclass[11pt,a4paper]{article}

\usepackage{gastex}

%\newcommand{\showLoesung}{1} %<---als Schalter
%\newcommand{\showInhalt}{1} %<---als Schalter

\usepackage{alltt,moreverb,amsmath,enumerate}
\usepackage[normalem]{ulem}
\usepackage[T1]{fontenc}
\usepackage{ae,aecompl} %helvet,mathptm
%\usepackage[left=15mm,right=15mm,top=20mm,bottom=20mm]{geometry}
\usepackage[margin=.5in]{geometry}
%\usepackage[latin1]{inputenc} % f�r Linux
\usepackage[utf8]{inputenc} % Umlaute etc. direkt schreiben (unter Windows)
\usepackage[german]{babel}
\usepackage[url]{oth-logoPNG}
%\usepackage{i2sym,i2ams}

\usepackage{tikz}
\usetikzlibrary{arrows,shapes,trees,positioning,automata,decorations.pathreplacing,decorations.pathmorphing}
\usepackage{tkz-graph}
\usepackage{color}

\usepackage{longtable}
\usepackage{tabularx}

%\usepackage{epic}
%\usepackage{eepic}
\usepackage{comment,ifthen}
\usepackage{../include/todo}

\usepackage[T1]{fontenc}
\usepackage{textcomp}

\usepackage{listings}                   % Listings in Core-Erlang und Maude
\usepackage{lstmisc}

\usepackage{epic}                       % Bildbefehle (picture)
%\usepackage{eepic}                      % erweiterte Bildbefehle

\usepackage{bbm}                        % Mengensymbole (N,C,R,B)
\usepackage{latexsym}                   % zusaetzliche Mathesymbole
\usepackage{amsmath}                    % Mathepaket von der AMS
\usepackage{amstext}
\usepackage{amsfonts}
\usepackage{stmaryrd}                   % zusaetzliche Mathesymbole
\usepackage{mathtools}
\usepackage{amsthm}
\usepackage{cancel}

\usepackage{hyperref}
\usepackage{url}                        % Zum Setzen von URLs in typewriter-face

\pagestyle{empty}

\let\epsilon=\varepsilon
\let\phi=\varphi

\frenchspacing

\setlength{\parindent}{0pt}
\setlength{\textwidth}{18.6cm}
\setlength{\textheight}{26.5cm}
\setlength{\hfuzz}{1mm}

%%% Read dates of assignments from file
\usepackage{xparse}
\ExplSyntaxOn
\ior_new:N \g_hringriin_file_stream

\NewDocumentCommand{\ReadFile}{mm}
 {
  \hringriin_read_file:nn { #1 } { #2 }
  \cs_new:Npn #1 ##1
   {
    \str_if_eq:nnTF { ##1 } { * }
      { \seq_count:c { g_hringriin_file_ \cs_to_str:N #1 _seq } }
      { \seq_item:cn { g_hringriin_file_ \cs_to_str:N #1 _seq } { ##1 } }
   }
 }

\cs_new_protected:Nn \hringriin_read_file:nn
 {
  \ior_open:Nn \g_hringriin_file_stream { #2 }
  \seq_gclear_new:c { g_hringriin_file_ \cs_to_str:N #1 _seq }
  \ior_map_inline:Nn \g_hringriin_file_stream
   {
    \seq_gput_right:cx 
     { g_hringriin_file_ \cs_to_str:N #1 _seq }
     { \tl_trim_spaces:n { ##1 } }
   }
  \ior_close:N \g_hringriin_file_stream
 }

\ExplSyntaxOff

\ReadFile{\uebungsabgabe}{../skel/UEBUNGSABGABE.def}

%%% Read subject info from file
\newcommand{\dozent}[1]{\def\DOZENT{#1}}
\newcommand{\tutoren}[1]{\def\TUTOREN{#1}}
\newcommand{\vorlesung}[1]{\def\VORLESUNG{#1}}
\newcommand{\semester}[1]{\def\SEMESTER{#1}}

\InputIfFileExists{../skel/VORLESUNG.def}{\providecommand{\TUTOREN}{}}%
{\typeout{***********}
 \typeout{Warnung: Kein File vorhanden, das die Vorlesung spezifiziert!}
 \typeout{Spezifikation muss daher im Text des Blattes oder ueber die
          Tastatur erfolgen.}
 \typeout{***********}}

\def\Uebung#1#2#3{
  \othLehrstuhlLogo[\DOZENT]
  \begin{center}
	{~\\[-2em]\Large\bf \VORLESUNG}\\[0.5em]
    \LARGE --~Tutorium #1 (Übung #2)~--\\[4mm]
  \
  \normalsize
  \textbf{#3}
    \rule{\textwidth}{0.1pt}\\[1cm]
  \end{center}
}

\def\Hinweis#1{
	{~\\[-3em]\bf Hinweis: }
	\begin{minipage}[t]{16.5cm}
	#1
	\end{minipage}\\[1em]
    \rule{\textwidth}{0.1pt}
}

\def\Tipps#1{
	{~\\[-3em]\bf Tipps: }
	\begin{minipage}[t]{16.5cm}
	#1
	\end{minipage}\\[1em]
    \rule{\textwidth}{0.1pt}
}
  
\def\MyHeader{
  \othLehrstuhlLogo[Prof.~Dr.~rer.~nat.~Carsten~Kern]%[Carsten~Kern,~Stefan~Rieger]
}

\newcommand{\sem}[1]{[\![#1\,]\!]}

\def\aufgabe#1#2{\subsection*{Aufgabe #1 (#2)}\par}
\def\endaufgabe{}

\newenvironment{loesung}{\subsection*{L\"osungsvorschlag:}}{}
\newenvironment{hinweis}{}{}
\ifthenelse{\isundefined{\showLoesung}}{\excludecomment{loesung}}{\pagestyle{plain}\excludecomment{hinweis}}

\newenvironment{tipps}{}{}
\ifthenelse{\isundefined{\showTipps}}{\excludecomment{tipps}}{\excludecomment{hinweis}}

\newenvironment{inhalt}{\subsection*{Kommentar:}}{}
\ifthenelse{\isundefined{\showInhalt}}{\excludecomment{inhalt}}{}

\long\def\Exercise#1#2{\begin{exercise}{#1}#2\end{exercise}}

\def\underbar#1{%
  \setbox0=\hbox{#1}%
  \dimen0=\dp0\relax%
  \dp0=0pt%
  \setbox0=\hbox{\underline{\box0}}%
  \dp0=\dimen0\relax%
  \box0%
  }

\makeatletter
\def\@makeunderbar[#1]#2{\expandafter\def\csname#1\endcsname{\underbar{#2}}}
\def\makeunderbar{\@ifnextchar[{\@makeunderbar}{\@makeunderbar[]}}
\makeatother

\def\T{\mathrm{T}}
\def\P{\mathrm{P}}
\def\CT{\mathrm{CT}}
\def\COp{\mathrm{COp}}

\makeunderbar{Comp}
\makeunderbar{Ops}
\makeunderbar{trans}
\makeunderbar[strans]{s-trans}
\makeunderbar[ntrans]{n-trans}
\makeunderbar{fix}

\def\labelenumi{\alph{enumi})}
\let\<=\langle
\let\>=\rangle

\parindent=0pt
\parskip=1ex

\definecolor{javared}{rgb}{0.6,0,0} % for strings
\definecolor{javagreen}{rgb}{0.25,0.5,0.35} % comments
\definecolor{javapurple}{rgb}{0.5,0,0.35} % keywords
\definecolor{javadocblue}{rgb}{0.25,0.35,0.75} % javadoc
 
\lstset{language=C++,
basicstyle=\ttfamily\footnotesize,
keywordstyle=\color{javapurple}\bf,
stringstyle=\color{javared},
commentstyle=\color{javagreen}\it\bf,
morecomment=[s][\color{javadocblue}]{/**}{*/},
numbers=left,
numberstyle=\tiny\color{gray},
stepnumber=1,
numbersep=10pt,
tabsize=3,
showspaces=false,
showstringspaces=false}

\usepackage{enumitem}

\begin{document}
\thispagestyle{empty}

\Uebung{1}{-}{-}  % FIXME: Blattnummer, Datum, Zeit

%%%%%%%%%%%%%%%%%%%%%%%%%%%%%%%%%%%%%%%%%%%%%%%%%%%%%%%%%%%%%%%%%%%%%%

\begin{aufgabe}{1}{$O$-Notation}
Zeigen Sie, dass:
\begin{enumerate}
    \item $100n + n \log n \neq O(n)$
    \item $13n^2 + 35n + 42 = O(n^2)$, indem Sie die Grenzwert-Definition der $O$-Notation verwenden.
    \item $\operatorname{log}\left(\frac{3^n}{n + 1}\right) = O(n)$
    \item $\sum\limits_{i=1}^{n} 2^i = O(2^n)$
    \item $(n + 4)^5 = \Theta(n^5)$, ohne den Ausdruck auszumultiplizieren.
    \item $\log(3n^4) = \Theta(\log n)$
    \item $n^{2 + \cos(\pi n)} = \Omega(n) = O(n^3)$
    \item $\sum\limits_{i=1}^{n} \frac{n + 1}{2^i - 1} = \Omega(n)$
\end{enumerate}
Zeigen Sie außerdem, dass für beliebige Funktionen $f, g, h \in \mathbb{N} \rightarrow \mathbb{R}^{+}$ gilt:
\begin{enumerate}
    \setcounter{enumi}{8}
    \item $f(n) = O\big(g(n)\big)$ genau dann, wenn $g(n) = \Omega\big(f(n)\big)$
    \item aus $f(n) = O\big(g(n)\big)$ und $g(n) = O\big(h(n)\big)$ folgt $f(n) = O\big(h(n)\big)$ (die $O$-Notation ist transitiv)
\end{enumerate}
\begin{description}
    \item[Hinweis:] Für manche Teilaufgaben ist die geometrische Reihe nützlich: $\sum\limits_{i=0}^n a^i = \frac{1 - a^{n+1}}{1 - a}$ für alle $a \in \mathbb{R} \setminus \{1\}$
\end{description}
\end{aufgabe}


\begin{aufgabe}{2}{Analyse von Algorithmen}
\begin{enumerate}
\item
Implementieren Sie eine Funktion \texttt{duplicates} in einer Programmiersprache Ihrer Wahl, welche ein Array von Zahlen als Eingabe erhält und als Boolean zurück gibt, ob das Array Duplikate (zwei oder mehr Elemente des gleichen Werts) enthält.

Beispiele: $\operatorname{duplicates}(\left[4, 1, 2, 4\right]) = \texttt{true}$;  $\operatorname{duplicates}(\left[4, 3, 2, 1\right]) = \texttt{false}$; $\operatorname{duplicates}(\left[3, 3, 2, 3\right]) = \texttt{true}$

Achten Sie bei Ihrer Implementierung darauf, dass Sie keine zwei Werte des Arrays mehr als einmal vergleichen.

\item
Wie viele Vergleiche (\texttt{==}, \texttt{<}, \texttt{>=}, $\ldots$) benötigt Ihr Programm im Worst Case, abhängig von der Anzahl der Elemente $n$?

\item Schätzen Sie anhand der Anzahl von Vergleichen die Worst-Case-Laufzeit Ihrer Funktion in $O$-Notation ab.
\item Angenommen, Sie wissen, dass die Werte des Eingabearrays aufsteigend sortiert sind.
Wie können Sie Ihr bestehendes Programm verbessern? Welche Laufzeit hat es nun?

\end{enumerate}
\end{aufgabe}

\begin{aufgabe}{3}{Binäre Suche}
Mittels binärer Suche kann effizient überprüft werden, ob ein Array von Zahlen einen bestimmten Wert enthält.
Vorraussetzung ist, dass die Zahlen aufsteigend sortiert sind.
Dafür wird der gesuchte Wert $x$ mit dem Wert $m$ in der Mitte des Arrays verglichen. Hier gibt es drei Fälle:
\vspace{-3mm}
\begin{itemize}
    \item $x = m$: Der gesuchte Wert wurde gefunden. Fertig!
    \item $x < m$: Der gesuchte Wert befindet sich, falls er existiert, links von $m$. Führe die binäre Suche rekursiv auf der linken Hälfte des Arrays aus.
    \item $x > m$: Analog zu Fall 2, nur auf der rechten Hälfte
\end{itemize}
\begin{enumerate}
    \item Implementieren Sie binäre Suche als rekursiven Algorithmus, vergessen sich nicht den Basisfall.
    \item Konvertieren Sie Ihre Implementierung in einen Iterativen Algorithmus.
    \item Geben Sie die Worst-Case-Laufzeit der rekursiven Variante als Rekursionsgleichung an.
\end{enumerate}
    
\end{aufgabe}


\begin{aufgabe}{4}{Registermaschinensimulator}
\begin{enumerate}[label=\alph*)]
    \item Implementieren Sie ein Registermaschinenprogramm, welches eine Zahl $n \in \mathbb{N}$ einliest und alle Fibonacci-Zahlen $\operatorname{fib}(i)$ mit $\operatorname{fib}(i) \leq n$ ausgibt. Bei $n = 20$ soll das Programm also die Zahlen $(1, 1, 2, 3, 5, 8, 13)$ aus\-ge\-ben; Bei Eingabe 1 nur $(1, 1)$.
    \item \textbf{Bonusaufgabe (schwer):} Erweitern Sie Ihren Registermaschinensimulator, sodass dieser auch indirekte Adressierung untersützt.
    Fügen Sie dafür folgende Befehle hinzu:
    \begin{table}[h!]
        \centering
        \begin{tabular}{|l|l|}
        \hline
        \textbf{Befehl} & \textbf{Semantik} \\ \hline
        \texttt{LDI} & $f(0) = f\big(f(\mathrm{adresse})\big)$ \\ \hline
        \texttt{STI} & $f\big(f(\mathrm{adresse})\big) = f(0)$ \\ \hline
        \end{tabular}
    \end{table}

    Implementieren Sie anschließend den Sieb des Eratosthenes auf der Registermaschine.
    Das heißt, schreiben Sie ein Programm, welches eine Zahl $n \geq 2$ einliest und alle Primzahlen $p_i$ mit $p_i \leq n$ ausgibt.
\end{enumerate}
\end{aufgabe}

\end{document}
