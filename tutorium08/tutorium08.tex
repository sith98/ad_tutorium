\documentclass[11pt,a4paper]{article}

\usepackage{gastex}
\usepackage{etoolbox}
% \newcommand{\showLoesung}{2} %<---als Schalter
%\newcommand{\showInhalt}{1} %<---als Schalter

\usepackage{alltt,moreverb,amsmath,enumerate}
\usepackage[normalem]{ulem}
\usepackage[T1]{fontenc}
\usepackage{ae,aecompl} %helvet,mathptm
%\usepackage[left=15mm,right=15mm,top=20mm,bottom=20mm]{geometry}
\usepackage[margin=.5in]{geometry}
%\usepackage[latin1]{inputenc} % f�r Linux
\usepackage[utf8]{inputenc} % Umlaute etc. direkt schreiben (unter Windows)
\usepackage[german]{babel}
\usepackage[url]{oth-logoPNG}
%\usepackage{i2sym,i2ams}

\usepackage{tikz}
\usetikzlibrary{arrows,shapes,trees,positioning,automata,decorations.pathreplacing,decorations.pathmorphing}
\usepackage{tkz-graph}
\usepackage{color}

\usepackage{longtable}
\usepackage{tabularx}

%\usepackage{epic}
%\usepackage{eepic}
\usepackage{comment,ifthen}
\usepackage{../include/todo}

\usepackage[T1]{fontenc}
\usepackage{textcomp}

\usepackage{listings}                   % Listings in Core-Erlang und Maude
\usepackage{lstmisc}

\usepackage{epic}                       % Bildbefehle (picture)
%\usepackage{eepic}                      % erweiterte Bildbefehle

\usepackage{bbm}                        % Mengensymbole (N,C,R,B)
\usepackage{latexsym}                   % zusaetzliche Mathesymbole
\usepackage{amsmath}                    % Mathepaket von der AMS
\usepackage{amstext}
\usepackage{amsfonts}
\usepackage{stmaryrd}                   % zusaetzliche Mathesymbole
\usepackage{mathtools}
\usepackage{amsthm}
\usepackage{cancel}

\usepackage{hyperref}
\usepackage{url}                        % Zum Setzen von URLs in typewriter-face

\pagestyle{empty}

\let\epsilon=\varepsilon
\let\phi=\varphi

\frenchspacing

\setlength{\parindent}{0pt}
\setlength{\textwidth}{18.6cm}
\setlength{\textheight}{26.5cm}
\setlength{\hfuzz}{1mm}

%%% Read dates of assignments from file
\usepackage{xparse}
\ExplSyntaxOn
\ior_new:N \g_hringriin_file_stream

\NewDocumentCommand{\ReadFile}{mm}
 {
  \hringriin_read_file:nn { #1 } { #2 }
  \cs_new:Npn #1 ##1
   {
    \str_if_eq:nnTF { ##1 } { * }
      { \seq_count:c { g_hringriin_file_ \cs_to_str:N #1 _seq } }
      { \seq_item:cn { g_hringriin_file_ \cs_to_str:N #1 _seq } { ##1 } }
   }
 }

\cs_new_protected:Nn \hringriin_read_file:nn
 {
  \ior_open:Nn \g_hringriin_file_stream { #2 }
  \seq_gclear_new:c { g_hringriin_file_ \cs_to_str:N #1 _seq }
  \ior_map_inline:Nn \g_hringriin_file_stream
   {
    \seq_gput_right:cx 
     { g_hringriin_file_ \cs_to_str:N #1 _seq }
     { \tl_trim_spaces:n { ##1 } }
   }
  \ior_close:N \g_hringriin_file_stream
 }

\ExplSyntaxOff

\ReadFile{\uebungsabgabe}{../skel/UEBUNGSABGABE.def}

%%% Read subject info from file
\newcommand{\dozent}[1]{\def\DOZENT{#1}}
\newcommand{\tutoren}[1]{\def\TUTOREN{#1}}
\newcommand{\vorlesung}[1]{\def\VORLESUNG{#1}}
\newcommand{\semester}[1]{\def\SEMESTER{#1}}

\InputIfFileExists{../skel/VORLESUNG.def}{\providecommand{\TUTOREN}{}}%
{\typeout{***********}
 \typeout{Warnung: Kein File vorhanden, das die Vorlesung spezifiziert!}
 \typeout{Spezifikation muss daher im Text des Blattes oder ueber die
          Tastatur erfolgen.}
 \typeout{***********}}

\def\Uebung#1#2#3{
  \othLehrstuhlLogo[\DOZENT]
  \begin{center}
	{~\\[-2em]\Large\bf \VORLESUNG}\\[0.5em]
    \LARGE --~Tutorium #1 (Übung #2)~--\\[4mm]
  \
  \normalsize
  \textbf{#3}
    \rule{\textwidth}{0.1pt}\\[1cm]
  \end{center}
}

\def\Hinweis#1{
	{~\\[-3em]\bf Hinweis: }
	\begin{minipage}[t]{16.5cm}
	#1
	\end{minipage}\\[1em]
    \rule{\textwidth}{0.1pt}
}

\def\Tipps#1{
	{~\\[-3em]\bf Tipps: }
	\begin{minipage}[t]{16.5cm}
	#1
	\end{minipage}\\[1em]
    \rule{\textwidth}{0.1pt}
}
  
\def\MyHeader{
  \othLehrstuhlLogo[Prof.~Dr.~rer.~nat.~Carsten~Kern]%[Carsten~Kern,~Stefan~Rieger]
}

\newcommand{\sem}[1]{[\![#1\,]\!]}

\def\aufgabe#1#2{\subsection*{Aufgabe #1 (#2)}\par}
\def\endaufgabe{}

\newenvironment{loesung}{\subsection*{L\"osungsvorschlag:}}{}
\newenvironment{hinweis}{}{}
\ifthenelse{\isundefined{\showLoesung}}{\excludecomment{loesung}}{\pagestyle{plain}\excludecomment{hinweis}}

\newenvironment{tipps}{}{}
\ifthenelse{\isundefined{\showTipps}}{\excludecomment{tipps}}{\excludecomment{hinweis}}

\newenvironment{inhalt}{\subsection*{Kommentar:}}{}
\ifthenelse{\isundefined{\showInhalt}}{\excludecomment{inhalt}}{}

\long\def\Exercise#1#2{\begin{exercise}{#1}#2\end{exercise}}

\def\underbar#1{%
  \setbox0=\hbox{#1}%
  \dimen0=\dp0\relax%
  \dp0=0pt%
  \setbox0=\hbox{\underline{\box0}}%
  \dp0=\dimen0\relax%
  \box0%
  }

\makeatletter
\def\@makeunderbar[#1]#2{\expandafter\def\csname#1\endcsname{\underbar{#2}}}
\def\makeunderbar{\@ifnextchar[{\@makeunderbar}{\@makeunderbar[]}}
\makeatother

\def\T{\mathrm{T}}
\def\P{\mathrm{P}}
\def\CT{\mathrm{CT}}
\def\COp{\mathrm{COp}}

\makeunderbar{Comp}
\makeunderbar{Ops}
\makeunderbar{trans}
\makeunderbar[strans]{s-trans}
\makeunderbar[ntrans]{n-trans}
\makeunderbar{fix}

\def\labelenumi{\alph{enumi})}
\let\<=\langle
\let\>=\rangle

\parindent=0pt
\parskip=1ex

\definecolor{javared}{rgb}{0.6,0,0} % for strings
\definecolor{javagreen}{rgb}{0.25,0.5,0.35} % comments
\definecolor{javapurple}{rgb}{0.5,0,0.35} % keywords
\definecolor{javadocblue}{rgb}{0.25,0.35,0.75} % javadoc
 
\lstset{language=Java,
basicstyle=\ttfamily\footnotesize,
keywordstyle=\color{javapurple}\bf,
stringstyle=\color{javared},
commentstyle=\color{javagreen}\it\bf,
morecomment=[s][\color{javadocblue}]{/**}{*/},
numbers=left,
numberstyle=\tiny\color{gray},
stepnumber=1,
numbersep=10pt,
tabsize=3,
showspaces=false,
showstringspaces=false}

\usepackage{enumitem}
\usepackage{algpseudocode}
\usepackage{caption}
\usepackage{subcaption}
\usepackage{placeins}
\usepackage{multicol}
\usepackage{tikz}
\usetikzlibrary{shapes}

\begin{document}
\thispagestyle{empty}

\Uebung{8}{9}{Simon Thelen}{2. Dezember 2021}  % FIXME: Blattnummer, Datum, Zeit

%%%%%%%%%%%%%%%%%%%%%%%%%%%%%%%%%%%%%%%%%%%%%%%%%%%%%%%%%%%%%%%%%%%%%%

\ifcsdef{showLoesung}{
\textbf{Bitte beachten Sie:} Die Lösungen können trotz sorgfältiger Prüfung Fehler enthalten.
Bei Fragen oder Unklarheiten kontaktieren Sie bitte den Tutor oder Dozenten in Tutorien, Übungen oder nach Vorlesungen.
}{}

\begin{aufgabe}{1}{Hashing}
    \begin{enumerate}
        \item Gegeben sei eine Hashtabelle mit $m = 11$, die Hashing mit Verkettung verwendet.
        Welchen Inhalt hat die Hashtabelle, wenn Sie die folgenden Werte einfügen, unter Verwendung der Divisionsmethode sowie unter Verwendung der Multiplikationsmethode (mit $x = \frac{\sqrt{5} - 1}{2}$):
        92, 18, 17, 26, 71, 56, 4, 29?
        % 14, 55, 92, 18, 17, 26, 71, 56, 4, 29?
        \item Gegeben sei eine Hashtabelle mit $m = 13$ und Hashing mittels offener Adressierung.
        Gegeben seien folgende Hashfunktionen:
        \begin{enumerate}[label=\roman*)]
            \item Lineares Sondieren: $h_1(s, i) = (s + i) \bmod m$
            \item Quadratisches Sondieren: $h_2(s, i) = (s + \frac{1}{2} (i^2 + i)) \bmod m$
            \item Doppeltes Hashing: $h_3(s, i) = (s + i \cdot h'_3(s)) \bmod m$, $h'_3(s) = 1 + s \bmod 7$
        \end{enumerate}
        Fügen Sie jeweils unter Verwendung der drei Hashfunktionen die folgenden Werte in die Hashtabelle ein:
        24, 11, 50, 37, 40.
        Geben Sie jeweils den Inhalt der Tabelle nach dem Einfügen aller Werte an.
        \item
        Gegeben sei eine Hashtabelle mit $m = 77$.
        Begründen Sie jeweils für folgende Hashfunktionen, warum sie bei Verwendung von offener Adressierung nicht geeignet sind:
        \begin{enumerate}[label=\roman*)]
            \item $h_1(s, i) = (s + 7i) \bmod{m}$.
            \item $h_2(s, i) = (s + h_2(s)) \bmod{m}$, $h_2(s) = s \bmod 5$
            \item $h_3(s, i) = (s + h_2(s)) \bmod{m}$, $h'_2(s) = 1 + s \bmod{13}$
        \end{enumerate}
        \begin{description}
            \item[Zur Erinnerung:] $\big(h(s, 0), h(s, 1), \ldots, h(s, m - 1)\big)$ soll eine Permutation der Menge $\{0, 1, \ldots, m - 1\}$ sein.
        \end{description}
        % \item Gegeben sei eine Hashfunktion $\hat{h} : U \rightarrow \{0, 1, \ldots m - 1\}$ sowie $h(s, i) = (\hat{h}(s) + 2i) \bmod{m}$.
        % Welche Eigenschaft muss $m$ haben, damit $h$ eine geeignete Hashfunktion für Hashing mit offener Adressierung ist?
    \end{enumerate}
\end{aufgabe}

\begin{aufgabe}{2}{Kollisionen}
    Die folgenden Aufgaben verwenden stets Hashing mit Verkettung.
    Außerdem können Sie davon ausgehen, dass die verwendeten Hashfunktionen Werte gleichmäßig auf alle Plätz verteilen (also jeder Platz der Tabelle gleich wahrscheinlich ist).
    \begin{enumerate}
        \item In eine Hashtabelle mit $m = 100$ werden $50$ Werte eingefügt.
        Wie wahrscheinlich ist es, dass die ersten $10$ Plätze unbesetzt bleiben?
        \item Gegeben sei eine Hashtable mit $m = 14$. Wie viele Werte müssen eingefügt werden, bis im Durchschnitt insgesamt 2 Kollisionen zu erwarten sind?
        Wie viele sind es bei $m = 95$?
        \item Wie viele Werte dürfen bei $m = 14$ und bei $m = 95$ maximal eingefügt werden werden, damit die Wahrscheinlichkeit, mindestens eine Kollision zu erhalten, höchstens 10\% beträgt?
        \item Wie viele Elemente müssen in eine leere Hashtabelle mit $m = 10$ eingefügt werden, damit jeder Bucket im Durchschnitt ein Element enthält? Gehen Sie gemäß des \glqq{}5-Punkte-Plans\grqq{} aus der Vorlesung vor.
    \end{enumerate}
\end{aufgabe}

\begin{aufgabe}{3}{Hashtabellen in der Praxis}
    \begin{enumerate}
        % \item Gegeben sei eine Hashtabelle mit $m$ Buckets (mit Verkettung).
        % Angenommen, Sie achten bei jedem Einfügen eines neuen Wertes darauf, dass die Liste des entsprechenden Buckets sortiert bleibt.
        % Welche durchschnittliche Laufzeit erhalten Sie für Suchen, Einfügen und Löschen abhängig von der Anzahl der Elemente in der Hashmap ($n$) und $\alpha = n / m$.
        \item Nennen Sie zwei Vorteile und zwei Nachteile von Hashtabellen als Such-Datenstruktur im Vergleich zu balancierten Suchbäumen.
        \item Nennen Sie zwei Vorteile und zwei Nachteile von Hashing mit offener Adressierung in der Praxis im Vergleich zu Hashing mit Verkettung.
%         \item In der Praxis werden häufig Schlüssel-Wert-Paare in Hashtabellen verwaltet.
%         Solche Tabellen unterstützen meist folgende Operationen:
%         \begin{lstlisting}[language=c++]
% void insert(key, value);
% void remove(key);
% bool contains(key);
% value getValue(key);
%         \end{lstlisting}
%         Beschreiben Sie knapp, wie diese Operationen auf Basis der Operationen aus der Vorlesungen in durchschnittlich konstanter Laufzeit (bei kleinem Belegungsfaktor) umgesetzt werden können.
        \item Um Strings in Hashtabellen verwalten zu können, muss zunächst jede Zeichenkette in eine natürliche Zahl konvertiert werden, auf die eine Hashfunktion angewendet werden kann.
        Eine Möglichkeit dies zu tun, ist es, die einzelnen (ASCII-)Werte des Strings einfach aufzuaddieren.
        Nennen Sie ein mögliches Problem dieses Verfahrens und machen Sie einen Vorschlag, wie man dieses Problem lösen könnte.
    \end{enumerate}
\end{aufgabe}

\begin{aufgabe}{4}{Anwendung von Hashtabellen}
    Gegeben sei ein Array $(a_1, a_2, \ldots, a_n)$ mit $a_i \in \mathbb{Z}$ für $1 \leq i \leq n$ sowie eine Zahl $s \in \mathbb{Z}$.
    Gesucht sind zwei Elemente des Arrays $a_i$ und $a_j$ ($i \neq j$) mit $a_i + a_j = s$ (vgl. Aufgabe~2 aus Übungsblatt~5).
    Implementieren Sie einen Algorithmus mit durchschnittlicher Laufzeit $\Theta(n)$ in Pseudocode oder einer Programmiersprache Ihrer Wahl, welcher überprüft, ob solche Elemente existieren, und, falls ja, $i$ und $j$ ausgibt.
    Verwenden Sie eine Hash\-ta\-bel\-le~(z.B. \texttt{unordered\_map} in C++ oder \texttt{HashMap} in Java).
\end{aufgabe}

\begin{aufgabe}{5}{Universelles Hashing}
    \begin{enumerate}[label=\alph*)]
        \item Gegeben sei eine Hashtabelle mit $m = 23$, welche fürs die Divisionsmethode verwendet.
        Geben Sie $n$ Werte an, die, eingefügt in die Hashtabelle, dafür sorgen würden, dass die Suchoperation im Erfolgsfall durchschnittlich Laufzeit $O(n)$ benötigt.
        \item \label{universal}
        Sei $p = 5$, $m = 3$ und $U = \{  1, 2, \ldots, p - 1\}$.
        Sei nun $H_{p,m} = \{h_{a,b}(s) = ((as + b) \bmod{p}) \bmod{m} \mid a, b \in U\}$ eine Menge von Hashfunktionen über dem Universum $U$, aus der zufällig mit gleicher Wahrscheinlichkeit eine Funktion $h$ ausgewählt wird.
        Mit welcher Wahrscheinlichkeit ist $h(1) = h(3)$?
        \begin{description}
            \item[Tipp:] Probieren Sie alle möglichen Werte für $a$ und $b$ aus. Für welche Werte erhalten Sie eine Kollision?
        \end{description}
        \item \label{universal2}
        Schreiben Sie ein Programm in einer Programmiersprache Ihrer Wahl, welches $p$ und $m$ als Eingabe erhält und $\max\{P\big(h(s_1) = h(s_2)\big) \mid s_1, s_2 \in U, s_1 \neq s_2\}$ berechnet, unter der Voraussetzung, dass $h$ wie in Teilaufgabe \ref*{universal} aus $H_{p,m}$ ausgewählt wird.
        \item 
        Wenden Sie Ihr Programm aus Teilaufgabe \ref*{universal2} auf die Werte $p$ und $m$ aus Teilaufgabe \ref*{universal} an und begründen Sie anhand des Ergebnis Ihres Programms, dass $H_{5, 3}$ universell ist.  
    \end{enumerate}
\end{aufgabe}



\end{document}